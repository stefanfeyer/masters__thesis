\chapter{Experiment Design and System Implementation}
\label{chapter:studysetting_conduction}
\section{Experiment Design}
\label{section:study}

\subsection{Independet Variables: Visual Perspectives}
The last chapter pointed out five possible visual perspecitves in a scenario with one teacher and one student, compare figure~\ref{fig:perspectives}. All visual perspectives are worth an investigation and a comparable study with all five visual perspectives is desirable. Though, to reduce complexity and the number of participants\footnote{Due to COVID-19 pandemic}, this work will focus on three visual perspectives.\\
Figure~\ref{fig:perspectives} shows three main classes of visual perspectives: ego-centric, exo-centric and perspectives which contain both. To answer the research question, it is indispensably to examine at least one of each class. The ego-centric VP is unique and though choosen by default. The exo-centric VP can be realised as purely exo-centric or augmented exo-centric. The combination of ego-centric and exo-centric can be realised as ego \& exo-centric or ego \& augmented exo-centric. But before the exo-centric VP and the combination can be chosen, a closer look on the mechanics that makes Motor Learning in VR possible is necessary.

\subsubsection{Excursion: Mechanics for Motor Learning in Virtual Reality}
For teaching movments in Virtual Reality, in the exo-centric visual perspective the following issue arises: the GV can move out of the field of view of the learner by the movement itself. Szenario: the learner and the GV stand side-by-side, the learner sees the GV to the left. The GV now indicates a movement to turn by 90 degrees to the right. As soon as the learner follows this movement, the GV will be located behind the learner after the movement ended. A GV standing behind the learner cannot be seen by the learner.\\
This issue is solved in existing work with either the restriction of movements~\cite{freethrowsimulator,elearningma} or multiple representations of the GV arround the learner~\cite{thaichichua,mythaichicoaches}. The restriction of movements has an strong influence in the task design and is therefore not desirable for the study proposed in this thesis. Consequentially, for exo-centric visual perspectives multiple representations for the GVs on strategic positions arround the learner are necessary.\\
In the ego-centric visual perspective, another issue araises during the teaching of locomotion movements. To understand this issue, two aspects have to be clear before. (1) The nature of an ego-centric GV is to be located inside the learner at any time. (2) A GV indicates movements by moving itself. If the guidance visualisatoin is about to indicate a movement away from the learner, the GV is moving out of the students body. But a GV that is outside of the learners body is no longer ego-centric.\\
A possible soltuion is given by the centricity continuum by Wang and Milgram~\ref{fig:ego-exo-continuum}. Following the nature of the centricity continuum, the tethering distance can be increased by a small ammount and the visual perspective can still be classified as ego-centric. But now araises the question, of how far the tethering distance can be increased, with which the perspective still feels ego-centric, but the indication of the movement is considerable. For simplicity, this distance is further called ego-centric tethering distance (ETD). To determine a reasonable ETD, a formative study was conducted. During this study, one\footnote{A larger study was not possible because of the COVID-19 pandemic} non biased\footnote{The person had no prior knowledge about the system or motor learning.} person was asked to follow movements in the ego-centric visual perspective. The first movement was conducted with an ETD of 5cm. For the following movements the ETD was increased by 5cm each. The subjective assesment of the participant and my observations yielded best for an ETD between 15cm and 30cm. These two values are further called:
\begin{itemize}
	\item[] $ETD_{min}=15cm$
	\item[] $ETD_{max}=30cm$
\end{itemize}
Based on $ETD_{min}$ and $ETD_{max}$ the speed mechanic is developed. The speed mechanic controls the speed of the playback of the GV. At $ETD_{min}$ and below, the animation plays at normal speed. At $ETD_{max}$ the GV stops. Between $ETD_{min}$ and $ETD_{max}$ the animation speed of the GV is linearly interpolated, compare figure~\ref{fig:speed_mechanic}.
\begin{figure}[htb]
	\centering
	\includegraphics[width=\textwidth]{figures/speed_mechanic_chart.png}
	\caption[speed mechanic chart]{speed mechanic chart}
	\label{fig:speed_mechanic}
\end{figure}
The speed-mechanic was evaluated by one person\footnote{Different person than the first one. This person had no prior knowledge about the system nor Motor Learning. A Larger evaluation was not possible because of COVID-19 pandemic.}. The participant followed the GV in the ego-centric visual perspective. Observations showed that the participant could follow the movement at ease. The opinion of the participant about the speed mechanic was very positive.\\
With this short excursion, a reasonable decision for the exo-centric VP and the combination can be chosen.\\

In the ego-centric visual perspective, the learner sees the GV inside the own body. Here, the learner can see the relation of the own body to the GV directly. In the pure exo-centric visual perspective this relation cannot be seen. Thereby, the position of the learner in relation to the GV must be guessed. That, in turn, makes the application of the speed machanic - which is necessary for ego-centric guidance - not possible. A mechanic that is used in all conditions but one could lead to biased data, compare table~\ref{tab:mechanics}.
\begin{table}[htb]
	\centering
	\includegraphics[width=\textwidth]{figures/mechanics_comparison.png}
	\caption[Mechanics for Motor Learing in Virtual Reality]{Mechanics speed and multiple representations and in which VP they are applied.}
	\label{tab:mechanics}
\end{table}
The mechanic of multiple representations does not influence the validity of the study, because the mechanic would solve an issue that does not exist in the ego-centric perspectives. FUrthermore, any VP  with more than one representation is an exo-centric VP.\\
In the augmented exo-centric VP, a virtual copy of the learner is located inside the exo-centric GV. The copy lets the learner see the relation of the own body to the GV. Furthermore, augmenting the exo-centric GV with the learner is widely used and evaluated in related work~\cite{YouMove,thaichichua}. Consequently, the augmented exo-centric VP will serve as the exo-centric VP.\\
With the ego-centric and exo-centric VP set, the combination can be chosen. In the ego-centric VP the learner has a direct comparison of the own posture to the posture of the GV in the ego-centric VP. In the augmented exo-centric VP the learner has a direct comparison of the own posture and the posture of the GV in the exo-centric VP. To have the direct comparison from the ego-centric VP AND the exo-centric VP, the ego \& augmented exo-centric VP is chosen as the combination. The ego \& augmented exo-centric VP is the true combniation of ego-centric and augmented exo-centric.\\
For simplification, the augmented exo-centric VP will be further called exo-centric VP, and the ego \& augmented exo-centric will be further called ego \& exo-centric VP.\\
The ego-centric VP, exo-centric VP and the ego \& exo-centric VP are the independent variables of the study and form the three study conditions EGO, EXO, EGO \& EXO.

\subsection{Task Design}
Hornb\ae{}k~\cite{hornbaek} identifyed three main typs of tasks in HCI studies: representative tasks, simple tasks and tasks that use task specific hypothesis. RQ1 states the main investigation field is Motor Learning. Motor Learning is strongly related to real-world movements, evidently the study task is a representative task.\\
Real-world tasks that include the handling of physical load, can found in a wide range of activities. For example, a storekeepers job is to clear a palette of cardboard boxes. This task includes to unload the palette, scale the boxes, measure the dimensions of the boxes and finally store them in a rack. Another example is the work at a grinding machine. The worker takes a slug from a shelf and works on it until the sulg becomes a workpiece. After that the workpiece is carried to a measurement instrument to be varified. There are plenty of other examples, but these two already clarify that a tasks which includes the handling of physical load consitst out of the elemental tasks for manual material handling: lift, lower, push, pull, hold.\\
The idea for the study task is to chain these elemental task together to a Unit-Combined-MMH task, that representatively stand for a wide range of tasks that includes the handling of physcal load. To achieve this, several aspects have to be taken into consideration: (a) the artifacts with which the learner will interact, (b) a reasonable task decomposition that allows the investigation of sub-tasks. Furthermore, the study needs a (c) structure. (c) will reveal the necessity of three tasks. These tasks have to be equally (d) complex. This section will subsequently discuss (a-d) and propose the task for the study.

\subsubsection{(a) Artifacts}
A task that include the handling of physical load obviously need a physical load. In real-world tasks, the physical load can be everything a human can handle. The physical load for this task should fulfil the following criteria. First, the load should have a significant weight, that it is actually percepted as a load, but at the same time, any healty person with no previous illneses can handle it without getting injured. Secondly, the physical load should give a enough freedom for interactions. A simple box fulfils the criteria and furthermore has a relation to physical loads of real-world tasks like the handling of parcels. With a physical load, the elemental tasks of lift and lower can be realised by lifting and lowering the box from and to the floor.\\
Push and pull can be realised by pushing and pulling the box on the floor, but it can feel cumbersome. Moreover, in real-world task pushing and pulling a box is made possible in a more ergonomic height if possible, not least for security reasons. To support push and pull, a table is introduced. This table stands representatively for example the grinding machine or a parcel sorting table.\\
Finnaly, the transitions between the elemental tasks have to be supported, to increase the real-world reference. This is achieved by providing a waypoint. The waypoint is a plate on the floor and helps to brings sence in movements. This plate representatively stands for example for a scale or second machine. Walking to a scale or lower a box to the scale on the floor increases the real-world reference more than just an empty place in the room.

\subsubsection{(b) Sub-tasks}
The goal is to create a Unit-Combined-MMH task with the elemental tasks push, pull, lift, lower and hold. The process of desining the task is complex and happend iteratively.\\
In the first approach, was a task with four ocuurences of every elemental task. For lifting the box from the scale and carry the box to the table, obviously a new task type had to be introduced: carry. Because carry is not an elemental task and for simplicity, elemental tasks and newly introduces task types are from now on referred as sub-tasks. This first approach of designing a task revealed an issue: chaining a given amout of sub-tasks together so that the task is still conductable is hard to achieve. To overcome the inflexebility, a new sub-tasks is introduced: walk. Walk means locomotion without the box in hand. With walk, the box can be pushed from one side of the table, and then be pushed from the other side of the table, which achieves flexibility in task design. Otherwise, on push will always follow pull.\\
The next approach was a task that consists out of the sub-tasks push, pull, lift, lower, carry, walk and hold. Each sub-task appeared four times. The task was evaluated with one\footnote{larger studies covid} person. This person had to follow the instructions in the ego-centric VP and exo-centric VP. During the conduction of the task, the person started to look around and correct the own position during the sub-task hold. An interview afterwards showed, that the person thought the GV stopped because his position was too far away from the GV. It became clear, that the speed-mechanic and the sub-task are not compatible. For the participant of the study it is indistinguishable if he/she is to far away from the GV of if it is the sub-task hold. Because of this indistinguishableness, the sub-task hold is excluded from the task as sub-task. However, hold is still part part of the whole tasks: between the transitions of the tasks (for example between lift ends and lower starts) is a small pause which is equivalent to hold, but this sequence is to short to log. Furthermore, hold is part of the sub-task carry, where the box is holded in front of the body. But hold is not a stand alone sub-task and though connot be evaluated isolated.\\
A new task was desinged with the sub-tasks push, pull, lift, lower, carry and walk. During the design special attention was payed to the magnitide of the movements. For example, every push should be equally far. Lift and lower from and to the scale and lift and lower from and to the table are very diffrent in the magnitude. This resulted in two new sub-tasks: pick and place. Pick means to pick up the box from the table, place means to place the box on the talble, while lift and lower the target remained the scale on the floor.\\
The next stage of the task consits out of the sub-tasks push, pull, lift, lower, carry, walk, pick and place. The task was inspected by an professional physiologist. The physiologist was asked to describe the sub-tasks in detail and perform every sub-task ergonomically. The professionals description of the sub-tasks are listed in table~\ref{tab:sub-tasks}. From the performance of the professional physiologist and the description of the sub-tasks could be derived several insights. The sub-tasks push and pull are very similar in their conduction. The same applies for the pairs lift and lower as well as pick and place. This meant for the evaluation, that the variations of movements is nearly halfed, and the possibility of making mistakes is reduced. Example: for push and pull, one foot has to be placed to the back while the other foot remains under the hip. The hands do the same for every push and pull. If in the participant does perform the sub-task intrinscally correct without the perception of the GV, the study wil not meassure the influence of the perspective. To increase the number of sources of error, two new sub-tasks are introduced: turn and fold. Turn means to turn the box by 90 degrees on the table, fold means to tilt the box from one side to another. The difference of the hand movement to push and pull is obvious. The difference for the feet results from the fact, that during turn and fold, the weight of the box remains on the table and the force to apply on the box is significantly lower than during push and pull. This results in a different feet placement, which are hip wide under the hip.\\
With the introduction of turn and fold all sub-tasks are introduced. A new task was created with all 10 sub-tasks. To assess all sub-tasks multiple times, they appear four times per task. The pair lift and lower and the only in magnituded diffrent pair place and pick realate to each other. To be presented equally in the task among the other sub-tasks they should also only be present four times. Because, lift and lower are measured with the RM (6.2) (see next section) lift and lower was decided to appear three times each, and pick and place one time each. Unfortunately, only one time each pick and place meants that all sub-tasks that do not happen at the table had to be conducted in seuquence. To regain flexibility in the task, it was decided that pick and place will occur 2 times each. This results in 34 sub-tasks per task. Table~\ref{tab:sub-tasks} provides an overview over the sub-tasks and their corresponding description, as well as the occurences per task.

\begin{table}[H]
	\centering
	\includegraphics[width=\textwidth]{figures/sub_tasks_definition.png}
	\caption[Description of sub-tasks]{Sub-tasks that appear in every task.}
	\label{tab:sub-tasks}
\end{table}

\subsubsection{(c) Study Structure}
In the study three conditions are compared: EGO, EXO and EGO \& EXO. The main question of this section is how to assign the participant to the independent variables. The key destinction is between within-subject design and bewteen-subject design~\cite{hornbaek}. In the within-subject design the participant would experience all VPs. In the between-subject design, the participants would experience only one VP. Within-subject desings typically can detect the differences between the conditions more precisely, ibid.. Furthermore, within-subject designs need less participants~\footnote{The COVID-19 pandemic makes it hard to find enough participants.} than between subject designs, ibid.. For these reasons, the study is conducted in a within-suubject design. The within-subject design does also have a drawback: the participants gain experience about the task and the conditions during the study. This means that one condition is influence by another condition, which the participant already experienced. Furthermore, there is a asymetrical carry-over effect between the conditions: EGO \& EXO contains condition EGO and EXO and thereby influences EGO and EXO more than EGO and EXO influences EGO \& EXO. To reduce the task related learning effect, three tasks with an nearly equal complexity are created. With this, a participant will face for every condition a different task. The tasks are still similar and the learning effect persits. A reduction of the influence of the learning effect on the outcome can be countered out by counterbalancing the task. Similar to the task, the carry-over effect between the conditions can be also countered by counterbalancing the conditions. Hornb\ae{}k proposes in this case to cross the conditions with the task and use a Greco-Latin square. Three conditions and three tasks in a Greco-Latin square results in blocks of nine participants. This block is depicted in figure~\ref{fig:study_session_plan}. The study should be conducted with at least 4 blocks (4x9=36 participants) \todo{warum?????}.The first session of every study is for acclimatisation and is excluded from evaluation.

\begin{figure}[htb]
	\centering
	\includegraphics[width=0.5\textwidth]{figures/study_session_plan.png}
	\caption[Study structure]{Experiment structure: within-subject desing in a Greco-Latin square.}
	\label{fig:study_session_plan}
\end{figure}

\subsubsection{(d) Equal Task Complexity}
A study participant will face in every condition another task. For the validity of the study it is indispensable that these three tasks have a nearly equal complexity. As described in (b) a task consits of 10 subtasks that occur a specific amount. The main idea to ensure a comparable complexity is to use the sub-tasks for all three tasks in an equal amount. This means, the 34 sub-tasks of task one occure in task two and three, but in a different order. Table~\ref{tab:tasks} lists all three tasks. For every task, the a sub-task number ST1-ST34 is provided. Every sub-task number stands for a sub-task, which comes with a description and the sub-task ID. Reading the description from top to bottom are the instructions the learner recieves from the GV during one condition. The mirror mentioned in the first line is another waypoint, which is necessary for technical reasons and is described in section~\ref{sec:selfperception}.

\begin{table}[H]
	\centering
	\includegraphics[width=\textwidth]{figures/tasks.png}
	\caption[Description of tasks]{tasks}
	\label{tab:tasks}
\end{table}

\subsection{Measures: dependent variables}
\label{sec:measures}
This works aim is to answer the main researchquestion RQ1: How does the visual perspective on a virtual GV influence Motor Learning in Virtual Reality. To answer this research question, the proposed study has to generate data that is able to answer the sub-research questions RQ1.1-4. This section will provide the underlying paradigma to every sub-research question and explain which measures are necessary.\\

\textbf{RQ1.1} How does the visual perspective on a virtual guidance visualisatioon influence movements' accuracy?\\
\textbf{Paradigma:} The more exact the learners movements maches the GV movements, the better the learner could follow the instruction of the GV. For RQ1.1.1 the limbs of the learner and the limbs of the GV are compared. For RQ1.1.2 the box' accuracy is compared. For RQ1.1.3, both are compared and addionally, the current sub-task is taken into consideration. The accuracy can indicate, how the particular movement is suited for the VP.
\begin{itemize}
	\item[] \textbf{RQ1.1.1} How does the visual perspective on a virtual guidance visualisation influence movements' accuracy of the own body?\\
	\textbf{Measures}: (1) Euclidean distance between the learners and GVs hands, feet, head and hip in meters. (2) Angle between learners and GVs the hands, feet, head and hip in degrees.

	\item[] \textbf{RQ1.1.2} How does the visual perspective on a virtual guidance visualisation influence the accuracy of handling physical load?\\
	\textbf{Measures}: (3) Euclidean distance between the learners and GVs phsical load. (4) Angle between the learners and GVs physical load in degrees.

	\item[] \textbf{RQ1.1.3}How does the visual perspective on a virtual guidance visualisation influence sub-tasks' accuracy?\\
	\textbf{Measures}: (1-4), additionally matched to the sub-tasks that is currently performed (5).
\end{itemize}	
(1-4) gives insights to what extend the learner could follow the GV for the whole task. (5) can extract specific sub-tasks for which the learner could follow the GV to a certain extend. For example, in the ego-centric VP the overall accuracy for a task is lower than in the other VPs, but the accuracy for the sub-tasks lift and lower is higher than in other VP. For this example, measure (5) can extract specific sub-tasks that are performed better or worse than in other VPs.\\

\textbf{RQ1.2} Does the visual perspective on a virtual guidance visualisation influence the transfer of ergonomic principles?\\
\textbf{Paradigma:} the more exact the learners RM matches the GVs RM, the better the ergonomic principles could be transfered.\\
\textbf{Measures:} (6) Risk Measurements: (6.1) spine bend in degrees, (6.2) squat distance in meters, (6.3) good base in meters \todo{check names}, (6.4) box near body.\\
Muckell et al.~\cite{muckell} used four RM in their work to assess the perofrmance of the conducted movements. Three of them are assessed in the proposed study, too.
\begin{itemize}
\item[] (6.1) spine bend is defined by the difference in degrees between the straight upward vector and the back of the learner. For all sub-task, spine bend schould be in a certain window. This window is calculated form the teacher, who recorded the movement. Spine bend indicates, if the learner could percept the correct posture of his back.

\item[] (6.2) squat distance is defined by the distance in meters between the feet. For the sub-tasks lift and lower, the squat distance schould be in a certain window. This window is calculated form the teacher, who recorded the movement. For the other sub-tasks, squat distance is not applied, because the knees do not bend in the other sub-tasks. Squat distance indicates, if the learner could percept that he should bend his knees during lift and lower.

\item[] (6.3) good base is defined by the distance in meters between the feet. For the sub-tasks push, pull, turn, fold, lift, lower, pick and place, the squat distance schould be in a certain window. This window is calculated form the teacher, who recorded the movement. Good base indicates, if the learner could percept the correct posture of the feet.
Muckell et al.\cite{muckell} additionally use the RM spine twist in their work. This RM cannot be applied for this study because of the multiple perspective mechanic. The learner has mulitiple GV around and is free of choice which one to look at. The turn of the head implies spine twist. Though, spine twist has a low validity and reliability.

\item[] (6.4) box near body. For the study task design, a profesional physiologist was consulted (compare \todo{section}). During the interview, all movements were described in detail, compare \todo{figure}. During the sub-task carry the box should be as near as possible to the body while the elbows should have bend angle of 90 degrees. Unfortunately, the bend angle could not be determined during a study for technical reasons, see chapter \todo{implementation}. Fortunately, the distance between the box and the body can be determined. Box near body, is defined as the distance in meters between the box and the body of the learner. For the sub-task carry, box near body should be in a certain window. The window is calculated by from the teacher who recorded the movement.
\end{itemize}

RM (6.1-4) are different from accuracy measurements (1-5), because they are independent from the position of the learner and the GV. For example, in the exo-centric VP, a learner cannot percept the correct position where he/she should stand. The learner thereby stands 15cm away from the position he/she should stand. The overall accuracy is thereby lower. But the learner could percept the positioning of his/her feet correctly. In this case the RM (6.3) are fulfiled while the accuracy is biased.\\

\textbf{RQ1.3} How does the visual perspective on a virtual guidance visualisation influence the learner's visual focus?\\
Measures: (7) looking at\\
Paradigma: the learners visual focus is on the object the learner is looking at.\\
The learner interacts with a box and has multiple GVs around and inside the learner. Looking at can give insights on which GV the learner is focusing, the frequency of focus changes and the role of the physical load.

\textbf{RQ1.4} What is the subjective personal preference of the learner for the visual perspectives?\\
Measures: (8) qualitativ data; likert scales, semi-structured interview, digging into occurences.\\
The qualitative data serves not only to investigate the learners personal preference, but can be also serves as triangulation method for measures (1-7).\\

The last measure is the (9) task completion time measured in secondmiliseconds. The speed-mechanic regulates the speed of the animation of the GV. The further the learner is located to the GV, the slower the animation speed of the GV until it stops completely at $EDT_{max}$. The task completion time can gives insights on to what extend the learner could percept the desired position of his/her body in relation to the GV. This measure relates with (1) and can be used for triangulation. Additionally, it is to aspect that the TCT is decreasing from condition to condition, because the participant aclimated. By that, TCT could give insights about the learning effect between the conditions. Finally the study is recorded by video. If during the evaluation arises questions about a specifiv topic, the recordings can be consulted.

\section{\exgo - Design and Implementation}
\label{section:system}
The last section describes an experiment to investigate the influence of the visual perspective on Motor Learning in Virtual Reality. For the conduction of this study, a system is necessary. This system is called \exgo. This section elucidates the development of \exgo. The starting point is the creation of a self-perception of the learner. Section~\ref{sec:selfperception} describes how the learner gets a digital body in the Virtual Reality. After that, in section~\ref{sec:perspectives}, the VPs have to be implemented which in the study serve as the conditions. Section~\ref{sec:recording} deals with the recording of the movements of task 1-3. Section~\ref{sec:limitations}

\subsection{Self-Perception}
\label{sec:selfperception}
There are various options of devices to dive into the Virtual Reality. In the preceeding project report, several devices were evaluated and the decision was made for the Valve Index\footnote{\todo{todo}}, because its refreshrate, screen solution and the possibility to wear glasses underneath the head-mounted display (HMD). To determin the position and orientation of the HMD, the so-called Lighthouse\footnote{lighthouse} is utilised. A Lighthouse is tracking volume spanned by at least two Base-Stations\footnote{base stations} at the opposit corners of the room. To improve the tracking and for the avoidance of untracked areas, e.g. under the table, \exgo uses four base-stations, one for each corner fo the room, to span the Lighthouse.\\

Till now, the learner can move in an empty virtual world seeing nothing. The next step is to replace the empty virtual world with a meaningful environment. For this, the Game Engine Unity 3D\footnote{Unity} is used. In Unity, a basic room was created. Four light yellow walls, a parquet floor and indirectional lighting. The parquet floor serves a purpos: it has a structure with frequent staight lines which makes it later easier to align the artifacts like table and scale. The room is kept simple to not distract the participant of the study.\\

The next step is to add the learners body this room. To achieve this, the body of the learner needs to be tracked. In the preceeding porject report multiple full-body tracking systems were compared. The decision was made for Vive Trackers\footnote{Vive trackers}, because cease of coordinate system matching, the lower latency and less work-intensive calibration process. The learner wears six Vive Tracker in total. Five of them plus the HMD are necessary for the full-body tracking of the learner. The remainder is necessary for RM (6.1), which is later explained in section~\ref{sec:logging}. Two trackers are located on the feet, two trackers are located on the hands. The last tracker is located on the lumbar vertebra (back hip). The trackers are attached to the learner by special Vive Tracker Straps\footnote{\href{https://www.google.com/search?q=vive+tracker+straps}{https://www.google.com/search?q=vive+tracker+straps}, accessed 10.03.2021}.\\
The Lighthouse tracks the Vive Trackers and HMD which send their position to the PC (simplifyed). Here SteamVR\footnote{steam vr} receives the information and forward it to Unity. In Unity the SteamVR Plugin\footnote{\href{https://assetstore.unity.com/packages/tools/integration/steamvr-plugin-32647}{https://assetstore.unity.com/packages/tools/integration/steamvr-plugin-32647}} provides the information in a workable condition. This information is now about to be translated into a rendering of a human-like avatar at the position of the learners body. This requires several steps. First, an avatar is imported. To create the avatar Reallusion Character Creator 3\footnote{\href{https://www.reallusion.com/character-creator/}{https://www.reallusion.com/character-creator/}} were used. To match the gender of the participant, a male and a female character were created, wearing the same clothes. Based on the demographic questionnair, the gender can be set and the participant will see an avatar complying with the participants gender. Secondly, the information about the trackers position and orientation in the tracking volume have to be translated into human movements that meet the learners movement. This is achieved by Inverse Kinematics (IK). IK arises from the field of robotics. A robot arm consits of limbs and joints. Each limb has a specific length and each joint has a specific range of angles to move. The length and the angles are rules. Given an endpoint the robot has to reach with the most outer limb and the rules, the angle of each joint can be calculated to fulfil the goal. This process can be mapped to a human body, too. Unity provides a third-person plugin called FinalIK\footnote{\href{https://assetstore.unity.com/packages/tools/animation/final-ik-14290}{https://assetstore.unity.com/packages/tools/animation/final-ik-14290}} that is capable for the calculations in question. On the one hand, FinalIK is powerful and unrivalled in functionalty compared to other IK tools and though influenced the choice to use Unity for \exgo heavily. On the other hand, to match the needs of the study, extensive adjustments were necessary. The preceeding project report gives a detailed description of the process, while here a summary is depicted. The main task is to transfere the information from SteamVR to FinalIK in a meaningful way, so that FinalIK animates the learners body faithfully.\\
SteamVR registers the Vive Tracker in the order they are switched on. To increase the reliabliity of \exgo, a script was created that assings the tracker by the hardware ID. The tracker are then assigned to the VRIKCalibrationController. The VRIKCalibrationController matches the tracker with the avatar and resizes the avatar to the learners height. FinalIK is constructed to work with controllers in the users hands. In \exgo, the study participant needs the hands to interact with the box, though the controller are replaced with Vive Tracker on the back on the hands. Shifting the reference points of the hands yields in a faithful representation of the learners hands. The feet needed adjustments in a similar way. Finally, VRIK is able to solve the movements. Solving is the process of translating the tracker information into an animated avatar.\todo{pipeline?}\\
VRIK requires a calibration before use. For calibration, the person attached with the trackers needs to perform a T-Pose in certain direction. To ease the calibration process, a mirror is placed in the room. The participant can be asked to look into the mirror and expand the arms, which leads to the correct orientation of the participant during the calibraion process. Immediate with the calibration the mirror dissapears. Beacuse the participant is standing in front of the mirror when the task is about to start anyway, the position in front of the mirror is choosen as the starting point and end point of every task.


\subsection{Artifacts}
The learner is able to see the own body in an empty room. The task include the handling of physical load on a table and a scale. To create the table, the scale and the box which will serve as physical load, they need to be constructed physically and digitally.\\
The first version of \exgo used a cardboard box (27cm x 26cm x 24cm). During the develompent it become clear that the size of the cardboard box was to handy to serve as physical load. To determine a suitable size several boxes of diffrent dimenions were tested. With nine different boxes, a set of sub-tasks were performed. The mayor insight from this test was that the length of the sides of the box should be different, to visually see the direction of the box. Furthermore, the box should be percepted as physical load by having a reasonable size and a certain weight, but not beeing to heavy to not limit the study participants to strong humans, or beeing a threat to the bodies of the participants. The sizes were set for 35cmx30cmx25cm. The mesures  of the box differntiate by 5cm in every dimension. This makes it clear to see the orientation of the box. The final box was constructed from three layerd wood with a strength of 27mm. The resulting weight was 5.8kg. To evaluate the weight of the box, one male person\footnote{A larger evaluation was not possible because of the COVID-19 pandemic. Evaluation with at least one female person is desireable} was asked to perform every the sub-tasks. Observations revealed no incidences that contradict to use the box as physical load. The person rated the weight as "ok". He had no problems to move the box. The box was painted in three high contrast colours: black, white, red. Each opposit side in the same colour. The painting facilitates the visual perception of the orientation of the orientation of the box. The digital pendant of the box is a cube in the same colour and size. To translate the position and orientation of the physical box to the virtual one, a Vive Tracker is attached to the box and fixated with a screw. On the plus side of using a screw is the prevention of any relocation of the tracker. The downside is that tremors caused by placing the box on e.g. the table, are transfered directly into the tracker. This causes the tracker to loose tracking. To interupt the transfere of tremors a shock-absorbing insulation is placed between the tracker and the box.\\
During the development of \exgo an office table (120cm x 70cm x 72cm) was used. The size of the table prooved to be suitable for the task. The digital pendant to the physical table is a cube in the same size and colour of the tabletop. The position and orientation of the table is assessed by a Vive Tracker. Unfortunately, the tracker is placed inside the "working area" on the top of the table. To shift the tracker out of the working area, a new tabletop was constructed. The new tabletop is out of three layared wood, with increased size of 5cm in each dimension. The tracker is attached on the most outer edge, though out of the working area. To prevent tremors passing from the table into the tracker, a shock absorbing insulation is apllied here, too.\\
The last artifact to create is the scale. The scale is a waypoint in the room where the participants perform lift and lower to the ground. The scale is a rectangulat plate of 45cmx45cm, so that the box can be placed on the scale easily. To shift the tracker out of the area where the box will be placed, the plate is extended by 5cm. The tracker is attached to the most outer edge of the extension with a screw and shock-absorbing insulation. The digital pendant is a cube with the same dimensions as the physical plate, excluding the extended area.\\

\subsection{Study Setting}
\begin{figure}[H]
	\centering
	\includegraphics[width=\textwidth]{figures/study_setting.png}
	\caption[study setting]{study setting}
	\label{fig:study_setting}
\end{figure}
Meanwhile, \exgo consists out of a room, avatar representing the learner, tbale, box and scale. This section aims to attain an overview about te alignment of these elements. Figure~\ref{fig:study_setting} shows the real-world room in which the study is conducted. Figure~\ref{fig:learner_positions} depict all positions that are described in the task (compare table~\ref{tab:tasks}). The outer line represents the Lighthouse or tracking volume, which is approximately 400cmx400cm. On the left wall, the mirror is located. In front of the mirror, the starting and end position of the box is seated. Beside the box is position mirror, the start and end point of the learner. The table is placed int the middle of the wall to the left of the mirror. Around the table the positions table center, table right and table left is located. At the opposit wall of the mirror the scale is placed. In front of the scale, is the position scale.
\begin{figure}[htb]
	\centering
	\includegraphics[width=\textwidth]{figures/learner_positions.png}
	\caption[learner positions]{learner positions}
	\label{fig:learner_positions}
\end{figure}

\subsection{Guidance Visualisation}
\label{sec:gv}
The next task is to add the GV to \exgo which the learner can mimic. The GV is an avatar like the learners avatar, with the difference, that the motion og th GV is driven by the pre-recorded tasks 1-3. The recording of the tasks was also performed with \exgo. To use \exgo as a recorder, an copy of all trackers and the HMD is created. This recorder-copy is packed as one GameObject with the trackers as childs. The parent GameObject is recorded during the performance of the movements. For the GV, a similar GameObject as the recorder is created and serves as Input for VRIK. An detailed description is given in the project report. In this section the main points of the process are described: the recording of the movements and the resizing of the GV to the size of the learner.\\
The movements in the task have to be performed ergonomically. The measures to evaluate ergonomic movements are the RM. To serve as a strong baseline, a professional for ergonomic movements should record the movements. Because of the COVID-19 pandemic, all approaches to record the movements by an proffesional failed. Eventually, I myself were trained by a phsiologist and recorded the movements by myself. The final recordings were examined by the physiologist. Overall, the movements were rated by the physiologist as "by and large correct". The back is not always straight or in the correct angle. In task 2 during a push and in task 3 during a pull, the feet are misplaced.\\
With the recording of the tasks at hand, the GV can be animated. For the ego-centric VP it is inevitable, that GV and learner having the exact same size, else the learner cannot percept the GV correctly. Furthermore, the table, box and scale mussn't resize. The solution is to record two sets of object synchonised. The first set contains the objects that have to be resized, namely the GV, the second set contains objects that mussn't be resized, namely the box, table and scale. The recordings were synchonised by a script, the playing of the animations in \exgo, too. The resizing of the GV takes place in three steps. First, the learner and the GV are calibrated. Then the height difference is measured between learner and GV. In the last step, FinalIK is removed from the GV, the animated GameObject containing all trackers is removed, then FinalIK reattached and calibrated. Thus the learner and the GV have the same size.


\subsection{Perspectives}
\label{sec:perspectives}
To fulfil the studies needs, the VPs have to be implemented. 

The ego-centric VP requires besides the learner, one ego-centric GV. The learner needs to stay inside the GV. This is achieved by the speed-mechanic. The distance of the learner is calculated with the help of the tracker at the hip of the learner. The hip is projected to the floor, likewise the GV hip tracker. The projection to the floor is necessary, because the speed-mechanic would apply in the case the GV bends the knees during lift and lower. This restrics the possibility to perform an error by the learner: if the learner does not bend the knees during lift and lower, the GV would stop and remind the learner to bend the knees. To investigate if the learner could percept to bend the knees, the learner must be allowed to do the error. The speed-mechanic relies on the distance between the two projected points on the floor. Additionally, the distance finds application in another functionality. In the ego-centric VP, the learner is loacted inside the teacher. This means, that the viewport of the learner is inside the heade of the GV and let the learner see the inside of the GV head. This leads to distraction by the partly rendered inner head. The solution is to remove the head rendering, if the distance is below $EDT_{max}$ and renitiate the rendering above $EDT_{max}$. The rendering is removed by replacing the materials array of the head with an material array that contains only invisible materials.\\

In the exo-centric VP, four exo-centric GV are located around the learner. The positions of the exo-centric GV were determined after the task was recorded. The difficulty is to determine reasonable positions of the exo-centric GVs. First, in any point in time during the performance of every task, the learner must be able to see a GV by only turning the head. Secondly, the GV and the learner should not move through a table or scale of another GV. The solution to the first part in informed by Chua et al.~\cite{thaichichua}. Chua et al. chose four representation that are in front, behind, left and right. The latter prooved to be impossible, if the exo-centric representations should be near enough to be able to be observed by the learner. This is a limitation of \exgo. The GV need to be shifted to far away from the learner to not cross the artifacts of other GV. In a distance in which no crossing occurs, the movements are barely visible to be mimicked correctly. The exo-centric representations were then placed in a distance, that allows to be observed by the learner,  the GV can access all positions without standing in an artifact at that position and that the learner never crosses a table of another GV. Standing at table center: the GV to the left is shifted by two meters to the left, the GV to the right, two meters to the right. The GV in frton is shifted by 1.5 meters to the front. The GV in the back is shifted 3 meters to the back. Figure~\ref{fig:multireppositions} shows the positions of the exo-centric VP. Additionally, to the exo-centric GVs a virtual copy of the student needs to be rendered. The virtual copy of the learner is shifted by the same values. In the last VP, the ego \& exo-centric VP, the learner has an ego-centric VP, as well as the exo-centric GV with the corresponding virtual copies of the learner. The implementation of copying the learner and the GV and shifting them to the position is rather complex and described in detail in the project report.\\
\begin{figure}[htb]
	\centering
	\includegraphics[width=\textwidth]{figures/positions.png}
	\caption[Positions of exo-centric GVs]{shift calc}
	\label{fig:multireppositions}
\end{figure}

\subsection{Quantitative Data Aquisition}
\label{sec:logging}
\begin{table}[H]
	\centering
	\includegraphics[width=\textwidth]{figures/logging_detail.png}
	\caption[logging detail]{Detailed overview of logs produced by \exgo\ per frame. L: learner, GV guidance visualistion, ED: euclidean distance. *head position and rotation is biased in exo-centric conditions because of multiple GV the L can focus on. **All trackers are logged for backup reasons: after the study is conducted a measurement can become interesting that was not of imporance before. With these values any measurement can be calculated post-study.}
	\label{fig:logging_detail}
\end{table}


Section~\ref{sec:measures} defined the measures taht are necessary to answer the research questions. \exgo must be capable to assess all measures. This section explains how \exgo assesses the measures. An overview about all measures is listed in table~\ref{tab:logging_detail}. Table~\ref{tab:logging_detail} lists the logging ID, a description about what the measurement is measuring, the unit in which the measurement is measured and for which research question the measurement ist assessed. The quantitative data aquisition can be divided in several classes: (i) accuracy measurements (1-5), (ii) ergonomic measurements (6), (iii) focus measurement (7) and (iv) time measurement (9). 

\subsubsection{(i) Accuracy Measurements}
The accuracy measurements assess the discrepancy between the movements of the learner and the movements of the GV. Accuracy measurements are subdivided in distance-based measures and angle-based measure. Distance-based measures rely on the the Euclidean distance between the body parts of the learner and the body parts of the GV. The reference point for the body part is the tracker which is attached to the body part. The body parts are: hip, left hand, right hand, left foot, right foot, head. The distance between the box of the learner and the box of the teacher is an accuracy measurement, too. Likewise the body parts, the distance between the two boxes is the Euclidean distance between the tracker of the box and the tracker of the GV. Please note, the trackers are not visible to learner during the study. The trackers of the GV are the recorded trackers. Angle-based accuracy measutrements assess the discrepancy of orientation between the body parts and box of the learner and the GV, and is measured in angle. The calculation of the angle-based measurements comply with the calculations of the distance-based measurements. Means, the angle between the corresponding trackers are measured. To conclude: distance-based measurements assess the error in the positioning, angle-based measurements assess the error in orientation.

\subsubsection{(ii) Ergonomic Measurements}
The ergonomic measurements are the four risk measurements: support base, squat, upright stance, and hip-box distance. Support base is the distance between the feet. For push and  pull, lift and lower, turn and fold, pick and place each a window in which the distance should be located is defined. The definition of the window is based on the movement of the perons who recorded the movement. The percentage of time the learner is inside the window, is the the outcome of the measurement. For a better understanding imagine the folowing example: the window for push and pull for the support base is 20cm-30cm. During the performance of push the learners feet distance was inside the window for 90 seconds. The whole performance of push lasted 100 seconds. The RM support base yeilds in a score of 90\%. Because the movements were recorded by myself and not by a professional, the window defined on my own movements would me missleading. Because of this predefined error, the definition of a window is renounced.\\
The measurement for squat is the distance of the hip and floor. It indicates if the lerner bended the knees correctly. It is applied in the sub-tasks lift and lower. Calculations of the RM score of squat complies with support base.\\
Upright stance is the measurement of the spine bend is correctly. For upright stance, a additional tracker is applied on the back of the student, above the hip tracker. The anlde of spine bend is the angle between the upright vector and the vector of the upper hip tracker. Upright stance is applied for push and pull, lift and lower, turn and fold.. The bend angle during pick and place depends heavily on the position of the box on the table and thereby varies. Because of this variation, a window cannot be defined for pick and place and thoufh the RM upright stance is not applied to pick and place. Calculations of the RM score of upright stance complies with the calculations of the preceeding RM.\\
Hip-box distance is calculated by the distance between the hip tracker and the box tracker. it is applied for the sub-task carry. The calculations comly with the preceeding RMs. The limitation of hip-box distance is that the measurement is influenced by the circumfence of the torso of the learner. An formative evaluation of hip-box distance was not possible due to the COVID-19 pandemic.

\subsubsection{(iii) Focus Measurement}
The virtual room the learner sees in \exgo is filled with tables, boxes, GVs and virtual copies of the learner. To assess on what the learner is focusing during the movements, every object revieved a name. In every frame, raytracing is performed. The rays' origin is the HDM and expands straight forward. The name of the object first hit by the ray is written into the log file. The name is coded with the position 0-4 (see positions in firgure~\ref{fig:learner_positions}) and an object identifyer (box, table, scale, GV, lerner). A test revealed a systematic error by pointing too high. To correct the discrepancy, colliders of the objects were increased. The tables' and scales' collider hight is increased by 20cm. The box colliders height is doubled. The learners' and GVs' avatar were wraped into an capsule collider with a hight 200cm of and a radius of 30cm. The values were determined by experimentation. To test the values, all sub-tasks were performed and for any point the the object which is hit by the ray is displayed. The displayed name complied with the object in focus in nearly any point in time. Using an eye-tracker would increase the accuracy, but was available.

\subsubsection{(iv) Time Measurement}
The speed-of the animation of the GV is determined by the distance between the learner and the GV. Comparing the time the learnerer needed to perform the task with the time the tasks lasts without the speed-mechanic can draw conclusions about the position of the learner to he GV. The tasks differe in the amount of time to performed. The plain-times without the speed-mechanic are: task 1 takes 172128ms, task 2 189040ms and task 3 176668ms. The substraction of the plain-time from the total time is the additional time the learner needed to perform the movement. The time discrepancy can also be applied to specific sub-tasks. This measurement will mainly used in the evaluation for triangulation. 

\subsection{Qualitative Data Aquisition}
\label{sec:quali_logging}
The qualitative data assess during a study session relies on two questionairs. After each condition, the participant is asked to fill the after session questionaire (appendix A). After all conditions, the participant is interviewd. The guideline of the semi-structured interview is attached in appendix B. The questionaires aim to assess the participants impressions and opinions about the VPs. In the questionaires, a different wording is applied to ease understanding. For exmaple, the GV is called virtual teacher. 

\subsubsection{After Session Questionaire}
The after session questionaire starts with a question about the subjective overall performance of the learner. 
\begin{itemize}
	\item[Q1:] How accurate did your movements comply with the virtual teacher?
	\item[A:] Likert scale from one (very good) to 7 (very poor)
	\item Linked resarch questions: RQ1.1.1-3, RQ1.4
	\item Triangulation for (1-4,6)
\end{itemize}
The answer to this questions gives insights how accurate the participant assesses the performed movements. Furthermore, this question can be used to determine, if the qualitative accuracy comply with the subjective opinion of the participants. The next question aims to assess the users subjective performance for the sub tasks. The participant is asked to fill in a table. Each line represents a sub-task. Each sub-task can be rated on a Likert scale from 1 (very good) to 7 (very poor).
\begin{itemize}
	\item[Q2:] During the task there were several smaller reoccurring movements, like pulling or lifting the box. Please rate these smaller movements, to what extend you could follow to movements.
	\item[A:] Likert scale from one (very good) to 7 (very poor) for each sub-task.
	\item Linked resarch questions: RQ1.1.1-3, RQ1.4
	\item Triangulation for (1-4,5,6)
\end{itemize}
\begin{figure}[H]
	\centering
	\includegraphics[width=0.5\textwidth]{figures/sub-task-rating.png}
	\caption[sub task rating]{sub task rating}
	\label{fig:subtaskrating}
\end{figure}
Beside the objective opition about the participants performance, the answers of this question can be used to compare with the qualitative data. Question three aims to assess the subjective accuracy of the participants body parts.
\begin{itemize}
	\item[Q3:] Please rate to what extend you think you could align your body parts with the teachers body parts. 
	\item[A:] Likert scale from one (very good) to 7 (very poor) for each body part.
	\item Linked resarch questions: RQ1.1.1-3, RQ1.4
	\item Triangulation for (1-4,5,6)
\end{itemize} 
\begin{figure}[H]
	\centering
	\includegraphics[width=0.5\textwidth]{figures/body-parts-acc.png}
	\caption[sub task rating]{sub task rating}
	\label{fig:bodypartsacc}
\end{figure}
This question assesses how good or bad the participant could see the body parts of the GV. The last question is not handed to the participant. It serves as the basis for an semi-structured interview. It gives the possibility to dig into extreme values of the questions answerd before and in any incidences occured during the performance of the task. Furthermore, the participant is encouraged to speek frank about the session. 
\begin{itemize}
	\item[Q4:] (As interview question) Did you have problems to follow the instructions? 
	\begin{itemize}
		\item E.g. because you could not see some body parts?
		\item E.g. bad perception related to the perspective?
		\item Go into extreme values of this questionnaire! 
		\item Address critical incidences!
	\end{itemize}
	\item[A:] Take down of participants statements.
\end{itemize}

\subsubsection{Semi-Structured Interview}
After all three sessions done, the participant is interviewed. The interview is semi-structured. The guideline contains seven main questions, partly with additional hints to dig deeper or to lower the entry threshold for the participant to start reporting.
\begin{itemize}
	\item[Q5:] You saw three visual perspectives: ego-centric, exo-centric and the combination. What do you think about these perspectives?
	\begin{itemize}
		\item entry question, encourage participant to talk frank, address interesting mentions
	\end{itemize}
	\item[] Linked research question: RQ1.4

	\item[Q6:] Prioritise the perspectives by to what extend you think you could follow the movements. (1 best to 3 worst) 	
	\begin{itemize}
		\item Why did you decide for the prioritisation?
	\end{itemize}
	\item[] Linked research question: RQ1.4

	\item[Q7:] Imagine you want to learn a movement in VR. Which perspective would you use for that?
	\begin{itemize}
		\item -	Or would you use a totally different one?
	\end{itemize}
	\item[] Linked research question: RQ1.4

	\item[Q8:] In which of the three perspectives was the easiest to understand? 
	\begin{itemize}
		\item Was there a perspective that confused you?
		\item Was there a perspective you did not understand right away?
		\begin{itemize}
			\item What do you think caused the confusion?
		\end{itemize}
	\end{itemize}
	\item[] Linked research question: RQ1.4

	\item[Q9:] What do you think are the advantages and disadvantages of the perspectives?
	\item[] Linked research question: RQ1.4
	
	\item[Q10:] Could you see some body parts better or worse in the perspectives?
	\begin{itemize}
		\item What about your legs, arms, back?
		\item Could you detect that during lift and lower you should squat?
		\item Could you detect that you should step back during push and pull?
	\end{itemize}
	\item[] Linked research question: RQ1.4
	
	\item[Q11:] Did you miss a feature?
	\begin{itemize}
		\item Dig for improvements for \exgo or experiment design.
	\end{itemize}

	\item[Q12:] (Space to ask for critical incidences, if any occurred.)
\end{itemize}


\subsection{Study Procedure}
\label{sec:procedure}
As soon the participant enters the room, the participant receives a warm welcome to feel comfotable.\\
The process start with a Welcome letter (appendix C), followed by the informed constent and a demographic questionaire. In the mean time, \exgo is set up by choosing the condition, set the gender of the participant as well as the log is configured with the participant ID and task id. After the demographic questionaire, a spoken explanation about what is about to happen is given. Then the trackers are attached to the participant. The calibration process is explained: looking in the mirror and extend the arms. An explanation of the perspective is provided. Then, the first session is started. \exgo gets started, the cameras and screen recording are set up and the participant gets the HMD. The participant is asked to stand in front of the mirror to calibrate. For calibration, the key C is pressed at the PC. To identify the cam recordings a sign is held into the cameras. The task is started with the key S. During the session, attantion is payed to the cable of the HMD to avoid stumble of the participant. Furthermore, the participant is observerd. After the session ended, the HMD is removed. The participant fills in the after session questionaire. Session two and three are conducted likewise. After all three sessions, the trackers are removed. The participant is interviewed. The payment is given and the recipted singed. In the very end, the participant is thanked and said goodby. If appeared, door step talk is appreciated.

\subsection{Limitations}
\label{sec:limitations}
\exgo and the study is designed for a task that includes the handling of physical load. If the outcome can be applied to movements without a physical load is questionable. Furthermore, the exo-centric GV sometimes walk through artifacts (table, scale) of other GV, which can cause confusion to the study participant. The movements are not recorded by an professional, errors in ergonomic movements are possible. Lastly, only a small number of participants participated in the formative tests to evaluate partial aspectives of \exgo. Especially, the hip-box disance is not tested, because multiple persons with diffrent physique would be necessary.