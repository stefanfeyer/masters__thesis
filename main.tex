% HCI Group Konstanz Seminar Template
%
% ATTENTION: This template was designed to fit the most common requirements for different types of reports (e.g., Seminar to the Project or Project Report). However, if you feel the need to add an extra component or layout feature please talk to your advisor. Please do not change anything on this template unless it is explicitly allowed or agreed with your advisor. However, it is allowed to add packages that do not alter the design/layout of this template.
%

\documentclass[11pt, paper=a4, parskip, twoside, open=right]{scrreprt}
\usepackage[T1]{fontenc}
\usepackage[utf8]{inputenc}
\usepackage{hciknseminar}
\usepackage[ngerman, english]{babel} % Spell checking (Using ngerman for german spell checking). The last language is the standard language for the document.
\usepackage{enumitem}
\usepackage{wrapfig}

\bibliography{bibliography} % Reference to bibliography.bib file

% ============= Commands =============
\usepackage{float} % for H, remove if anything breaks

\usepackage{booktabs}
\usepackage{graphicx}
\usepackage{tabularx}


\newcommand{\exgo}{E(x$\mid$g)o}
\newcommand{\todo}[1]{\textcolor{red}{todo: #1}}
\newcommand{\combi}{EGO \& EXO}


% ============= Abstract =============
\abstract{
	Motor learning is a substantial part of life, like in sports, arts or the ergonomic handling of physical load. Motor learning is traditionally done with the help of a teacher. If a teacher is not available, a digital guidance visualisation in Virtual Reality can be consulted. When motor learning is done together with a human teacher, a learner can watch the teacher's movements from an exo-centric visual perspective (third-person perspective). In contrast, in Virtual Reality a guidance visualisation can be seen from the ego-centric visual perspective (first-person perspective), too. The change of the visual perspective on the guidance visualisation influences motor learning. 
	However, the empirical evidence about how the change of the visual perspective influences motor learning is low, especially for full-body movements, for tasks which include the ergonomic handling of a physical load and visual perspectives that utilise ego-centric and exo-centric guidance visualisations simultaneously. Furthermore, the field of ego-centric guidance of locomotion movements is unexplored.
	This master's thesis proposes an experiment to close this research gap. The experiment compares an ego-centric visual perspective, an exo-centric visual perspective and a visual perspective which combines both. The experiment utilises a task which consits out of elemental tasks of handling physical load. For the evaluation of the experiment, accuracy measurements, ergonomic measurements, the learners visual focus and qualitative data is taken.
	The experiment was evaluated with a pilot study and proved to be suitable to generate the data to close the above mentioned research gap. First data indicates that the presence of an ego-centric guidance visualisation positively influences accuracy and attracts the the visual focus of the learner.
}

% ============= Document =============
\begin{document}
	% =========== Title page ============
	\begin{titlepage}
		\begin{center}
			{\LARGE \textbf{Ego OR Exo:\\Comparing Visual Perspectives on Guidance Visualisations for Motor Learning}}
			\\[1cm]
			{\Large \textbf{Masterarbeit}}
			\\[1cm]
			{\Large vorgelegt von}
			\\[0.5cm]
			{\LARGE \textbf{Stefan Paul Feyer}}
			\\[0.5cm]
			{\Large an der}
			\\[0.5cm]
			\includegraphics[width=0.4\textwidth]{figures/unisignet.pdf}
			\\[1cm]
			{\Large \textbf{Sektion Mathematik und Naturwissenschaft}}
			\\[1cm]
			{\Large \textbf{Fachbereich Informatik und Informationswissenschaft}}
			\\[2cm]
			\begin{minipage}[c]{0.8\textwidth}
				\begin{description}[style=multiline]
					\item {\Large \textbf{1. Gutachter:} Prof. Dr. Harald Reiterer}
					\item {\Large \textbf{2. Gutachter:} Dr. Karsten Klein}
				\end{description}
			\end{minipage}
			\vfill
			{\LARGE \textbf{Konstanz, 2021}}
		\end{center}
	\end{titlepage}

\thispagestyle{empty}
\chapter*{Acknowledgements}
\thispagestyle{empty}
Thank you Betzi, beeing a loyal helper during the development of \exgo.
\newpage
\thispagestyle{empty}
\ 
% Abstract
\makeabstract
% Table of contents
\tableofcontents
\addcontentsline{toc}{chapter}{Contents} % Add contents to table of contents
% List of Figures
\listoffigures
\addcontentsline{toc}{chapter}{\listfigurename} % Add list of figures to table of contents
% List of Tables
\listoftables
\addcontentsline{toc}{chapter}{\listtablename} % Add list of tables to table of contents
\newpage
\thispagestyle{empty}
\ 
% Prepare chapters
\clearpage
\setcounter{romanPages}{\value{page}} % Update variable for roman pages
\pagenumbering{arabic} % Turn on page numbering again

\chapter{Introduction}
(3 pages)\\
Motivation: Motorlearning wichtig zur aneignung von bewegungen. am besten mit echtem lehrer. wenn dieser nicht verfügbar, motor learning in VR möglich und sinnvoll, siehe xyz. allerdings ist der einfluss der perspektive auf die virtuelle guidance vis. noch wenig untersucht. deswegen diese arbeit hier.\\
ferner, wenig motorlearning in zusammenhang mit physical load und wenig "laufen mit egozentrischer anleitung".\\
Daraus folgt die forschungsfrage ... und ihre sub forschungsfragen ...\\
Um daten zu generieren um diese forschungsfragen zu beantworten wurde Exgo entwickelt. Eine studie wurde designed um mit diesem system die notwendigen daten zu generieren.\\
Diese arbeit ist eine evluierung dieser studie.
\section{Outline}
übersicht über dieses dokument
\chapter{Motor Learning in Virtual Reality}
\label{chapter:theoretical_background}
The aquisition and improvement of movements is called motor learning~\cite{mlbook}. 

\section{Mixed Reality}
\label{section:mixed_reality}
Milgram and Kishinho~\cite{mrcontinuum} describe Mixed Reality for visual displays on a continuum (see seminar thesis chapter 2.3). In Virtual Reality the environment is blocked completely while in Augmented Reality the environment is visible and augmented with digital elements. During Motor Learning the visual perception of the own body is desireable, though the approach of augmenting the real-world body with a virtual guidance visualisation is promising. But todays AR-technology provide a small field of view. A solution to this could be the video see-through technology, but this is also limited by latency and distrotion.\\
The perception of the own body can also be achieved by rendering the body of the learner and though can be established in VR. This solution is applied in the proposed study design and the study takes place in Virtual Reality.

\section{Motor Learning}
\label{section:motor_learning}
Motor Learning is achieved through instruction, trying, imitation or a combination of them. The process of Motor Learning can be divided into three parts: cognitive stage, assosiative stage and autonomous stage. In the cognitive stage, training methods are most efficient and the performance gain ist the highest among the stages~\cite{mlbook}. Tasks that belong to this stage are thereby best suited for a study. A detailed description of the stages can be found in the preceeding seminar thesis.\\
Movements can be calassified by two means: by the particular movements  and based on the perceptual attributes. Based on the particular movements, the classification is described by a continuum, compare figure~\ref{fig:movement_classification1}. On the extremes of the continuum are discrete movements and continuous movements. Between these extremes, serial movements are located. For a detailed description please refere to seminar thesis chapter 2.2. Discrete movements are too short for an evaluation. Continuous movements do not have an recognisable beginning and thereby they are not suitable for the study in question either. Serial movements are basically chained discrete movements with a recognisable beginning and end. This allows to determine a task decomposition compare section \ref{handlingphysicalload} and a evaluation of particular tasks. Discrete movements are widely used for reseach in Motor Learning, for example ~\cite{lightguide,mythaichicoaches,elearningma}, therefore, the study task design is based on discrete movements.\\
The classification based on the perceptual attributes is also represented by an continuum and includes the environment the movement is performed compare figure~\ref{fig:movement_classification2}. At the extremes of the continuum open skills and closed skills are located. For closed skills, the environment is predictable while in open skills the environment is not predictable. The study aims to analyse the learners performance of following a movement and not how the learner can adapt to environmental changes, the task for this study must be located on the left hand side of the continuum: closed skills.
\begin{figure}[htb]
	\centering
	\includegraphics[width=\textwidth]{figures/movement_classification.png}
	\caption[Movement classification 1]{Movement classification 1 \cite{mlbook}}
	\label{fig:movement_classification1}
\end{figure}
\begin{figure}[htb]
	\centering
	\includegraphics[width=\textwidth]{figures/movement_classification2.png}
	\caption[Movement classification 2]{Movement classification 2 \cite{mlbook}}
	\label{fig:movement_classification2}
\end{figure}

\subsection{Measurements for Motor Learning}
\label{section:measures_for_ml}

\section{Visual Perspectives}
\label{section:visual_perspectives}
Wang and Milgram~\cite{centricitycontinuum} describe visual perspectives by the centricity continuum~\ref{fig:ego-exo-continuum}. On the left extreme on the continuum, the ego-centric visual perspective is located, on the right extreme the exo-centric visual perspective can be found, while the middle part represents tethered visual perspectives. By moving from the left to the right the so called tethering distance increases. The tethering distance describe the distance of the anchor point of the eyes to the object to control. In this work the object to control is the learners avatar guidance visualisation. Furthermore, Wang and Milgram distinguish tethered visual perspective in dynamic and rigid, a detailed description is given in the seminar thesis chapter 2.1. 
\begin{figure}[htb]
	\centering
	\includegraphics[width=\textwidth]{figures/ego_exo_continuum.PNG}
	\caption[Centricity continuum]{Centricity continuum by Wang and Milgram~\cite{centricitycontinuum}}
	\label{fig:ego-exo-continuum}
\end{figure}
Given a scenario where one learner mimics the movement of one teacher, five different visual perspectives are possible:
\begin{figure}[htb]
	\centering
	\includegraphics[width=\textwidth]{figures/perspectives.png}
	\caption[Possible perspectives]{Possible perspectives with one real world student and one real world teacher.}
	\label{fig:perspectives}
\end{figure}
\begin{itemize}
	\item Ego-centric: the avatar of the teacher is located inside the body of the avatar of the learner; the learner sees the guidance visualisation inside the own body, compare figure~\ref{fig:perspectives} top left.
	\item Exo-centric: the avatar of the guidance visualisation is located outside of the avatar of the learner; the learner sees the guidance visualisation e.g. in front of him/her, compare figure~\ref{fig:perspectives} top right.
	\item Ego \& exo-centric: the combination of ego-centric and exo-centric. The learner sees the guidance visualisation as well as inside and outside of the own body, compare figure~\ref{fig:perspectives} middle left.
	\item Augmented exo-centric: the guidance visualisation is located outside of the leraners avatar, additionally, a virtual copy of the student is located inside the exo-centric guidance visualisation, compare figure~\ref{fig:perspectives} middle right.
	\item Ego \& augmented exo-centric: the combination of the ego-centric visual perspective and the augmented exo-centric visual perspective; the learner sees the guidance visualisation inside the own body, as well as outside. Additionally, a virtual copy of the learner is located inside the exo-centric guidance visualisation, compare figure~\ref{fig:perspectives} bottom.	
\end{itemize}
All visual perspectives are worth an investigation and a comparable study with all five visual perspectives is desirable. Though, to reduce complexity and the number of participants\footnote{Due to COVID-19 pandemic}, this work will focus on three visual perspectives: ego-centric, augmented exo-centric and ego \& augmented exocentric for the following reasons. In the ego-centric visual perspective, the learner sees the teacher inside the own body. Here, the learner can see the relation of the own body to the teachers body directly. In the exo-centric visual perspective this relation cannot be seen. Thereby, the position of the learner in relation to the guidance visualisation must be guessed. That, in turn, makes the application of the speed machanic which is necessary for ego-centric guidance - described in the next section - not possible. A mechanic that is used in all conditions but one could lead to biased data, compare table~\ref{tab:mechanics}.
\begin{table}[htb]
	\centering
	\includegraphics[width=\textwidth]{figures/mechanics_comparison.png}
	\caption[mechanics comparison]{mechanics comparison}
	\label{tab:mechanics}
\end{table}
The mechanic of multiple representations does not influence the validity of the study, because the mechanic would solve an issue that does not exist in the ego-centric perspectives.\\
In the augmented exo-centric perspective, a virtual copy of the learner is located inside the exo-centric guidance visualisation. The copy lets the learner see the relation of the own body to the guidance visualisation. Furthermore, augmenting the exo-centric guidance visualisation with the learner is applied and evaluated in related work~\cite{YouMove,thaichichua}. 
Because the speed machanic cannot be applied and the method of augmenting exo-centric guidance visualisation with a virtual copy of the learner, the augmented exo-centric perspective will be used in the proposed study. For simplicity reasons, the augmented exo-centric visual perspective will be further called exo-centric visual perspective.\\
The third visual perspective that will be used in the proposed study design is the combination of the ego-centric and exo-centric visual perspective: augmented exo-centric visual perspective, which will be further called ego \& exo-centric visual perspective.\\

\subsection{Mechanics for Motor Learning in Virtual Reality}
For teaching movments in Virtual Reality, in the exo-centric visual perspective the following issue arises. The guidance visualisation can move out of the field of view of the learner by the movement itself. Szenario: the learner and the guidance visualisation stand side-by-side, the learner sees the guidance visualisation on the left of hin/her. The guidance visualisation now indicates a movement to turn by 90 degrees to the right. When the learner follow thsi movement, the guidance visualisation will be located behind the learner after the movement ended. A guidance visualisation standing behind the learner cannot be seen by the learner.\\
This issue is solved in existing work with either the restriction of movements~\cite{freethrowsimulator,elearningma} or multiple representations of the guidance visualisation arround the learner~\cite{thaichichua,mythaichicoaches}. The restriction of movements has an strong influence in the task design and is therefore not desirable for the study proposed in this thesis, consequential for exo-centric visual perspectives multiple representations fo the guidance visualisations on strategic positions arround the learner are used.\\
In the ego-centric visual perspective, another issue araises during the teaching of transitional movements in space: walking. To understand this issue, two aspects have to be clear before. (1) The nature of an ego-centric guidance visualisation is to be located inside the learner at any time. (2) A guidance visualisation indicates movements by moving itself. If the guidance visualisatoin is about to indicate a movement away from the learner, the guidance visualisation is moving out of the students body. But a guidance visualisation that is outside of the learners body is no longer ego-centric.\\
A possible soltuion can give the centricity continuum by Wang and Milgram~\ref{fig:ego-exo-continuum}. Following the nature of the centricity continuum, the tethering distance can be increased by a small ammount and the visual perspective can still be classified as ego-centric. But now araises the question, of how far the tethering distance can be increased, with which the perspective still feels ego-centric, but the indication of the movement is considerable. For simplicity reasons, this distance is further called ego-centric tethering distance (ETD). To determine a reasonable ETD, a small formative study was conducted\footnote{A larger study was not possible because of the COVID-19 pandemic}. During this study, a non biased\footnote{The person had no prior knowledge about the system or motor learning.} person was asked to follow movements in the ego-centric visual perspective. The first movement was conducted with an ETD of 5cm. For the folowing movements the ETD was increased by 5cm each. The subjective assesment of the participant and my observations yielded best for an ETD between 15cm and 30cm. These two values are further called:
\begin{itemize}
	\item[] $ETD_{min}=15cm$
	\item[] $ETD_{max}=30cm$
\end{itemize}
Based on $ETD_{min}$ and  $ETD_{max}$ the speed mechanic is developed. The speed mechanic controls the speed of the playback of the guidance visualisation. At $ETD_{min}$ the animation plays at normal speed, at $ETD_{max}$ the guidance visualisation stops. Between $ETD_{min}$ and $ETD_{max}$ the animation speed of the guidance visualisation is linearly interpolated, compare figure~\ref{fig:speed_mechanic}.
\begin{figure}[htb]
	\centering
	\includegraphics[width=\textwidth]{figures/speed_mechanic_chart.png}
	\caption[speed mechanic chart]{speed mechanic chart}
	\label{fig:speed_mechanic}
\end{figure}
The speed mechanic was evaluated by one\footnote{Different person than the initial. This person had no prior knowledge about the system or motor learning. Larger evaluation not possible because of COVID-19 pandemic.} person. The participant followed the guidance visualisation in the ego-centric visual perspective. Observations showed that the participant could follow the movement at ease. The opinion of the participant about the speed mechanic was very positive.


\section{Handling Physical Load}
\label{handlingphysicalload}
The handling of physical load is composed of five elemental tasks: lift, lower, push, pull and hold~\cite{mmh}. Additionally, there are non-elemental tasks like turning and sliding, ibid.. This work will use a study tasks that include the handling of physical load. Evidently, the task should consist of these elemental tasks. A tasks which consists out of the elemental tasks can be generalised to other tasks, to certain degree. To gain a strong data basis, multiple elemental tasks can be chained together (so called Unit-Combined-MMH ibid.). The task design for the proposed study founds on this basis and additionally should have a reference to real-world tasks. For the task, a simple box serves as physical load, compare chapter \todo{3}.\\
The real world provides a variety tasks that include the handling of physical load. For example in a pacel transhipment point, wearhouse workers, grinders or people that work at test stands. They pic up physical load, carry, turn and pull it. To imitate such a task, additional artifacts were created to be used in the proposed study. A table for push and pull, that could be interpreted as a machine or sorting platform and a plate on the floor that could depict for example a scale. Between the table and the scale the box can be carried. The box can be lowered and lifted from and to the scale.\\
As described in chapter \todo{4}, the legs during push and pull are in the same position if executed ergonomically. To gain variation, which is necessary to get insight how the learner can see the feet of the guidance visualisation, two non-elemental tasks were introduced: turn and fold. During this task the foot placement is different to push and pull. With the sub-tasks lift, lower, push, pull, carry, turn and fold the first task was designed. It consisted out of 28 sub-tasks were every sub-task occured 4 times. During the design of the task became clear that an additional element had to be introduced: walking. This enables more variation in the task. For example, the task executer stands at the left side of the table and pushes the box away. Now the executor can walk to the left side and pull the box.


Lift and lower can be realised by lifting the box from the floor and lower the box to the floor. For push and pull, another artifact is necessary: a table makes the execution of push and pull easier than on the ground.


manual material handling~\cite{mmh}, single: lift lower push pull carry hold, hold out because of confusion with speed mechanic, introduced carry because of variaation and flexibility in task. unit/combined mmh tasks classification.\\
baua classification?
%https://www.baua.de/EN/Topics/Work-design/Physical-workload/Types-of-workload/Types-of-workload_node.html https://www.baua.de/DE/Themen/Arbeitsgestaltung-im-Betrieb/Gefaehrdungsbeurteilung/Expertenwissen/Physische-Belastung/Heben-Halten-Tragen/Heben-Halten-Tragen_node.html



\section{Related Work: Motor Learning in Virtual Reality}
\label{section:related_work}
Training movements in Virtual Reality was investigated previously in several works. The preceeding seminar thesis (see chapter 3) provided an overview over 23 (compare table~\ref{tab:rw_overview}) of these works and evaluated six of them in detail: Tai Chi Trainer by Chua et al.~\cite{thaichichua}, YouMove by Anderson et al.~\cite{YouMove}, VR Dance Trainer by Chan et al.~\cite{vrdancetrainer}, OneBody by Hoang et al.~\cite{onebody}, LightGuide by Sodhi et al.~\cite{lightguide} and Pyhsio@Home by Tang et al.~\cite{physioathome}. Special attention was payed to the visual perspective, task, guidance visualisation and their independent and dependent variables they used in their investigations. Finnaly, the results were compared and the results of their works. An overview is depicted in table~\ref{tab:rw_overview_detail}. This work is informed by these works in various aspects. Chua et al. used the ego \& augmented exo-centric visual perspective, Hoang et al. and Sodhi et al. the ego-centric visual perspective. These visual perspectives prooved to be suited for the evaluation of motor learning in VR and is adopted for the proposed study design, compare section~\ref{section:visual_perspectives}. Furthermore, Chan et al. and Chua et al. used high realisitc avatars as guidance visualisation, which are used in the proposed study design, compare seminar thesis chapter 3.3. Furthermore, recent research indicates, that high realism avatars outperforms abstract avatars~\cite{max,perspectivematters} All authors used a performance measure to evaluate the performend movements of the participants of their studies. Especially the distance based measures informed the measures used in the proposed study design, compare~\ref{section:measures_for_ml}.\\
The relatively new technology of Vive Trackers in combination with inverse Kinematics (see project report chapter 2.1 and 2.2) is not used by the above mentioned works. Sra et al.~\cite{samesetup} used this technology in 2018 for their system Your Place and Mine to render human shaped avatars.\\
The results of related work yielded in no clear conclusion about the influence of the perspectives on motor learning. Chua et al. found no difference in the performance between the visual perspectives, Anderson et al. and Chan et al. found out that their exo-centric visual perspectives in Virtual Reality outperforms traditional video guidance. The works of Hoang et al. and Sodhi et al. conclude that the ego-centric perspective outperforms the exo-centric visual perspective. But an investigation of how the visual perspective influences motor learning was not investigated. Recently, in December 2020, Yu et al.~\cite{perspectivematters} conducted three independent studies to close this gap. In the first study Yu et al. compared the ego-centric visual perspective and a 2D-mirror for single arm movements. In the second study they compared the ego-centric and exo-centric visual perspective for Yoga. In the third study they compared the ego-centric visual perspective with an 3D-mirror for arm movements. Yu et al. conclude their findings in a design guideline for systems training Motor Learning in Virtual Reality: use the ego-centric visual perspective if the type of motion allows, consider alternatives for other types of motions, ibidem. In all three studies the ego-centric visual perspective outperformed the other perspectives, if the movement was clearly visible from the ego-centric visual perspective. This work, in contrast, focus on full body movements that include the handling of physical load. Furthermore, this work provides a third visual perspective where the ego-centric and exo-centric visual perspective is combined.
\begin{table}[htb]
	\centering
	\includegraphics[width=\textwidth]{figures/detail_paper_overview.png}
	\caption[Overview seminar evaluation]{Overview seminar evaluation}
	\label{tab:rw_overview_detail}
\end{table}

\begin{table}[htb]
	\centering
	\includegraphics[width=\textwidth]{figures/rw_overview.png}
	\caption[Overview seminar evaluation]{Overview Related Work divided by perspective and task}
	\label{tab:rw_overview}
\end{table}
\section{Research Contribution Statement}
\label{delimination_contribution}
The conduction of the proposed study will produce data that serves as a reasonable basis for designers of VR Motor Learning systems.
bekannte arbeiten und deren ergebnisse über motor learning in VR\\
\\
auf basis dieses kapitels wird die studie geformt


\chapter{Experiment Design}
\label{chapter:studysetting_conduction}
This master's thesis proposes an experiment that answers the research question RQ1: How does the visual perspective on a virtual guidance visualisation influence motor learning in Virtual Reality? This chapter describes the design of the experiment. First, the independent variables, namely the VPs, are determined in section~\ref{sec:visualPerspecticves}. Afterwards, the task for the experiment is developed in section~\ref{sec:taskDesign}. Finally, section~\ref{sec:measures} describes the independent variables of the experiment.

\section{Visual Perspectives}
\label{sec:visualPerspecticves}
The last chapter pointed out five visual perspectives, compare figure~\ref{fig:perspectives}. All VPs are worth investigating, and a comparative experiment with all five visual perspectives is desirable. However, to reduce complexity and the number of participants\footnote{Because of the COVID-19 pandemic.}, this work will focus on three visual perspectives.\\
Figure~\ref{fig:perspectives} shows three main classes of VPs: ego-centric, exo-centric and perspectives which contain both. To answer the research question, it is indispensable to examine at least one of each class. The ego-centric VP is the only VP in the g-class and thus chosen by default. The exo-centric VP can be realised as purely exo-centric or augmented exo-centric. The combination of ego-centric and exo-centric can be realised as ego- \& exo-centric or ego- \& augmented exo-centric. However, before the exo-centric VP and the combination can be chosen, a closer look at the mechanics that make motor learning in VR possible is necessary.

\subsection{Excursion: Mechanics for Motor Learning in Virtual Reality}
\label{sec:mechanics}
For teaching movements in Virtual Reality in the exo-centric VP, the following issue arises: the GV can move out of the learner's field of view by the movement itself. Szenario: the learner and the GV stand side-by-side. The learner sees the GV to the left. The GV now indicates a movement to turn by 90 degrees to the right. As soon as the learner follows this movement, the GV will move out of the field of view of the learner. After the movement ends, the GV is located behind the learner. The learner cannot see a GV standing behind himself/herself.\\
This issue is solved in existing work with either the restriction of movements~\cite{freethrowsimulator,elearningma} or multiple representations of the GV around the learner~\cite{thaichichua,mythaichicoaches}. The restriction of movements has a strong influence on the task design and is therefore not desirable for the experiment proposed in this master's thesis. Consequentially, for exo-centric visual perspectives, multiple representations for the GVs on strategic positions around the learner are necessary.\\

In the ego-centric VP, another issue arises during the teaching of locomotion movements. To understand this issue, two aspects have to be clear before: (1) the nature of an ego-centric GV is to be located inside the learner at any time. (2) A GV indicates movements by moving itself. If the GV is about to indicate a movement away from the learner, the GV is moving out of the student's body. However, a GV that is outside of the learner's body is no longer ego-centric.\\
A possible solution is given by the centricity continuum by Wang and Milgram~\ref{fig:ego-exo-continuum}. Following the centricity continuum's nature, the tethering distance can be increased by a small amount, and the visual perspective can still be classified as ego-centric. But now the question arises of how far the tethering distance can be increased with which the perspective still feels ego-centric, but the indication of the movement is considerable. For simplicity, this distance is further called ego-centric tethering distance (ETD). To determine a reasonable ETD, an informal formative test\footnote{A formal study with more participants was not possible because of the COVID-19 pandemic. This holds for all upcoming formative tests.} was conducted with one participant. The participant was a former Computer Science Student with expertise in VR systems but had no prior knowledge about motor learning. During the formative test, the participant was asked to follow movements in the ego-centric visual perspective. The first movement was conducted with an ETD of 5cm. For the following movements, the ETD was increased by 5cm each. After each movement, the participant was asked about the ability to follow the movements. The subjective assessment of the participant and my observations yielded best for an ETD between 15cm and 30cm. These two values are further called:
\begin{itemize}
	\item[] $ETD_{min}=15cm$
	\item[] $ETD_{max}=30cm$
\end{itemize}
Based on $ETD_{min}$ and $ETD_{max}$ the \textit{speed mechanic} is developed. The \textit{speed mechanic} controls the speed of the playback of the GV. At $ETD_{min}$ and below, the animation plays at normal speed. At $ETD_{max}$ the GV stops. Between $ETD_{min}$ and $ETD_{max}$ the animation speed of the GV is linearly interpolated, compare figure~\ref{fig:speed_mechanic}.
\begin{figure}[htb]
	\centering
	\includegraphics[width=0.6\textwidth]{figures/speed_mechanic_chart.png}
	\caption[Animation speed for the \textit{Speed mechanic}]{\textit{Speed mechanic}: animation speed of the GV in relation to the distance between learner and GV.}
	\label{fig:speed_mechanic}
\end{figure}
The \textit{speed mechanic} was evaluated by an informal formative test with one participant. The participant was a PhD student of Computer Science and had little experience with VR systems, and none in motor learning. The participant's task was to follow the GV in the ego-centric visual perspective. Observations showed that the participant could follow the movement at ease. The opinion of the participant about the speed-mechanic was very positive ("It did not run away. I had no problem to follow the woman (ed: GV).").\\
With this excursion, a reasonable decision for the exo-centric VP and the combination can be made.\\

In the ego-centric visual perspective, the learner sees the GV inside the own body. Here, the learner can see the relation of the own body to the GV directly. In the pure exo-centric visual perspective, this relation cannot be seen. Thereby, the position of the learner in relation to the GV must be guessed. That, in turn, makes the application of the speed-mechanic - which is necessary for ego-centric guidance - nearly impossible. A mechanic that is used in all conditions but one could lead to biased data, compare table~\ref{tab:mechanics}.
\begin{table}[htb]
	\centering
	\includegraphics[width=0.8\textwidth]{figures/mechanics_comparison.png}
	\caption[Application of mechanics per visual perspective]{Application of \textit{speed mechanic} and \textit{multiple representations} per VP.}
	\label{tab:mechanics}
\end{table}
The mechanic of multiple representations does not influence the experiment's validity because the mechanic would solve an issue that does not exist in the ego-centric perspectives. Furthermore, any VP with more than one representation is an exo-centric VP.\\
In the augmented exo-centric VP, a virtual copy of the learner is located inside the exo-centric GV. The copy lets the learner see the relation of the own body to the GV. Furthermore, augmenting the exo-centric GV with the learner is widely used and evaluated in related work~\cite{YouMove,thaichichua}. Consequently, the augmented exo-centric VP will serve as the exo-centric VP.\\
With the ego-centric and exo-centric VP set, the combination can be determined. In the ego-centric VP, the learner has a direct comparison of the own posture to the GV's posture in the ego-centric VP. In the augmented exo-centric VP, the learner has a direct comparison of the own posture and the GV's posture in the exo-centric VP. For a direct comparison of the own posture and the GVs posture in the ego-centric VP AND the exo-centric VP, the ego- \& augmented exo-centric VP is chosen as the combination. The ego- \& augmented exo-centric VP is the true combination of ego-centric and augmented exo-centric.\\
For simplification, the augmented exo-centric VP will be further called exo-centric VP, and the ego- \& augmented exo-centric will be further called ego- \& exo-centric VP.\\
The ego-centric VP, exo-centric VP and the ego- \& exo-centric VP are the independent variables of the experiment and form the three experiment conditions EGO, EXO, EGO \& EXO.

\section{Task Design}
\label{sec:taskDesign}
Hornb\ae{}k~\cite{hornbaek} identified three main types of tasks in HCI studies: representative tasks, simple tasks and tasks that use task-specific hypothesis. RQ1 states that the main investigation field is motor learning. Motor learning is strongly related to real-world movements. Evidently, the experiment task is a representative task.\\
Real-world tasks that include the handling of physical load can found in a wide range of activities. For example, a storekeepers job is to clear a palette of cardboard boxes. This task includes unloading the palette, scaling the boxes, measuring the dimensions of the boxes and finally storing them in a rack. Another example is the work at a grinding machine. The worker takes a slug from a shelf and works on it until the slug becomes a workpiece. After that, the workpiece is carried to a measurement instrument to be verified. There are plenty of other examples, but these two already clarify that tasks which include the handling of physical load consist out of the elemental tasks for manual material handling: \textit{lift}, \textit{lower}, \textit{push}, \textit{pull}, \textit{hold}.\\
The idea for the experiment task is to chain these elemental tasks together to create a Unit-Combined-MMH task that representatively stands for a wide range of tasks that includes the handling of physical load. To achieve this, several aspects have to be taken into consideration: (a) the artefacts with which the learner will interact, (b) a reasonable task decomposition into subtasks and their chaining that allows the investigation of subtasks. Moreover, the experiment needs a (c) structure. (c) will reveal the necessity of three tasks, which have to be (d) equally complex. This section will subsequently discuss (a-d) and propose the task for the experiment.

\subsection{(a) Artefacts}
A task that includes the handling of physical load obviously needs a physical load. In real-world tasks, the physical load can be everything a human can handle. The physical load for this task should fulfil the following criteria. First, the load should have a significant weight, that it is perceived as a load, but at the same time, any healthy person with no previous illnesses can handle it without getting injured. Secondly, the physical load should give enough freedom for interactions. A simple box fulfils the criteria and has a relation to physical loads of real-world tasks like the handling of parcels. With a physical load, the elemental tasks of lift and lower can be realised by lifting and lowering the box from and to the floor.\\
\textit{Push} and \textit{pull} can be realised by pushing and pulling the box on the floor, but it can feel cumbersomely. Moreover, in real-world task pushing and pulling a box is made possible in a more ergonomic height if feasible, not least for security reasons. To support \textit{push} and \textit{pull}, a table is introduced. This table stands representatively, for example, for the grinding machine or a parcel sorting table.\\
Finally, the transitions between the elemental tasks have to be supported to increase the real-world reference. This is achieved by providing a waypoint. The waypoint is a plate on the floor and helps to bring sense in movements. This plate representatively stands, for example, for a scale or second machine. Walking to a scale or lower a box to the scale on the floor increases the real-world reference more than just an empty place in the room. For simplification, in the following, the addressed waypoint is called scale.

\subsection{(b) Subtasks}
\label{sec:subTasks}

\begin{figure}[H]
	\centering
	\includegraphics[width=\textwidth]{figures/sub-tasks.png}
	\caption[Depiction of subtasks]{Depiction of subtasks. a) \textit{lift} and \textit{lower}, b) \textit{push} and \textit{pull}, c) \textit{turn}, d) \textit{fold}, e) \textit{pick} and \textit{place}, f) \textit{carry}.}
	\label{fig:sub-tasks}
\end{figure}

The goal is to create a Unit-Combined-MMH task with the elemental tasks \textit{push}, \textit{pull}, \textit{lift}, \textit{lower} and \textit{hold}. The process of designing the task was complex and took place iteratively. In the following, the process of designing the task is structured by the iterations (task Mk I - task Mk V). For visualisation, the subtasks are depicted in figure~\ref{fig:sub-tasks}\\

\subsubsection{Task Mk I}
The first approach was a task with four occurrences of every elemental task. For lifting the box from the scale and carry the box to the table, obviously, a new task type had to be introduced: \textit{carry}. Because \textit{carry} is not an elemental task and for simplicity, elemental tasks and newly introduced task types are referred to as subtasks. The designing of task Mk I revealed an issue: chaining a given amount of subtasks together so that the task is still conductible is hard to achieve. To overcome the inflexibility in task design, a new subtask is introduced: \textit{walk}. \textit{Walk} means locomotion without the box in hand. With \textit{walk}, the box can be pushed from one side of the table and then be pushed from the other side of the table, which achieves flexibility in task design. Otherwise, on \textit{push} will always follow \textit{pull}.\\
Outcome: new subtask \textit{walk} introduced to increase flexibility in task design.

\subsubsection{Task Mk II}
In task Mk II the subtasks \textit{push}, \textit{pull}, \textit{lift}, \textit{lower}, \textit{carry}, \textit{walk} and \textit{hold} are about to chained together. Each subtask appeared four times. Task Mk II was informally tested with one participant. The participant had to follow the instructions in the ego-centric VP and exo-centric VP. During the task's conduction, the participant started to look around and correct the own position during the subtask \textit{hold}. An interview afterwards showed that the participant thought the GV had stopped because his position had been too far away from the GV. It became clear that the speed-mechanic and the subtask \textit{hold} are not compatible. It is indistinguishable for the experiment participant if he/she is too far away from the GV or if it is the subtask \textit{hold}. Because of this indistinguishableness, the subtask \textit{hold} is excluded from the task. However, \textit{hold} is still part of the whole tasks: between the transitions of the tasks (for example, between \textit{lift} ends and \textit{lower} starts) is a slight pause which is equivalent to \textit{hold}. However, this sequence is too short to log reasonably. Furthermore, \textit{hold} is part of the subtask \textit{carry}, where the box is held in front of the body. However, \textit{hold} is not a stand-alone subtask and thus can not be evaluated isolated.\\
Outcome: subtask \textit{hold} is eliminated because of ambiguity.

\subsubsection{Task Mk III}
A new task was designed with the subtasks \textit{push}, \textit{pull}, \textit{lift}, \textit{lower}, \textit{carry} and \textit{walk}. During the design, special attention was paid to the magnitude of the movements. For example, every \textit{push} should be equally far. \textit{Lift} and \textit{lower} from and to the scale and \textit{lift} and \textit{lower} from and to the table are very different in magnitude. This resulted in two new subtasks: \textit{pick} and \textit{place}. \textit{Pick} means to pick up the box from the table, \textit{place} means to place the box on the table. For \textit{lift} and \textit{lower} the target remained the scale on the floor.\\
Outcome: new subtasks \textit{pick} and \textit{place} introduced. This ensures an equal magnitude for every subtask.

\subsubsection{Task Mk IV}
For task Mk IV the subtasks \textit{push}, \textit{pull}, \textit{lift}, \textit{lower}, \textit{carry}, \textit{walk}, \textit{pick} and \textit{place} are chained together. The task was inspected by a professional physiologist with four years of work experience. The physiologist was asked to describe the subtasks in detail and perform every subtask ergonomically. The professional's description of the subtasks are listed in table~\ref{tab:sub-tasks}. Through the performance of the physiologist and the description of the subtasks could be derived several insights. The subtasks \textit{push} and \textit{pull} are similar in their conduction. The same applies to the pairs \textit{lift} and \textit{lower} as well as \textit{pick} and \textit{place}. For the evaluation, this meant that the variations of movements are nearly halved, and thus the possibility of making mistakes is reduced. Example: for \textit{push} and \textit{pull} one foot has to be placed to the back while the other foot remains under the hip. The hands do the same for every \textit{push} and \textit{pull}. If the participant performs the subtask intrinsically correct without the perception of the GV, the experiment will not measure the influence of the perspective.\\
To increase the number of sources of error, two new subtasks are introduced: \textit{turn} and \textit{fold}. \textit{Turn} means to turn the box by 90 degrees on the table. \textit{Fold} means to tilt the box from one side to another. The difference in hand movement to \textit{push} and \textit{pull} is obvious. The difference for the feet results from the fact that during \textit{turn} and \textit{fold}, the box' weight remains on the table. The force to apply on the box is significantly lower than during \textit{push} and \textit{pull}. This results in a different feet placement, which is hip wide under the hip.\\
Outcome: new subtasks \textit{turn} and \textit{fold} introduced to increase the possibility of making errors.

\subsubsection{Task Mk V}
With the introduction of \textit{turn} and \textit{fold}, all subtasks are introduced. A new task was created with all ten subtasks. To assess all subtasks multiple times, they appear four times each per task. The pair \textit{lift} and \textit{lower} and the only in magnitude different pair \textit{place} and \textit{pick} relate to each other. To be presented equally in the task among the other subtasks, they should also only be present four times. Because \textit{lift} and \textit{lower} are measured with the RM (6.2) (see next section), \textit{lift} and \textit{lower} were decided to appear three times each, and \textit{pick} and \textit{place} one time each. Unfortunately, only one time each \textit{pick} and \textit{place} means that all subtasks that do not happen at the table had to be conducted in sequence. To regain flexibility in the task, it was decided that \textit{pick} and \textit{place} occur two times each. This results in 34 subtasks per task. Table~\ref{tab:sub-tasks} provides an overview of the subtasks and their corresponding description, as well as the occurrences per task.\\
Outcome: task 1 in table~\ref{tab:tasks}.

\begin{table}[H]
	\centering
	\includegraphics[width=\textwidth]{figures/sub_tasks_definition.png}
	\caption[Description of subtasks]{Subtask ID, description of subtasks and amount occurences of the subtask per task. Professional's description provided by a professional physiologist.}
	\label{tab:sub-tasks}
\end{table}

\subsection{(c) Experiment Structure}
\label{sec:studyStructure}
The experiment will compare three conditions: EGO, EXO and EGO \& EXO. The main question of this section is how to assign the participants to the independent variables. The key distinction is between within-subject design and between-subject design~\cite{hornbaek}. In the within-subject design, the participants would experience all conditions. In the between-subject design, the participants would experience only one condition. Within-subject designs typically can detect the differences between the conditions more precisely (ibid.). Furthermore, within-subject designs need fewer participants than between-subject designs (ibid.). For those reasons, the experiment is planned to be conducted in a within-subject design.\\
However, the within-subject design also has a drawback: the participants gain experience about the (i) task and the (ii) conditions during the experiment.\\

The solution for (i) is to create three tasks with nearly equal complexity. The participant will face a new task every condition. However, the tasks are still similar, and the learning effect persists. A further reduction of the influence of the learning effect on the outcome can be countered out by counterbalancing the task.\\

(ii) implies that one condition is influenced by another condition, which the participant already experienced. Additionally, there is an asymmetrical carry-over effect between the conditions: EGO \& EXO contains condition EGO and EXO\footnote{EGO \& EXO is the union of EGO and EXO}. Thereby, EGO \& EXO influences EGO and EXO more than EGO and EXO influence EGO \& EXO. The solution to (ii) is counterbalancing, to counter the effect out.\\

Hornb\ae{}k proposes, in this case, to cross the conditions with the task and use a Greco-Latin square~\cite{hornbaek}. Three conditions and three tasks in a Greco-Latin square result in blocks of nine participants. A block is depicted in figure~\ref{fig:study_session_plan}. Apart from this, Hornb\ae{}k states that experiments conducted within-subject should be conducted with at least 20 participants (ibid.). Because one block requires nine participants, the experiment should be conducted with at least three blocks ($3x9=27$ participants). 
The participants will have different demographics, which can influence the experiment's outcome, too. To reduce the demographic effect, the first session of every participant is for acclimatisation and is excluded from evaluation.

\begin{figure}[htb]
	\centering
	\includegraphics[width=0.7\textwidth]{figures/study_session_plan.png}
	\caption[Experiment structure]{Experiment structure: within-subject desing in a Greco-Latin square.}
	\label{fig:study_session_plan}
\end{figure}

\subsection{(d) Equal Task Complexity}
An experiment participant will face in every condition another task. For the experiment's validity, it is indispensable that these three tasks have nearly equal complexity. As described in (b), a task consists of ten subtasks that occur with a specific amount. The main idea to ensure a comparable complexity is to use the subtasks for all three tasks in an equal amount but shuffled. This means the 34 subtasks of task one occur in task two and three but in a different order. Table~\ref{tab:tasks} lists all three tasks. For every task, the subtask number ST1-ST34 is provided. Every subtask number stands for a subtask, which comes with a description and the subtask ID. Reading the description from top to bottom is the instruction the learner receives from the GV during one condition. The mirror mentioned in the first line is another waypoint, which is necessary for technical reasons and is described in section~\ref{sec:selfperception}.

\begin{table}[H]
	\centering
	\includegraphics[width=\textwidth]{figures/tasks.png}
	\caption[Task description]{Task 1 - 3. ST\#: subtask number, ST ID: subtask ID. Reading the description of one task from ST1 to ST34 corresponds to the instructions the learner receives from the GV.}
	\label{tab:tasks}
\end{table}

\section{Dependent Variables}
\label{sec:measures}
This master's thesis aims to answer the main research question RQ1: How does the visual perspective on a virtual guidance visualisation influence motor learning in Virtual Reality? To answer RQ1, the proposed experiment has to generate data that can answer the sub-research questions RQ1.1-4. This section will provide the underlying paradigm to every sub-research question and explain which measures are necessary.\\

\textbf{RQ1.1} How does the visual perspective on a virtual guidance visualisation influence movements' accuracy?\\
\textbf{Paradigma:} The more exact the learner's movements matches the GV movements, the better the learner could follow the instructions of the GV. For RQ1.1.1, the limbs of the learner and the limbs of the GV are compared. For RQ1.1.2, the box' accuracy is compared. For RQ1.1.3, both are compared and additionally, the current subtask is taken into consideration. The accuracy can indicate how the particular movement is suited for the VP.
\begin{itemize}
	\item[] \textbf{RQ1.1.1} How does the visual perspective on a virtual guidance visualisation influence movements' accuracy of the own body?\\
	\textbf{Measures}: (1) Euclidean distance between the learner's and GV's hands, feet, head and hip in meters. (2) Angle between the learner's and GV's hands, feet and hip in degrees.
	
	\item[] \textbf{RQ1.1.2} How does the visual perspective on a virtual guidance visualisation influence the accuracy of handling physical load?\\
	\textbf{Measures}: (3) Euclidean distance between the learner's and GV's box. (4) Angle between the learner's and GV's physical load in degrees.
	
	\item[] \textbf{RQ1.1.3}How does the visual perspective on a virtual guidance visualisation influence subtasks accuracy?\\
	\textbf{Measures}: (1-4), additionally matched to the subtask that is currently performed (5).
\end{itemize}	
(1-4) gives insights to what extent the learner could follow the GV for the whole task. (5) can extract specific subtasks for which the learner could follow the GV to a certain extent. For example, in the ego-centric VP, the overall accuracy for a task is lower than in the other VPs, but the accuracy for the subtasks \textit{lift} and \textit{lower} is higher than in other VPs. For this example, measure (5) can extract specific subtasks that are performed better or worse than in other VPs.\\

\textbf{RQ1.2} Does the visual perspective on a virtual guidance visualisation influence the transfer of ergonomic principles?\\
\textbf{Paradigma:} the more exact the learner's RM matches the GVs RM, the better the ergonomic principles could be transferred.\\
\textbf{Measures:} (6) risk metrics: (6.1) \textit{upright stance} in degrees, (6.2) \textit{squat} in meters, (6.3) \textit{good base} in meters, (6.4) box-near-body in meters.\\

\begin{itemize}
	\item[] (6.1) \textit{upright stance} is defined by the difference in degrees between the straight upward vector and the back of the learner. For all subtask, \textit{upright stance} should be in a certain window, see~\ref{sec:ergonomicMeasurements}. Upright stance indicates if the learner could percept the correct posture of his back.
	
	\item[] (6.2) \textit{squat} is defined by the distance in meters between the feet. For the subtasks \textit{lift} and \textit{lower}, the squat distance should be in a specific window. For the other subtasks, \textit{squat} is not applied because the knees do not bend in the other subtasks. \textit{Squat} indicates if the learner could percept that he should bend his knees during \textit{lift} and \textit{lower}.
	
	\item[] (6.3) \textit{good base} is defined by the distance in meters between the feet. For the subtasks \textit{push}, \textit{pull}, \textit{turn}, \textit{fold}, \textit{lift}, \textit{lower}, \textit{pick} and \textit{place}, \textit{good base} should be in a specific window. Good base indicates if the learner could percept the correct posture of the feet.
	Muckell et al.\cite{muckell} additionally use the RM \textit{spine twist} in their work. This RM cannot be applied for this experiment because of the multiple-representation mechanic. The learner has multiple GV around and is free of choice which one to look at. The turn of the head implies spine twist. Thus, \textit{spine twist} would have low validity and reliability.
	
	\item[] (6.4) box-near-body. During the task design, a professional physiologist was consulted. During the interview, all movements were described in detail, compare table~\ref{tab:sub-tasks}. During the subtask \textit{carry}, the box should be as near as possible to the body, while the elbows should have a bend angle of 90 degrees. The physiologist stated this posture as important. This statement is the basis for introducing an additional risk metric related measurement: box-near-body. Unfortunately, the bend angle could not be determined during an experiment for technical reasons, see chapter~\ref{chapter:system}. Fortunately, the distance between the box and hip can be determined. Box-near-body is defined as the distance in meters between the learner's box and hip. For the subtask \textit{carry}, box-near-body should be in a certain window.
\end{itemize}

RM (6.1-4) are different from accuracy measurements (1-5) because they are independent of the learner's position and the GVs position. For example, in the exo-centric VP, a learner cannot percept the correct position where he/she should stand. The learner thereby stands 15cm away from the position he/she should stand. The overall accuracy is thereby lower. But the learner could percept the positioning of his/her feet correctly. In this case, the RM (6.3) are fulfilled while the accuracy is biased.\\

\textbf{RQ1.3} How does the visual perspective on a virtual guidance visualisation influence the learner's visual focus?\\
Measures: (7) \textit{looking at}\\
Paradigma: the learner's visual focus is on the object the learner is looking at.\\
The learner interacts with a box and has multiple GVs around and inside the learner. \textit{Looking at} can give insights on which GV the learner is focusing, the frequency of focus changes and the role of the physical load.

\textbf{RQ1.4} What is the subjective personal preference of the learner for the visual perspectives?\\
Measures: (8) qualitative data; Likert scales, semi-structured interview, digging into incidences. After each session, an \textit{after-session questionaire} is handed to the participant. After all three sessions, a semi-structured interview is conducted.\\
The qualitative data serves not only to investigate the learner's personal preference but also to serve as triangulation method for (1-7).\\

The last measure is the (9) task completion time (TCT) measured in milliseconds. The speed-mechanic regulates the speed of the animation of the GV. The further the learner is located to the GV, the slower the GV animation speed until it stops entirely at $EDT_{max}$. The task completion time can give insights into what extent the learner could perceive the desired position of his/her body in relation to the GV. This measure relates to (1) and can be used for triangulation.
Additionally, it is to expect that the TCT is decreasing from condition to condition because the participant acclimates. By that, TCT could give insights into the learning effect between the conditions. Finally, the experiment is recorded by video. If during the evaluation questions about a specific topic arise, the recordings can be consulted.


\chapter{\exgo - Design and Implementation}
\label{chapter:system}
The previous chapter describes an experiment to investigate the influence of the visual perspective on motor learning in Virtual Reality. For the conduction of this experiment, a system is necessary. This system is called \exgo. For an in-detail description of the implementation of \exgo\ the preceding project report to this master's this can be consulted~\cite{projectReport} which is also digitally attached to this document. This section elucidates the development of \exgo\ in a condensed form. The starting point is the creation of a self-perception of the learner. 
Section~\ref{sec:selfperception} describes how the learner receives a digital body (avatar) in Virtual Reality.
In section~\ref{sec:artefacts} the physical artefacts are added with which the learner will interact and where they are located in the digital and virtual space (\ref{sec:experimentSetting}). Then, the GVs are added in section~\ref{sec:gv}. Subsequently, section~\ref{sec:perspectives} describes the implementation of the VPs, which serve as conditions of the experiment. With the learners avatar, physical artefacts, GVs and the experiment conditions implemented, \exgo\ is able to teach Motor Learning in VR. To measure the performance of the experiment participants, the measures from section~\ref{sec:measures} are implemented and described in section~\ref{sec:logging} and~\ref{sec:quali_logging}. 
Finally, section~\ref{sec:limitations} evinces the limitations of \exgo.
After \exgo\ is complete, all actions to perform the experiment are known. These actions are assembled with the experiment design, and the experiment's procedure can be described. The experiment procedure is depicted in section~\ref{sec:procedure}.

\section{Self-Perception}
\label{sec:selfperception}
\begin{figure}[H]
	\centering
	\includegraphics[width=0.8\textwidth]{figures/hardware.png}
	\caption[Hardware]{Hardware utilised by \exgo. a) Valve Index, b) base-station, c) Vive Tracker 2, d) Vive Tracker 2 attached to Vive Tracker straps.}
	\label{fig:hardware}
\end{figure}

There are various options of devices to dive into Virtual Reality. Several devices have been evaluated, and the decision was made for the Valve Index\footnote{\href{https://www.valvesoftware.com/en/index}{https://www.valvesoftware.com/en/index}, accessed 29.3.2021} (figure~\ref{fig:hardware} a)), because of its refresh rate, screen solution, field of view and the possibility to wear glasses underneath the head-mounted display (HMD). To determine the position and orientation of the HMD, the so-called Lighthouse is utilised. A Lighthouse consists of at least two base-stations~\footnote{\href{https://www.valvesoftware.com/en/index/base-stations}{https://www.valvesoftware.com/en/index/base-stations}, accessed 29.3.2021} (figure~\ref{fig:hardware} b)). The base-stations are placed at opposite corners of a room and span the tracking volume. To improve the tracking and for the avoidance of untracked areas, e.g. under a table, \exgo\ uses four base-stations, one for each corner of the room, to span the Lighthouse.\\
With this setup, the learner can move in an empty virtual world. The next step is to replace the empty virtual world with a meaningful environment. To create the environment, the Game Engine Unity 3D\footnote{\href{https://unity.com/}{https://unity.com/}, accessed 29.3.2021} is used. In Unity3D, a basic room was created. Four light yellow walls, a parquet floor and unidirectional lighting. The parquet floor serves a purpose: it has a structure with frequent straight lines, making it easier to align the artefacts the learner will interact with. The room is kept simple not to distract the participant from the experiment.\\
Because high realistic GVs outperform stylised GVs~\cite{max} and indicator-based GVs for full-body movements tend to overwhelm learners~\cite{lightguide}, the decision was made to used high realistic human-shaped avatars for the learner and the GV. The next step is to add the learner's avatar to the empty room. To achieve this, the learner's body needs to be tracked. Multiple full-body tracking systems were compared. The decision was made for Vive Tracker 2\footnote{\href{https://www.vive.com/eu/accessory/vive-tracker/}{https://www.vive.com/eu/accessory/vive-tracker/}, accessed 29.3.2021} (figure~\ref{fig:hardware} c)), because of the cease of coordinate system matching, low latency and less work-intensive calibration process. The learner wears six Vive Trackers in total, compare figure~\ref{fig:tracker_placement}. Five of them plus the HMD are necessary for the full-body tracking of the learner. The remainder is necessary for RM (6.1), which is later explained in section~\ref{sec:logging}. Two trackers are located at Dorsum pedis\footnote{\label{fn:latin}Latin description: Dr. med. univ. Kilian Roth} (compare figure~\ref{fig:tracker_placement} a)), two trackers are located at Dorsum manus\footref{fn:latin} (compare figure~\ref{fig:tracker_placement} b)). One  tracker is located at Vertebra lumablis 5\footref{fn:latin} (L5) (compare figure~\ref{fig:tracker_placement} c)). The trackers are attached to the learner by special Vive Tracker Straps\footnote{\href{https://www.google.com/search?q=vive+tracker+straps}{https://www.google.com/search?q=vive+tracker+straps}, accessed 10.03.2021} (figure~\ref{fig:hardware} d)).\\
The Lighthouse tracks the Vive Trackers and HMD, which send their position to the PC. On the PC, SteamVR\footnote{\href{https://store.steampowered.com/app/250820/SteamVR/}{https://store.steampowered.com/app/250820/SteamVR/}, accessed 21.03.2021} receives the information and forwards it to Unity3D. In Unity3D, the SteamVR Plugin\footnote{\href{https://assetstore.unity.com/packages/tools/integration/steamvr-plugin-32647}{https://assetstore.unity.com/packages/tools/integration/steamvr-plugin-32647}} provides the information to a workable format. The tracking information is now about to be transformed into a rendering of a human-like avatar at the position of the learner's body. This requires several steps.\\First, an avatar is imported. To create the avatar, Reallusion Character Creator 3 3\footnote{\href{https://www.reallusion.com/character-creator/}{https://www.reallusion.com/character-creator/}, accessed 21.3.2021} was used. To match the gender of the participant, a male and a female character was created, wearing the same clothes. Based on the demographic questionnaire, the gender can be set, and the participant will see an avatar complying with the participant's gender.\\
Secondly, the tracker's position and orientation in the tracking volume have to be translated into human movements that meet the learner's movements. This is achieved by Inverse Kinematics (IK).\\
\begin{figure}[htb]
	\centering
	\includegraphics[width=\textwidth]{figures/trackerPlacement.png}	
	\caption[Tracker placement]{Tracker placement. a) front view - Vive Tracker at Dorsum pedis, b) side view - Vive Tracker at Dorsum manus, c) back view - Vive Tracker at Vertebra lumablis 5 (L5) and Vertebra thoracalis (T8).}
	\label{fig:tracker_placement}
\end{figure}
Short excursion: IK arises from the field of robotics. A robot arm consists of limbs and joints. Each limb has a specific length, and each joint has a specific range of angles to move. The length and angles are called rules. Given an endpoint the robot has to reach with the most outer limb, the angle of each joint can be calculated with the rules. This process can be mapped to a human body, too.\\

Unity3D provides a third-person plugin called FinalIK\footnote{\href{https://assetstore.unity.com/packages/tools/animation/final-ik-14290}{https://assetstore.unity.com/packages/tools/animation/final-ik-14290}, accessed 21.03.2021} that is capable of the calculations in question. On the one hand, FinalIK is powerful and unrivalled in functionality compared to other IK tools and though influenced the choice to use Unity3D for \exgo\ heavily. On the other hand, to match the needs of the experiment, extensive adjustments were necessary. The main task is to transfer the information from SteamVR to FinalIK in a meaningful way so that FinalIK animates the learner's body faithfully.\\
SteamVR registers the Vive Tracker in the order they are switched on. To increase the reliability of \exgo, a script was created that assigns the tracker by the hardware ID. The trackers are then assigned to a script called VRIKCalibrationController. The VRIKCalibrationController matches the tracker with the avatar and resizes the avatar to the learner's height. FinalIK is constructed to work with controllers in the user's hands. In \exgo, the experiment participant needs the hands to interact with the box, though the controllers are replaced with Vive Tracker on the back of the hands. Shifting the reference points of the hands yields a faithful representation of the learner's hands. The feet needed similar adjustments. Finally, FinalIK is able to solve the movements. Solving is the process of translating the tracker information into an animated avatar. For clarification, the complete rendering pipeline exemplary for the hip of the learner is attached in appendix~\ref{a:studentRenderingPipeline}.\\
VRIK requires calibration before use. For calibration, the person equipped with the trackers needs to perform a T-pose facing a specific direction. To ease the calibration process, a virtual mirror is placed in the room. The participant can be asked to look into the mirror and expand the arms, leading to the participant's correct orientation during the calibration process. Immediate with the calibration, the mirror disappears.\\
After the calibration, the system is ready to start with the task. Because the participant is now standing in front of the mirror, the position in front of the mirror is chosen as the starting point and endpoint of every task.\\
With the steps implemented in this section, the outcome results in a faithful representation of the learner, see figure~\ref{fig:selfPerception}.\\

The approach of translating a real-world person to the digital world with the help of Vive Trackers and IK was used by other researchers, too, for example~\cite{samesetup,perspectivematters}. 

\begin{figure}[H]
	\centering
	\includegraphics[width=\textwidth]{figures/selfPerception.png}	
	\caption[Digital representation of real-world person.]{Digital representation of real-world person with \exgo. a) real-world person, b) digital avatar representation in VR of the real-world person. Images not taken at the same time for pandemic reasons.}
	\label{fig:selfPerception}
\end{figure}

\section{Physical Artefacts}
\label{sec:artefacts}
\begin{figure}[H]
	\centering
	\includegraphics[width=\textwidth]{figures/artefacts.png}	
	\caption[Artefacts of \exgo]{Physical and corresponding digital artefacts of \exgo including measures in cm. a) box, b), table --- measures do not include the additional 6cm to attach the tracker, c) scale --- measures do not include the additional 5cm to attach the tracker.}
	\label{fig:artefacts}
\end{figure}
With the previously described elements in place, the learner can see the own body in an empty room. The task includes the handling of physical load on a table and a scale. The creation of the table, scale and the box, which will serve as physical load, starts with the construction of them - physically and digitally. Figure~\ref{fig:artefacts} shows the physical and digital versions of the box (a), table (b) and scale (c).\\
The first version of \exgo\ used a cardboard box (27cm x 26cm x 24cm) as physical load. During the development, it became clear that the cardboard box's size was too small and light to serve as a physical load. To determine a suitable size, several boxes of different dimensions were tested. With nine different boxes, a set of subtasks were performed. The major insight from this test was that the length of the box's sides should be unequal to see the direction of the box visually. Furthermore, the box should be perceived as physical load by having a reasonable size and a certain weight. Simultaneously, the box should not be too heavy and thereby limit the experiment participants to strong humans or be a threat to the participants' bodies. The sizes were set for 35cm x 30cm x 25cm. The measures of the box differ by 5cm in every dimension. This makes it clear to see the orientation of the box. The final box was constructed from three-layered wood with a strength of 27mm. The resulting weight was 5.8kg. To evaluate the weight of the box, one male participant\footnote{Evaluation with at least one male and one female person is desirable. This was not possible because of the COVID-19 pandemic.} was asked to perform all subtasks. The participant is a computer science PhD student and experienced with several sports activities. Observations revealed no incidences that contradict to use the box as physical load: the box could be safely held in hands, and it was visible that the person changed the own posture during the handling of the box to perform the movements more ergonomically. The posture change can be interpreted as a sign that the box is perceived as "load". The person rated the weight as "ok". He had no problems moving the box. The box was painted in three high contrast colours: black, white and red. Each opposite side was painted in the same colour. The painting facilitates the visual perception of the orientation of the box. The digital pendant of the box is a cuboid in the same colour and size. To translate the physical box's position and orientation to the virtual one, a Vive Tracker is attached to the box and fixated with a screw. The plus side of using a screw is the prevention of any relocation of the tracker. The downside is that tremors caused by placing the box on, for example, the table, are transferred directly into the tracker. This causes the tracker to lose tracking. To interrupt the transfer of tremors, shock-absorbing insulation is placed between the tracker and the box.\\
During the development of \exgo\, an office table (120cm x 60cm x 72cm) was used. The digital pendant to the physical table is a plate in the same size and colour as the tabletop. The position and orientation of the table are assessed by a Vive Tracker. Unfortunately, the tracker was placed on the top of the table inside the working area, where the box will be placed and shifted during the tasks. To shift the tracker out of the working area, a new tabletop was constructed. Because the used office table was too narrow, the width is increased by 10 cm. The new tabletop is out of three-layered wood, with an additional increased size of 6cm in length and width (126cm x 76cm instead of 120cm x 60cm) to provide additional space for the tracker. The tracker is attached on the most outer edge, thus out of the working area. To prevent tremors from passing from the table into the tracker, shock-absorbing insulation is applied between the tracker and tabletop.\\
The last artefact to create is the scale. The scale is a waypoint in the room where the participants perform \textit{lift} and \textit{lower} to the ground. The scale is a rectangular plate of 45cmx45cm so that the box can be placed on the scale easily. To shift the tracker out of the area where the box will be placed, the plate is extended by 5cm. The tracker is attached to the most outer edge of the extension with a screw and shock-absorbing insulation. The digital pendant is a plate with the exact dimensions of the physical plate, excluding the extended area. The physical artefacts were constructed and built by myself.\\

\section{Experiment Setting}
\label{sec:experimentSetting}
\begin{figure}[H]
	\centering
	\includegraphics[width=\textwidth]{figures/study_setting.png}
	\caption[Experiment setting]{Experiment setting with measures in cm.}.
	\label{fig:study_setting}
\end{figure}
Meanwhile, \exgo\ consists of a room, an avatar representing the learner, table, box and scale. In the following, an overview of the alignment of these elements is given. Figure~\ref{fig:study_setting} shows the real-world room in which the experiment is conducted. Figure~\ref{fig:learner_positions} depict all positions that are described in the task description (compare table~\ref{tab:tasks}). The outer line represents the Lighthouse or tracking volume, which is approximately 400cmx400cm. On the left wall, the mirror is located. In front of the mirror, the starting and end position of the box is seated. Beside the box is the position \textit{mirror}, the start and endpoint of the learner. The table is placed in the middle of the wall to the left of the mirror. Around the table, the positions \textit{table centre}, \textit{table right} and \textit{table left} are located. At the opposite wall of the mirror, the scale is placed. In front of the scale is the position \textit{scale}.
\begin{figure}[htb]
	\centering
	\includegraphics[width=\textwidth]{figures/learner_positions.png}
	\caption[Learner positions]{Learner positions. Green circles indicate positions of the learner mentioned in the task description in table~\ref{tab:tasks}. Red suqares mark artefacts, blue triangles mark Vive Trackers}
	\label{fig:learner_positions}
\end{figure}

\section{Guidance Visualisation}
\label{sec:gv}
The next task is to add the GV to \exgo, which the learner will mimic. The GV is an avatar like the learner's avatar, with the difference that the motion of the GV is driven by the pre-recorded tasks 1-3. The recording of the tasks was also performed with \exgo. To use \exgo\ as a recorder, a copy of all trackers and the HMD is created. This recorder-copy is packed in one GameObject with the trackers as children. The parent GameObject is recorded during the performance of the movements. For the GV, a similar GameObject as the recorder is created and serves as Input for VRIK. In this section, the main points of the process are described: the recording of the movements and the resizing of the GV to the size of the learner.\\

The movements in the task have to be performed ergonomically. The measures to evaluate ergonomic movements are the RM. To serve as a strong baseline, a professional for ergonomic movements should record the movements. Because of the COVID-19 pandemic, all attempts to record the movements by a professional failed. The whole laboratory was transported to a private facility, but because of temporal issues, the recording with the professional could not take place. Then the laboratory was transported to another private facility. Unfortunately, the room in which the laboratory was set up was not suitable for the recordings. The recorded movements by the professional had to be abandoned because of insufficient tracking coverage causing jitter. The laboratory was transported back to the university. Eventually, I was trained by a physiologist and recorded the movements by myself. The final recordings were examined by the physiologist. Overall, the movements were rated by the physiologist as "by and large correct". The back is not always straight or at the correct angle. In task 2 during a \textit{push} and in task 3 during a \textit{pull}, the feet are misplaced.\\

With the recording of the tasks at hand, the GV can be animated. For the ego-centric VP it is inevitable that GV and learner have the exact same size. Otherwise, the learner cannot perceive the GV correctly. Furthermore, the table, box, and scale must not resize. The solution is to record two sets of object synchronously. The first set contains the objects that have to be resized, namely the GV, the second set contains objects that must not be resized, namely the box, table and scale. The recordings were synchronised by a script, the playback of the animations in \exgo, too. The resizing of the GV takes place in three steps, compare figure~\ref{fig:resize}. First, the learner and the GV are calibrated. Then the height difference $\Delta y$ is measured between the learner and GV. In the last step, FinalIK is removed from the GV, the animated GameObject containing all trackers is removed, then FinalIK reattached and calibrated with the resized GameObject containing all trackers. Thus the learner and the GV have the same size.
\begin{figure}[htb]
	\centering
	\includegraphics[width=\textwidth]{figures/resize.png}
	\caption[Resizing the GV to the learner's height.]{Schematic description of the process to resizing the GV to the learner's height.}
	\label{fig:resize}
\end{figure}

\section{Visual Perspectives}
\label{sec:perspectives}
The next element to add to \exgo\ are the VPs the experiment will compare, namely the ego-centric VP, the exo-centric VP and the ego- \& exo-centric VP. The implementation of the VP is partly informed by Chua et al.~\cite{thaichichua}. Chua et al. used full-body high realism degree avatar representations as GVs teaching stationary Tai Chi. In the ego-centric VP, the GV is placed inside the learner's avatar. In the exo-centric VP Chua et al. decided to placed the virtual copy of the learner and the GV side by side.\\

The ego-centric VP requires, besides the learner, one ego-centric GV. The learner needs to stay inside the GV. This is achieved by the \textit{speed mechanic}, compare section~\ref{sec:mechanics}. The learner's distance to the GV is calculated with the help of the tracker at the hip of the learner and the recorded tracker at the hip of the GV. The positions of the trackers at the hip are projected to the floor. The projection to the floor is necessary because the \textit{speed mechanic} would apply if the GV bends the knees during \textit{lift} and \textit{lower}: if the GV bends the knees and the learner does not, the distance will increase between the two trackers in the y-component. This restricts the learner's ability to perform an error: if the learner does not bend the knees during \textit{lift} and \textit{lower}, the GV would stop and remind the learner to bend the knees. To investigate if the learner could percept to bend the knees, the learner must be allowed to make the error. This is why the \textit{speed mechanic} relies on the distance between the two projected points on the floor.\\
Additionally, the distance used for the \textit{speed mechanic} finds application in another functionality. In the ego-centric VP, the learner is located inside the teacher. This means that the learner's viewport is inside the head of the GV and let the learner see the inside of the GV head. This leads to distraction due to the partly rendered inner head. The solution is to remove the head rendering if the distance is below $ETD_{max}$ and reinitiate the rendering above $ETD_{max}$. The rendering is removed by replacing the materials array of the head with a material array that contains only invisible materials. Figure~\ref{fig:ego} visualises the ego-centric VP by showing the bird's-eye view (top) and the view from the ego perspective (bottom).\\

In the exo-centric VP, four exo-centric GVs are located around the learner. The positions of the exo-centric GV were determined after the task was recorded. The difficulty is to determine proper positions of the exo-centric GVs. First, at any point in time during every task's performance, the learner must be able to see a GV by only turning the head. Secondly, the GV and the learner should not move through a table or scale of another GV. The solution to the first part is informed by Chua et al.~\cite{thaichichua}. Chua et al. chose four representations that are in front, behind, left and right of the central learner. The second part proved to be impossible if the exo-centric representations should be near enough to be observed by the learner. A GV crossing through an artefact of another GV happens rarely and only during the subtask \textit{carry}, but is a limitation of \exgo. The GV needs to be shifted too far away from the learner not to cross other GVs artefacts. In a distance in which no crossing occurs, the movements are barely visible to be mimicked correctly. The exo-centric representations were then placed at a distance that allows being observed by the learner, and the learner can access all positions without standing in a digital artefact of another GV. The exo-centric GVs are positioned as follows. Standing at table centre and looking in the direction of the table: the GV to the left is shifted by two meters to the left, the GV to the right, two meters to the right. The GV in front is shifted by 1.5 meters to the front. The GV in the back is shifted 3 meters to the back. Figure~\ref{fig:multireppositions} shows the positions of the exo-centric VP. Additionally, to the exo-centric GVs, a virtual copy of the student needs to be rendered. The same values shift the virtual copy of the learner. Figure~\ref{fig:exo} visualises the exo-centric VP by showing the bird's-eye view (top) and the view from the ego perspective (bottom).\\

In the last VP, the ego- \& exo-centric VP, the learner has an ego-centric VP and the exo-centric GV with the corresponding virtual copies of the learner. The implementation of the ego- \& exo-centric VP is the union of the implementations of the ego-centric VP and exo-centric VP. Figure~\ref{fig:egoxo} visualises the ego- \& exo-centric VP by showing the bird's-eye view (top) and the view from the ego perspective (bottom).\\
\begin{figure}[htb]
	\centering
	\includegraphics[width=\textwidth]{figures/positions.png}
	\caption[Positions of representations.]{Position of representations.}
	\label{fig:multireppositions}
\end{figure}
\begin{figure}[H]
	\centering
	\includegraphics[width=\textwidth]{figures/perspectiveEGO.png}
	\caption[Ego-centric visual perspective]{Ego-centric VP from the bird's-eye view (top) and ego perspective (bottom) on the exact same scene. The GV in located inside the learner.}
	\label{fig:ego}
\end{figure}
\begin{figure}[H]
	\centering
	\includegraphics[width=\textwidth]{figures/perspectiveEXO.png}
	\caption[Exo-centric visual perspective]{Exo-centric VP from the bird's-eye view (top) and ego perspective (bottom) on the exact same scene. The GV's are located around the learner. Additionally, a virtual copy of the learner is located inside the exo-centric GVs.}
	\label{fig:exo}
\end{figure}
\begin{figure}[H]
	\centering
	\includegraphics[width=\textwidth]{figures/perspectiveEGOEXO.png}
	\caption[Ego- \& exo-centric visual perspective]{Ego- \& exo-centric VP from the bird's-eye view (top) and ego perspective (bottom) on the exact same scene. The GV's are located around the learner as well as inside. Additionally, a virtual copy of the learner is located inside the exo-centric GVs.}
	\label{fig:egoexo}
\end{figure}

\section{Quantitative Data Aquisition}
\label{sec:logging}
Section~\ref{sec:measures} defined the measures that are necessary to answer the research questions. \exgo\ must be capable of assessing all measures. This section explains how \exgo\ assesses the measures. An overview of all measures is listed in table~\ref{tab:logging_detail}. Table~\ref{tab:logging_detail} lists the logging ID, a description of what the measurement is measuring, the unit in which the measurement is measured and for which research question the measurement is assessed. Quantitative data acquisition can be divided into several classes: (i) accuracy measurements (1-5), (ii) ergonomic measurements (6), (iii) focus measurement (7) and (iv) time measurement (9). In the following, (i)-(iv) are explained in detail.
\begin{table}[H]
	\centering
	\includegraphics[width=\textwidth]{figures/logging_detail.png}
	\caption[Logfile description]{Detailed overview of logs produced by \exgo\ per frame. L: learner, GV guidance visualisation, ED: euclidean distance. *head position and rotation is biased in exo-centric conditions because of multiple GV the L can focus on. **All trackers are logged for backup reasons: after the experiment is conducted, a measurement can become interesting that was not of importance before. With these values, any measurement can be calculated post-experiment.}
	\label{tab:logging_detail}
\end{table}

\subsection{(i) Accuracy Measurements}
The accuracy measurements assess the discrepancy between the movements of the learner and the movements of the GV. Accuracy measurements are subdivided into distance-based measures and angle-based measure. Distance-based measures rely on the Euclidean distance between the learner's body parts and the body parts of the GV. The reference point for the body part is the tracker, which is attached to the body part. The body parts are: hip, left hand, right hand, left foot, right foot and head. The distance between the learner's box and the box of the teacher is an accuracy measurement, too. Similar to the body parts, the distance between the two boxes is the Euclidean distance between the tracker of the box and the recorded tracker of the GV. Please note, the trackers are not visible to the learner during the experiment.\\
Angle-based accuracy measurements assess the discrepancy in orientation between the body parts and box of the learner and the GV. The angles are measured in degrees. The calculation of the angle-based measurements complies with the calculations of the distance-based measurements. This means the angle between the corresponding trackers is measured. To summarise: distance-based measurements assess the positioning's error, angle-based measurements assess the error in orientation.\\

Similar accuracy measurements are used by~\cite{onebody,thaichichua,YouMove,physioathome,vrdancetrainer,lightguide}.

\subsection{(ii) Ergonomic Measurements}
\label{sec:ergonomicMeasurements}
\begin{figure}[H]
	\centering
	\includegraphics[width=\textwidth]{figures/riskMeasurements.png}
	\caption[Risk metrics calculation]{Calculation of the risk metrics. a) \textit{upright stance} --- angle between the upright vector and the vector from the L5 tracker to the T8 tracker, b) \textit{squat} --- Euclidean distance between L5 tracker and floor, c) \textit{good base} --- Euclidean distance between the feet tracker, d) \textit{box-near-body} --- distance between L5 tracker and box tracker.}
	\label{fig:rmCalc}
\end{figure}
The ergonomic measurements are the four injury risk metrics: \textit{good base}, \textit{squat}, \textit{upright stance}, and \textit{box-near-body}.\\
\textit{good base} is the distance between the feet, compare figure~\ref{fig:rmCalc} c). For \textit{push} and \textit{pull}, \textit{lift} and \textit{lower}, \textit{turn} and \textit{fold}, \textit{pick} and \textit{place} each a window in which the distance should be located is defined (see next of paragraph). The percentage of time the learner is inside the desired window is the outcome of the measurement. For a better understanding, imagine the following exemplenary  scenario: the window during \textit{push} for the \textit{good base} is 20cm-30cm. During the performance of \textit{push}, the learner's feet distance was inside the window for 90 seconds. The whole performance of \textit{push} lasted 100 seconds. The RM \textit{good base} yields in a score of 90\%.\\
The measurement for \textit{squat} is the distance between the hip and floor, compare figure~\ref{fig:rmCalc} b). It indicates if the learner bent the knees correctly and is applied in the subtasks \textit{lift} and \textit{lower}. A window is defined for \textit{squat}, too. Calculations of the RM score of \textit{squat} complies with the \textit{good base}.\\
\textit{Upright stance} is the measurement of the spine bend, compare figure~\ref{fig:rmCalc} a), which should be in a specific window, too. For \textit{upright stance}, an additional tracker is applied to the back of the learner at Vertebra thoracalis 8 (T8), which is around 20cm kranial of the lower hip tracker at Vertebra lumablis 5 (L5)\footnote{Latin description: Dr. med. univ. Kilian Roth}, compare figure~\ref{fig:tracker_placement} c). The angle of spine bend is the angle between the upright vector, and the vector of the upper hip tracker\footnote{Implementation for the calculations of the spine bend angle informed by Tanveer Singh Mahendra.}. \textit{Upright stance} is applied for \textit{push} and \textit{pull}, \textit{lift} and \textit{lower}, \textit{turn} and \textit{fold}. The bend angle during \textit{pick} and \textit{place} depends heavily on the box's position on the table and thereby varies. Because of this variation, a window cannot be defined for \textit{pick} and \textit{place}, and though the RM \textit{upright stance} is not applied to \textit{pick} and \textit{place}. Calculations of the RM score of \textit{upright stance} complies with the calculations of the preceding RM.\\
\textit{Box-near-body} is the Euclidean distance between the hip tracker and the box tracker, compare figure~\ref{fig:rmCalc} d). It is applied for the subtask \textit{carry}. The calculations comply with the preceding RMs. The limitation of \textit{box-near-body} is that the measurement is influenced by the circumference of the learner's torso. A formative test of \textit{box-near-body} was not possible due to the COVID-19 pandemic.\\

The definitions of windows for the RMs were planned to be done by a professional with reasonable knowledge about ergonomics. Unfortunately, because of the COVID-19 pandemic, it was not possible to invite the professional to the laboratory to define the windows for the RMs. This means for the pilot study that the RMs cannot be evaluated.

\subsection{(iii) Focus Measurement}
\label{sec:rayTrace}
\begin{figure}[H]
	\centering
	\includegraphics[width=\textwidth]{figures/focus.png}
	\caption[Focus assessment]{Visual focus assessment. Ego perspective of the learner. The ray is depicted with a green line.}
	\label{fig:focusAssessment}
\end{figure}
The virtual room the learner sees in \exgo\ is filled with tables, boxes, GVs and virtual copies of the learner. To assess on what the learner is focussing during the movements, every object was given a name. In every frame, raytracing is performed. The rays' origin is the HDM and expands straight forward. The name of the object first hit by the ray is written into the log file. The name is coded with the position 0-4 (see positions in firgure~\ref{fig:learner_positions}) and an object identifier (box, table, scale, GV, learner), compare figure~\ref{fig:focusAssessment}. A test revealed a systematic error of the ray pointing too high. To correct the discrepancy, the colliders of the objects were increased. The tables' and scales' collider height is increased by 20cm. The box colliders height is doubled. The learners' and GVs' avatar were wrapped into a capsule collider with a height 200cm of and a radius of 30cm. The values were determined by experimentation. To test the values, all subtasks were performed, and the object which was hit by the ray is displayed. The displayed name complied with the object in focus at nearly any point in time. Using an eye-tracker would increase the accuracy but was unavailable.


\subsection{(iv) Time Measurement}
The animation speed of the GV is determined by the distance between the learner and the GV (\textit{speed mechanic}). A slower played GV animation results in a longer task completion time. For ease of understanding, please consider the following two definitions: the time the task lasts without the \textit{speed mechanic} is defined as \textit{task norm duration} (TND). The time the learner needs more than the TND to fulfil the task is defined as \textit{over task norm duration} (OTND.)\\
OTND can draw conclusions about the learner's position in relation to the GV.
The tasks differ in the amount of time to be performed:
\begin{itemize}
	\item TND task 1: 172128ms
	\item TND task 2: 189040ms
	\item TND task 3: 176668ms
\end{itemize}
The OTND can be applied to specific subtasks, too. This measurement will mainly be used in the evaluation for triangulation.

Using the TCT for the evaluation of movements was previously done, for example by~\cite{onebody,YouMove,perspectivematters}.

\section{Qualitative Data Aquisition}
\label{sec:quali_logging}
The qualitative data assessment relies on one questionnaire after each session (\textit{after-session questionaire}) and a semi-structured interview after all three sessions (\textit{semi-structured intervie}w). The qualitative data assessment aims to determine the participant's impressions and opinions about the VPs. In the questionnaires, a different wording is applied to ease understanding. For example, the GV is called virtual teacher. 

\subsection{After Session Questionaire}
The \textit{after-session questionnaire} (Appendix~\ref{appendix:questionaire}) starts with a question about the subjective overall performance of the learner. 
\begin{itemize}
	\item[Q1:] How accurate did your movements match with the virtual teacher?
	\item[A:] Likert scale from one (very good) to 7 (very poor)
	\item Linked research questions: RQ1.1.1-3, RQ1.4
	\item Data triangulation for (1-4,6)
\end{itemize}
The answer to this question gives insights into how accurate the participant judges the performed movements. Furthermore, this question can be used to determine if the qualitative accuracy complies with the participants' subjective opinion. The next question aims to assess the user's subjective performance for the subtasks. The participant is asked to fill in a table. Each line represents a subtask. Each subtask can be rated on a Likert scale from 1 (very good) to 7 (very poor).
\begin{itemize}
	\item[Q2:] During the task, there were several smaller reoccurring movements, like pulling or lifting the box. Please rate these smaller movements to what extent you could follow the movements: 1 (very good) to 7 (very poor).
	\item[A:] Likert scale from 1 (very good) to 7 (very poor) for each subtask.
	\item Linked research questions: RQ1.1.1-3, RQ1.4
	\item Triangulation for (1-4,5,6)
\end{itemize}

\begin{figure}[H]
	\centering
	\includegraphics[width=0.6\textwidth]{figures/sub-task-rating.png}
	\caption[Rating template: subtasks]{Rating template for the subtasks.}
	\label{fig:subtaskrating}
\end{figure}

Beside the objective opition about the participants performance, the answers of this question can be used to compare with the qualitative data. Question three aims to assess the subjective accuracy of the participants body parts.
\begin{itemize}
	\item[Q3:] Please rate to what extend you think you could align your body parts with the teachers body parts: 1 (very good) to 7 (very poor). 
	\item[A:] Likert scale from 1 (very good) to 7 (very poor) for each body part.
	\item Linked resarch questions: RQ1.1.1-3, RQ1.4
	\item Triangulation for (1-4,5,6)
\end{itemize} 
\begin{figure}[H]
	\centering
	\includegraphics[width=0.6\textwidth]{figures/body-parts-acc.png}
	\caption[Rating template: body parts]{Rating template for the body parts.}
	\label{fig:bodypartsacc}
\end{figure}
This question assesses how good or bad the participant could perceive the body parts of the GV. The last question is not handed to the participant. It serves as the basis for a subsequent semi-structured interview question. It gives the possibility to dig into extreme values of the questions answered before and into incidents that occurred during the performance of the task.
\begin{itemize}
	\item[Q4:] (As interview question) Did you have problems to follow the instructions? 
	\begin{itemize}
		\item E.g. because you could not see some body parts?
		\item E.g. bad perception related to the perspective?
		\item Go into extreme values of this questionnaire! 
		\item Address critical incidences!
	\end{itemize}
	\item[A:] Notes of participants statements.
\end{itemize}

\subsection{Semi-Structured Interview}
After all three sessions are done, the participant is interviewed with the help of the semi-structured interview guideline (Appendix~\ref{appendix:interview}). The guideline contains seven main questions, some with additional hints to dig deeper or to lower the participant's entry threshold to start reporting.
\begin{itemize}
	\item[Q5:] You saw three visual perspectives: ego-centric, exo-centric and the combination. What do you think about these perspectives?
	\begin{itemize}
		\item entry question, encourage participant to talk frank, address interesting statements.
	\end{itemize}
	\item[] Linked research question: RQ1.4
	
	\item[Q6:] Prioritise the perspectives by how accurate you could follow the movements. (1 best to 3 worst) 	
	\begin{itemize}
		\item Why did you prioritize this way?
	\end{itemize}
	\item[] Linked research question: RQ1.4
	
	\item[Q7:] Imagine you want to learn a movement in VR. Which perspective would you use for that?
	\begin{itemize}
		\item Or would you use a totally different one?
	\end{itemize}
	\item[] Linked research question: RQ1.4
	
	\item[Q8:] Which of the three perspectives was the easiest to understand? 
	\begin{itemize}
		\item Was there a perspective that confused you?
		\begin{itemize}
			\item What do you think caused the confusion?
		\end{itemize}
		\item Was there a perspective you did not understand right away?
	\end{itemize}
	\item[] Linked research question: RQ1.4
	
	\item[Q9:] What do you think are the advantages and disadvantages of the perspectives?
	\item[] Linked research question: RQ1.4
	
	\item[Q10:] Could you see some body parts better or worse in the perspectives?
	\begin{itemize}
		\item What about your legs, arms, back?
		\item Could you detect that during \textit{lift} and \textit{lower} you should squat?
		\item Could you detect that you should step back during \textit{push} and \textit{pull}?
	\end{itemize}
	\item[] Linked research question: RQ1.4
	
	\item[Q11:] Did you miss a feature?
	\begin{itemize}
		\item Dig for improvements for \exgo\ or experiment design.
	\end{itemize}
	
	\item[Q12:] (Space to ask for critical incidences if any occurred.)
\end{itemize}

\section{Limitations}
\label{sec:limitations}
\exgo\ and the experiment is designed for a task that includes the handling of physical load. If the results of the experiment can be applied to movements without a physical load is questionable. Additionally, it cannot be assumed that the results of the experiment can be transferred for other physical loads that are significantly different in shape and weight. Furthermore, the exo-centric GV sometimes walk through artefacts (table, scale) of other GVs, which can confuse the experiment participant. The movements are not recorded by a professional, errors in ergonomic movements are possible. Furthermore, the subtasks have a specific magnitude. They can not stand for the same subtask with different magnitude. This means, for example, for \textit{lift}, the application of the outcome of the experiment for lifting up a box above the head is limited. Lastly, only a small number of participants participated in the formative tests to evaluate partial aspects of \exgo. Especially, the hip-box distance is not tested because multiple persons with different physique would be necessary. The artefact contribution of this master's thesis, locomotion guidance in the ego-centric VP, is limited by being evaluated by one participant.\\

\section{Experiment Procedure}
\label{sec:procedure}
\exgo\ is now complete, and the elements of the experiment design can be assembled with the technological elements of \exgo\ to form the final experiment procedure.\\
As soon as the participant enters the room, the participant receives a warm welcome to feel comfortable. The process starts with a welcome letter (appendix C), followed by the informed consent and a demographic questionnaire. In the meantime, \exgo\ is set up by choosing the condition, set the gender of the participant as well as the log is configured with the participant ID and task ID. After the demographic questionnaire, a spoken explanation about what is about to happen is given. Then the trackers are attached to the participant. The calibration process is explained ("Look into the mirror and extend your arms."). An explanation of the perspective is provided. Then, the first session is started. \exgo\ gets started, the cameras and screen recording are set up, and the participant gets the HMD. The participant is invited to calibrate. For calibration, the key C is pressed at the PC. To identify the camera recordings, a sign is held into the cameras. The task is started with the key S. During the session, the study conductor pays attention to the cable of the HMD to keep the participant from stumbling over it. Furthermore, the participant is observed. After the session ended, the HMD is removed. The participant fills in the after session questionnaire. Session two and three are conducted likewise. After all three sessions, the trackers are removed. The participant is interviewed. The payment is given, and the receipt signed. At the very end, the participant is thanked and said goodby. If it appears, doorstep talk is appreciated.


\chapter{Experiment Evaluation}
\label{chapter:study_evaluation}
The experiment presented in the previous chapter is evaluated with the help of a pilot study. For practising the conduction of the pilot study, one participant was invited before the actual pilot study. The pilot study was conducted with three male particiants between 25 and 29 years from the Computer Science department. All participants are experienced with VR devices, have rudimentary knowledge about the ergonomics of movements and rarely carry out tasks that include the handling of physical load. All of them are near-sighted, two of them weared glasses beneath the HMD. One participant previously used digital guidance for to conduct sports ergonomically. The purpos of the pilot study was to identify flaws in the experiment procedure (section~\ref{sec:evalProcedure}), \exgo including the assessment of the quantitative data(section~\ref{sec:evalSystem}) and in the essessment of the qualitative data (section~\ref{sec:evalQuali}). Furthermore, the eligibility of the acclimatisation method is discussed in section~\ref{sec:evalAccl}. Finally, a glimps on the data produced by the pilot study is provided in section~\ref{sec:evalDataAna}.

\section{Procedure}
\label{sec:evalProcedure}
In the beginning, the participant receives a welcome letter. Two participants stated that the information in the welcome letter (Appendix~\ref{appendix:welcomeLetter}) about how VR headsets work is not necessary. The welcome letter was then shortened by the removal of that passage. The informed consent is signed. The following demographic questionnaire allows putting the study's data into context. During the review of the produced study data, no further questions occurred. The demographic questionnaire needs no further improvements. While the participant reads the welcome letter and fills in the demographic questionnaire, the system is set up. The pilot study showed that the time to set up the system and reading the welcome letter and filling in the demographic questionnaire are corresponding.\\
Afterwards, the participant is told that he/she will be equipped with the trackers and where the trackers will be attached. Afterwards, the trackers are attached to the participant. To respect the privacy of the participant, the trackers are handed to the participant and instructions are given on how to attach them. The pilot study showed that the participants had problems following the instructions correctly. For the actual study, the participants will be asked if it is ok that the trackers are attached with the physical help of the study conductor.\\
Next, the participants received information about what to expect in the VR. The instructions contained information about the GV ("You will see one/multiple teachers.") and the task ("Please follow the instructions of the teachers as exactly as possible."). Furthermore, the participant was asked to pay attention to the ergonomics of the movements. Explanations about the speed-mechanic were provided, too ("The teacher will wait for you if you are too far away from the teacher. If that is happening, correct the placements of your feet, and it will go on."). No participant had difficulties understanding the instructions. Additionally, the participant was handed the box to get used to it before seeing its digital pendant in VR. Finally, the calibration of the system was explained ("Please look into the mirror which you will see there (study conductor pointing) and extend your arms like this (study conductor performing the T-pose)"). All participants understood how to calibrate easily. The introduction of the mirror as calibration facilitator proved to be helpful and suitable.\\
Subsequently, the camera recordings were started, the participant put on the HMD and the participant performed the first task. In this phase of the study, two errors occurred which needed adjustments to the process. In one case, the wrong task was chosen, which made participant 1 (PT1) perform task 1 (T1) two times in different perspectives. PT1 recognised that, too. Before starting the task, an additional checklist should be gone through, to ensure the correct task and perspective is chosen by the study conductor. The second error regards the identification and synchronisation of the video recordings. At the ceiling, a GoPro films the scene from above. A second camera catches the scene from the side of the tracking volume. For identification, a sign was held into the view of both cameras. This was forgotten twice by the study conductor. As an improvement, the sign should be placed beforehand in the area both cameras cover.\\
After the participant performed the first task, the participant took off the HMD and is asked to fill in the after-session questionnaire. The trackers stayed at the body of the participant. The pilot study showed that the tracker did not hinder the participants from sitting down and filling in the questionnaire. In the pilot study, a three-minute pause was conducted to allow the participant to recover. During that pause, the participant was asked about his/her wellbeing to check for VR induced motion sickness. All participants stated that they do not need a pause. The demographic questionnaire revealed that all participants were experienced with VR-system and they are used to wear VR HMDs. The pause will be maintained, because a person with no prior exposition to VR could feel different.\\
Session two and three are conducted in the same way as session one. With all three sessions done, the trackers are removed, and the semi-structured interview was conducted. Because the pilot participants were not paid, the pilot study ended here. The planned duration of the study was 75 minutes. All pilot studies took no longer than 55 minutes. However, a time buffer should be planned in, in case some participants need more time for the experiment. With an additional buffer of ten minutes, the planned study duration can be decreased to 65 minutes.\\
Additionally, for the study's evaluation, the participants were interviewed to get insights about the studie's and system's flaws. The participants were asked if the explanations were sufficient and if there were any confusing elements or unclear questions in the questionnaires or documents. Finally, the participants' opinion about possible improvements of \exgo\ and the study were asked. The results of those interviews informed section~\ref{sec:studyStructure}, too.\\
To conclude, vast parts of the planned study process proved to be suitable. Adjustments are made to the welcome letter, the trackers' attachment with the study conductor's help, an additional checklist to check the session task and perspective is introduced, and the camera recording identification is improved by placing the sign into the recording area beforehand.

\section{\exgo\ and Quantitative Data Assessment}
\label{sec:evalSystem}
All hardware artefacts of \exgo\ are suitable without objections. The study participants rated the box's size and weight as "okay" while still perceiving it as a physical load. The table's size is sufficient for all three tasks. At no time, the participants were in danger to collide with a physical artefact. The size of the scale is also sufficient. The box was always placed on the scale safely. The positions and itinerary between mirror, table and scale are without complaint. Regarding the hardware part of \exgo, the pilot study revealed two insights, one related to the trackers and one related to the HMD. The tracker at the hip is attached with a strap around the hip of the participant. While the subtasks \textit{lift} and \textit{lower}, the tracker is shifted upwards. The upwards shift affects the avatar's presentation and influences the accuracy measurements hip distance and the RM squat distance and upright stance. To prevent this, the student was asked to wear a belt. The tracker belt was then fixated to the participant belt with a band of velcro. This includes touching the participants in the lower hip area from behind. To prevent participants from feeling uncomfortable during the whole study, the fixation of the two belts should be performed with a clip that the participant can attach themselves. The second insight regarding the hardware of \exgo\ is the cable of the HMD. During the study, the study conductor handled the cable not to influence the participant. In one case, the cable was plugged out during a session. \exgo\ is designed for that case, and plugging in the cable again allows to continue with the session. However, in the actual study, this incident would lead to unusable data for all three sessions of that participant because, meanwhile the cable is plugged in again, the GV will move forward, and the error will be high during this phase. The actual study will benefit from a wireless HMD.\\
One participant stated that he/she could not identify the ownership of the box right away. As soon as the own box is in the participant's hands, it is no problem to tell which is the GV's box and which is the learner's box. However, if both boxes are stationary, the participant could not detect which is the own box. The box should be changed to light transparency for a better distinction between the learner's box and the GV's box. This will also have an influence on the perception of the box during \textit{lift}. During \textit{lift}, the learner's box is occluded by the GV's box for a short time. For conformity, the avatar, table and scale of the GV's should also be rendered with a light transparency.\\
The pilot study served as the last test before the actual study. An essential part of the pilot study is the review of the produced data. During the development, the measures could only be tested individually. The pilot study allowed for the first time to get a whole image. Fortunately, most of the logged data worked as intended. Only minor errors were detected. For spine bend, a last-minute edit caused an incorrect calculation. For EXO, in combination with task 2, an invisible error was detected: the props animation controller, which animates the GV's box, played the wrong task for the ego-centric GV. In EXO, the ego-centric GV is invisible but used to calculate the measurements. For EGO, the learner's avatar identification name (used for looking at, identifying what the learner is visually focussing on) is incorrect. Lastly, Unity3D natively uses comma as decimal separator. Most statistics programs natively use point as decimal separator. A log file of one session contains around 2.5 Million decimal separators. Converting a log file is time-consuming, and therefore, the logging should be changed to use a point as decimal separator.

\section{Qualitative Data Assessment}
\label{sec:evalQuali}
After each task, the participant fills in the \textit{after-session questionnaire} (Appendix~\ref{appendix:questionaire}). The last question of the questionnaire is not handed to the participant but asked personally. The feedback of the pilot study's participant was positive, and the three questions were clear and understandable. However, during the analysis of the quantitative data some aspects became interesting to consider the opinion of the participants. Therefore, the \textit{after-session questionnaire} is extended by the following questions:
\begin{itemize}
	\item[Q:] How ergonomic do you think your movements have been?
	\item[A:] Likert scale from 1 (very good) - 7 (very poor)
	\item[] Linked research questions: RQ1.2
	\item[] The self-assessment of the study participants' performance regarding the ergonomics of the movements can give insights into the participants' feelings. It is expected that the VPs with exo-centiric GVs is rated higher because the posture of the GV can be observed from the side. For example, the back of the GV's bend angle is hard to see in the ego-centric VP. Furthermore, the self-assessment can be put into relation to the quantitative data and used for triangulation.
	\item[Q:] On what did you focus most?
	\item[A:] Two choices: box, teacher
	\item[] Linked research questions: RQ1.3
	\item[] The assessment of the subjective participants' focus can give insights about the importance of box and avatar for the participants. Presumably, a participant who rates the box over the avatar focussed mainly on the box's accuracy and vice versa. This subjective data can be put into relation to the quantitative data for triangulation. If the quantitative data complies with the qualitative data, the evaluation of the risk metrics can be split into two groups: those who focussed mainly on the box and those who focussed mainly on the avatar. It could be interesting if these two groups score differently regarding the risk metrics.
	\item[Q:] Could you see a teacher at any point in time?
	\item[A:] Two choices: yes, no
	\item[] Linked research questions: none, evaluation of the positions of the GVs
	\item[] The positions of the GVs could only be tested sparsely. This question aims to evaluate if the positions of the GVs were suitable. If the participant crosses "no" in last question this can be addressed in the during the last personally asked question.
\end{itemize}
After all three sessions, the participant is interviewed with the help of the semi-structured interview guideline (Appendix~\ref{appendix:interview}). The guideline for the semi-structured interview proved to be suitable. However, during the interviews, one participant could not give a clear prioritisation of which perspective he/she would use if he/she had the choice. I gave in and did not insist on the prioritisation. This caused an inconsistency which becomes clear during the analysis of the qualitative data. In the actual study, every participant should provide a prioritisation.

\section{Acclimatisation Phase}
\label{sec:evalAccl}
The first task of each study is fo acclimatisation, where the learner gets used to \exgo. Thus, the data of the first task is excluded from the evaluation. It is assumed that the learning effect between session one and two is high, and between two and three, nearly no learning effect occurs. To evaluate if this assumption holds, the task completion time (TCT) could give insights because of the \textit{speed mechanic}. The \textit{speed mechanic} regulates the GV animation speed based on the distance between learner and GV and is applied in all perspectives. The higher the learner-GV distance, the lower the speed. A learner who is often located near to the ideal point yields a lower TCT. Comparing the TCT in the pilot test could at least indicate if the learning effect between session two and three is low by showing similar TCT.\\
Figure~\ref{fig:tct} shows the amount of ms the participants needed more than than the TND to complete the task (OTND). The order of the sessions is from left to right. Participant 1 (PT1) had a nearly equal OTND for all three tasks. Because of a mistake in choosing the pilot study task, PT1 faced T1 two times. However, in session three, the OTNT is slightly higher. PT2 shows the expected behaviour, having a high OTND in the first session and a nearly equal OTND for session two and three. PT3's OTND is strictly monotonically decreasing. If the OTNT behaviour of all the participants would be like PT2's behaviour, the choice of the acclimatisation method is correct. If the OTNT behaviour for all participants would look like PT3, the acclimatisation method would be incorrect. In this case a seperate condtition specific acclimatisation before every session should be conducted. If the OTND behavior for all participant would look like PT1, it could be discussed that no acclimatisation is necessary. Unfortunately, the data is ambiguous and does not allow an evaluation of the acclimatisation method. Thereby, using the first session for acclimatisation is maintained.

\begin{figure}[htb]
	\centering
	\includegraphics[width=0.8\textwidth]{figures/msOverTaskNorm.png}
	\caption[OTND of all sessions.]{Amount of miliseconds OTND per session. Prefix PT --- participant, Prefix T --- task.}
	\label{fig:tct}
\end{figure}

\section{Data Analysis}
\label{sec:evalDataAna}
Based on the pilot study, this section tries to give a first glimpse on the produced data. A pilot study serves to identify issues and faults in the system and study, and prepare the final study conduction. The previous sections described found issues and faults and the proposed solution for them. Some issues and faults impacted the data, which led to the exclusion of corresponding data. Furthermore, as described in section~\ref{sec:studyStructure}, for a full counterbalancing of tasks and conditions requires at least nine participants. Additionally, the data revealed, in some aspects, a high variation in both qualitative and quantitative data. Therefore, the depicted data is a rough estimation, and conclusions can not be drawn. The analysis is superficial, and a detailed analysis like significant verification is renounced. However, the first impression of a possible outcome can be given. All charts depicted in this section are similarly structured. The conditions in all charts have the same colour coding: \textcolor{blue}{EGO} is depicted in blue, \textcolor{orange}{EXO} is depicted in orange and the combination \textcolor{gray}{EGO \& EXO} is depicted in gray. For all charts (except for head angle) holds: the lower the bar, the better in the corresponding context.\\
Figure~\ref{fig:cumulatedError} can be perceived as an abstract for this section by showing the overall average error in distance and angle between the learner and the GV per VP. The ego-centric VP outperformed both the exo-centric VP and the ego- \& exo-centric VP in terms of accuracy, while the ego- \& exo-centric VP mostly scored better than the exo-centric VP. The subjective accuracy is also highest in the ego-centric VP. In contrast to the objective accuracy, the participants rated their accuracy worst in the ego- \& exo-centric VP.
\begin{figure}[htb]
	\centering
	\includegraphics[width=0.32\textwidth]{figures/averageDistanceError.png}
	\includegraphics[width=0.32\textwidth]{figures/averageAngleError.png}
	\includegraphics[width=0.32\textwidth]{figures/averageSubjectiveAccuracy.png}
	\caption[Overall accuracy]{Oberall average error in distance (left) and angle (middle). Overall subjective accuracy (right).}
	\label{fig:cumulatedError}
\end{figure}


\subsection{Accuracy}
Accuracy is clustered by distance and angle and applied for the body parts hands, feet, hip, head and box. Distance means the Euclidean distance between the learner's, e.g. hand and the GV's hand in meters, and describes the difference in position. Angle describes the difference in orientation and is measured in degrees. The overall average error per body is depicted in~\ref{fig:overallError}.\\
Section~\ref{sec:studyStructure} showed that some subtasks could be paired up, based on the similarity of the movements: \textit{lift}/\textit{lower}, \textit{push}/\textit{pull}, \textit{turn}/\textit{fold} and \textit{pick}/\textit{place}. Figure~\ref{fig:avgErrorPerSubTask} shows the average error in distance and angle per subtask and confirms the pairing of the subtasks by showing a relation between the pairs. Thus, the pairs of subtasks will be analysed in combination. \textit{Carry} and \textit{walk} are analysed seperately. This section analysis the accuracy based on the body parts (section~\ref{sec:evalAccBodyPart}: Accuracy per Body Part) and the accuracy during the subtasks (section~\ref{sec:evalAccSubtask}: Accuracy per Subtask) and compares the objective accuracy with the subjective accuracy.

\subsubsection{Accuracy per Body Part}
\label{sec:evalAccBodyPart}
\begin{figure}[H]
	\centering
	\includegraphics[width=0.49\textwidth]{figures/overallDistanceError.png}
	\includegraphics[width=0.49\textwidth]{figures/overallAngleError.png}
	\includegraphics[width=0.49\textwidth]{figures/subjectiveAccuracyPerBodyPart.png}
	\caption[Overall accuracy per body part]{Average error per body part. Left: distance error, right: angle error. Suffix D: distance, suffix A: angle. LH - left hand, RH - right hand, LF - left foot, RF - right foot.  Bottom: subjective accuracy per body part, Likert scale 1 (best) - 7 (worst).}
	\label{fig:overallError}
\end{figure}

The \textbf{hip} error indicates to what extent the learner could determine the correct location. The data indicates that the determination of one's own position and rotation is more straightforward with an ego-centric GV.\\
Figure~\ref{fig:overallError} shows a relation between the distance error of \textbf{hands}. This is expected since large parts have synchronised movements. For example, the hands touch the box simultaneously. The hands' error is lower in EGO than in EXO. Hands are directly visible in front of the learner and the direct comparison to the ego-centric GV is a possible explanation for the higher accuracy in EGO.\\
Surprisingly, \textbf{feet}'s error is lower in EGO, too. To see the feet, the learner must actively move the head, primarily if the box blocks the view on the feet. However, it seems easier to align the learner's feet with the GV feet in EGO.\\
The \textbf{head} angle is not comparable with the other angle-based accuracy measures. The presence of multiple exo-centric GVs in the EXO forces the learner to look into different directions. The difference between EXO and \combi\ could point out that the learner focussed in \combi\ on both the exo-centric GVs and ego-centric GV. The angle-based accuracy for the hand and feet reveal no clear trend. More participants could provide a clearer view.\\
The \textbf{box} distance and angle error are lower in EGO and \combi than in EXO. The presence of an ego-centric GV box increases the distance and angle accuracy for the box.\\

To summarise, the presence of an ego-centric GV increases the distance accuracy for all body parts. For the physical load, the presence of an ego-centric GV increases the angle accuracy, too. However, adding exo-centric GVs to an ego-centric GV increases the distance error. A possible reason why adding an ego-centric GV to exo-centric GVs could be, that the learner shares the focus with multiple GVs or that the presence of four exo-centric GVs overwhelms the learner. The study participants rated their overall accuracy highest in EGO and lowest in \combi, compare~\ref{fig:overallSubjectiveAccuracy} (left). The low subjective accuracy in \combi\ would underpin the theory that the learners were overwhelmed. However, when the participants were asked about their subjective accuracy for the body parts, arms, legs and back, a different picture arises, compare~\ref{fig:overallSubjectiveAccuracy}. The participants' opinion is differentiated, which causes a high standard deviation of around 1.5 on a Likert scale from 1-7. More participants could lead to a clearer view.\\

The described insights rely on the whole task containing all subtasks. Thus, the drawn deductions are only valid for a the whole task. Potentially, the accuracy in specific subtasks could differ from the overall accuracy. In the following, the subtasks are analysed and vetted if the deductions count for the specific subtasks.

\subsubsection{Accuracy per Subtask}
\label{sec:evalAccSubtask}
Figure~\ref{fig:avgErrorPerSubTask} depicts the overall distance and angle error for all subtasks, as well as the subjective accuracy for all subtasks. In all subtasks except \textit{pull}, the ego-centic VP yielded the highest accuracy. The next sections discuss the subtask's accuracy in detail.
\begin{figure}[H]
	\centering
	\includegraphics[width=0.49\textwidth]{figures/averageDistanceErrorPerSubTask.png}
	\includegraphics[width=0.49\textwidth]{figures/averageAngleErrorPerSubTask.png}
	\includegraphics[width=0.49\textwidth]{figures/subjectiveAccuracyBySubTask.png}
	\caption[Overall accuracy per subtask]{Overall average error per subtask. Left: distance error, right: angle error. Bottom: subjective accuracy per subtask, Likert scale 1 (best) - 7 (worst).}
	\label{fig:avgErrorPerSubTask}
\end{figure}

\subsubsection{\textit{lift}/\textit{lower}}
\begin{figure}[H]
	\centering
	\includegraphics[width=0.49\textwidth]{figures/distanceErrorLiftLower.png}
	\includegraphics[width=0.49\textwidth]{figures/angleErrorLiftLower.png}
	\caption[Average error per body part for subtasks \textit{lift}/\textit{lower}.]{Average error per body part for subtasks \textit{lift}/\textit{lower}. Left: distance error, right: angle error. Suffix D: distance, suffix A: angle. LH - left hand, RH - right hand, LF - left foot, RF - right foot.}
	\label{fig:errorLiftLower}
\end{figure}
Figure~\ref{fig:errorLiftLower} shows that in EGO, the hip, hand and head accuracy is higher than in an EXO. The presence of exo-centric GV seems to have a positive influence on the feet's accuracy. The box's accuracy in EXO is lower than in EGO and EXO. In the actual study, particular attention should be paid to the box during \textit{lift} and \textit{lower} to identify the cause of why EXO performed badly. In orientation, the box's error is low for all VPs. This is expected since the subtask does not include a change in orientation.
The subjective accuracy is by far lowest in EGO, followed by EGO and EXO.

\subsubsection{\textit{pick}/\textit{place}}
\begin{figure}[H]
	\centering
	\includegraphics[width=0.49\textwidth]{figures/distanceErrorPickPlace.png}
	\includegraphics[width=0.49\textwidth]{figures/angleErrorPickPlace.png}
	\caption[Average error per body part for subtask \textit{turn}/\textit{fold}.]{Average error per body part for subtask \textit{turn}/\textit{fold}. Left: distance error, right: angle error. Suffix D: distance, suffix A: angle. LH - left hand, RH - right hand, LF - left foot, RF - right foot.}
	\label{fig:errorPickPlace}
\end{figure}
\textit{Pick} and \textit{place} are \textit{lift} and \textit{lower} movements with a significant difference in magnitude. The accuracy of pick and place benefits from the presence of an ego-centric GV. The distance and angle error of the box is lowest in EGO.

\subsubsection{\textit{push}/\textit{pull}}
\begin{figure}[H]
	\centering
	\includegraphics[width=0.49\textwidth]{figures/distanceErrorPushPull.png}
	\includegraphics[width=0.49\textwidth]{figures/angleErrorPushPull.png}
	\caption[Average error per body part for subtasks \textit{push}/\textit{pull}.]{Average error per body part for subtasks \textit{push}/\textit{pull}. Left: distance error, right: angle error. Suffix D: distance, suffix A: angle. LH - left hand, RH - right hand, LF - left foot, RF - right foot.}
	\label{fig:errorPushPull}
\end{figure}
During \textit{push} and \textit{pull}, increased force is applied to the box. The physiologist suggested that one foot should be shifted to the back for \textit{push} and \textit{pull} because of the increased force application. The high difference in error between the left foot and right foot is based on different foot placement. Unfortunately, the participants realised the different foot placement not often. One participant stated that he did not realise to shift one foot back in EGO in the interview. However, in EXO, he saw it and applied it then also for \combi. This statement harmonises with the quantitative data, which shows the lowest accuracy in EGO. The left hand seems to have a high error in EXO, the right hand in EGO. The video revealed that the participants alternated the hand placement during \textit{push} and \textit{pull}. Based on the video observation, the high error could even out with more participants. The higher error of the head angle in EXO compared to \combi\ indicates that the participants shared the focus in \combi\ with the ego-centric GV and the exo-centric GV. The participants rated their movements more exact in \combi\ than in EGO and lowest in EXO.

\subsubsection{\textit{turn}/\textit{fold}}
\begin{figure}[H]
	\centering
	\includegraphics[width=0.49\textwidth]{figures/distanceErrorTurnFold.png}
	\includegraphics[width=0.49\textwidth]{figures/angleErrorTurnFold.png}
	\caption[Average error per body part for subtask \textit{turn}/\textit{fold}.]{Average error per body part for subtask \textit{turn}/\textit{fold}. Left: distance error, right: angle error. Suffix D: distance, suffix A: angle. LH - left hand, RH - right hand, LF - left foot, RF - right foot.}
	\label{fig:errorTurnFold}
\end{figure}
Most of the movements during \textit{turn} and \textit{fold} happened on the table. Thus the main focus is on the hands. The high error in EXO is noticeable. The consultation of the video recordings revealed that in EXO, the participants could not see the direction of \textit{turn} directly and changed the right hand after the movement began. Furthermore, after starting to \textit{turn} or \textit{fold} the box, the participants changed the hand's position, presumably to ease the movement. The perceived accuracy is highest in EGO, followed by \combi\ and EXO. 

\subsubsection{\textit{carry} and \textit{walk}}
\begin{figure}[htb]
	\centering
	\includegraphics[width=0.49\textwidth]{figures/distanceErrorCarry.png}
	\includegraphics[width=0.49\textwidth]{figures/distanceErrorWalk.png}
	\caption[Average error of subtasks \textit{carry} and \textit{walk}]{Average error of subtask \textit{carry} (left) and \textit{walk} (right). Suffix D: distance. LH - left hand, RH - right hand, LF - left foot, RF - right foot.}
	\label{fig:walkError}
\end{figure}
Teaching locomotion in the ego-centric VP with the help of the \textit{speed mechanic} is a novelty. The data revealed a nearly equal error for ego-centric guided walking and exo-centric guided walking. The position of the hands are not essential for walking and are not depicted. Adding a physical load to walking (\textit{carry}) has a strong influence on accuracy. The learner seems to focus on the box and tries to match the GV's box with the own box. This increases the accuracy of their own position for all body parts.

\subsection{Visual Focus}
In EGO, the learner is provided one ego-centric GV and will focus on it. If exo-centric GVs are added to the scene, the learner can focus on multiple GVs. Furthermore, it is interesting which percentage of time the learner focuses on the own/GVs box and own/GVs body.\\
\begin{figure}[htb]
	\centering
	\includegraphics[width=0.5\textwidth]{figures/lookingAtExternalRepresentations.png}
	\caption[\textit{Looking at} for exo-centric guidance visualisations]{\textit{Looking at} for exo-centric GVs. Percentage of time focussed on the box or the avatar of the GV.}
	\label{fig:lookingAtExternal}
\end{figure}
A pilot study helps to evaluate the experiment design and data acquisition. Conducting a pilot study before the actual study is vital. The proof is depicted in figure~\ref{fig:lookingAtExternal}. In section~\ref{sec:rayTrace} it is described how the looking at data acquisition method was developed and tested. The formative test was conducted with one person\footnote{More participants were not possible because of the COVID-19 pandemic.}, which was too little. The study data revealed that the data acquisition for the measure \textit{looking at} is not working correctly. Over 50\% of the time, the ray traces hit the environment. For the actual study, the artefacts and avatars' colliders should be adjusted, or if available, an eye-tracker should be used. Nevertheless, assuming the rays which hit something other than the environment are evenly distributed, some deductions can be made from the acquired data.\\
Figure~\ref{fig:lookingAtExternal} shows the percentage of time the learner focussed on the box, a body (learner's avatar and GV's avatar) and the environment. The learner focussed roughly twice as much on a body than on the box.\\
Figure~\ref{fig:posHeatMap} shows the positions of the GVs whereby a GV is the union of body, table and box. The positions are overlayed with a heat map. The orange circles stand for the percentage of time the learner focuses on that position. The orange circles stand for the exo-centric VP, the gray circle stand for the ego- \& exo-centric VP. The ego-centric VP is not depicted, because there are no exo-centric GVs to share the focus of the learner. The heat map provides two insights. First, the presence of an ego-centric GV influences the visual focus of the learner. In EXO, where no ego-centric GV is present, the learner focused 11\% of the time on the artefacts (table, box) of one's own position. In \combi, the learner focussed 45\% of the time on the ego-centric GV (avatar, box, table). If an ego-centric GV is present, it is more frequently focussed than exo-centric GVs. By implication, the learner is consulting the exo-centric GVs, if an ego-centric GV is present.
The second insight regards position four. In EXO and \combi, the learner did not focus on the GV at position four. Position four is superfluous for all three tasks.
\begin{figure}[htb]
	\centering
	\includegraphics[width=\textwidth]{figures/positionHeatMap.png}
	\caption[\textit{Looking at} heat map.]{Heat map of the learner's visual focus in EXO and EGO \& EXO. The heatmap shows the tables of the GVs, compare figure~\ref{fig:learner_positions}. The circles' size corresponds to the amount of time the learner focussed the representation. Ego-centric VP is not depicted because in the ego-centric VP no exo-centric GV exists.}
	\label{fig:posHeatMap}
\end{figure}

\subsection{Risk Metrics}
The risk metrics are not analysed. The reason are the missing windows the RMs are based on. Recap: for specific subtasks the specific RM should be between a minimum and a maximum. The time inside and outside the window between the minimum and maximum yield a score. To determine a window fo the RM a professional should be consulted. Because of the COVID-19 pandemic, a determination by a professional was not possible.\\
However, the accuracy data for \textit{lift} and \textit{lower} could lead to a guess about the performance of \textit{squat}. The feet accuracy is higher in EXO than in EGO. \textit{Squat} refers to the same body part. Thereby, the a assumption could be that \textit{squat} could score better in perspectives with an exo-centric GV. \textit{Good base} and \textit{upright stance} are referring to body parts that are not directly visible in EGO, though the assumption could be extended to \textit{good base} and \textit{upright stance}, too.

\subsection{Subjective Preferences}
The personal preferences of the participants tend towards perspectives with exo-centric GVs. If the participants had chosen a VP for the task, two participants would decide for \combi, and one would use EXO. Additionally, two participants stated that they could follow the GV best in \combi, one could follow the GV best in EGO. All but one participant ranked the ability to follow the GV worst in EGO. This competes with the accuracy, which is the lowest in EGO. Surprisingly, all participants stated that the most accessible VP was EGO. In EGO, the GV stands inside one's own body, which is not possible in real-world scenarios, and thus unusual. A limitation for the ease of understanding is the high proficiency and knowledge about VR of all participants.

\chapter{Conclusion}
\label{chapter:conclusion}
(3 pages)
\section{System and Study}
Zusammenfassung der Evaluation des Systems über die eignung zur durchführung einer Studie die daten generiert um die Forschungsfrage zu beantworten.\\
Zusammenfassung was gut und schlecht ist bei der Studienausführung.\\
reflexion und contribution, inkl. zu erwartende empirische contribution\\
\section{Outlook}
\label{section:study_improvements}
Was kann noch evaluiert werden mit diesem system?\\
anderer task ohne physical load, sitzend zur bedienung von maschinen ,realismusgrad der avatare, anzahl avatare, position von avataren, geschwindigkeit der anleitungsanimation...\\
wer hat welchen nutzen von der beantwortung der forschungsfrage: designer von motorlearning vr systemen.\\
bezug zu erweiterungen der implementierung\\
possible improvement: DTW https://towardsdatascience.com/dynamic-time-warping-3933f25fcdd\\

limitations:
- nur für elementar tasks mit der bestimmten magnutude. bewegungen oberhalb oder unterhalb auch nicht passend. 


% =========== Bibliography ===========
\chapter*{References} % Set custom bibliography heading
\renewcommand{\thepage}{\roman{page}} % Use roman page numbers again
\setcounter{page}{\theromanPages} % Set the page counter
\defbibheading{bibempty}{} % Remove standard bibliography heading
\printbibliography[heading=bibempty]
\addcontentsline{toc}{chapter}{References} % Add bibliography to table of contents
\appendix
\chapter{Glossarium}
\begin{itemize}
	\item[\textbf{VP}] --- visual perspective
	\item[\textbf{GV}] --- guidance visualisation 
	\item[\textbf{MR}] --- Mixed Reality
	\item[\textbf{VR}] --- Virtual Reality 
	\item[\textbf{L}] --- learner
	\item[\textbf{PT}] --- participant
	\item[\textbf{T}] --- task
	\item[\textbf{ED}] --- Euclidean distance
	\item[\textbf{ETD}] --- ego-centric tethering distance
	\item[\textbf{TCT}] --- task completion time
	\item[\textbf{TND}] --- task norm duration
	\item[\textbf{OTND}] --- over task norm duration
\end{itemize}
\chapter{Attachments}

\section{Learner Rendering Pipeline}
\label{a:studentRenderingPipeline}
\begin{figure}[H]
	\includegraphics[angle=90,height=18cm]{figures/studentRenderingPipeline.png}
\end{figure}

\newpage
\section{Study Documents}
%\includepdf[pages=-]{attachments/after_session_questionaire.pdf}
\thispagestyle{empty}
\ 
\newpage
\thispagestyle{empty}
\begin{figure}[b]
	\centering
	\includegraphics[width=0.8\textwidth]{figures/digitalAttachments.png}
\end{figure}
\end{document}
