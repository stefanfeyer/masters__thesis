\chapter{Motor Learning in Virtual Reality}
\label{chapter:theoretical_background}
This chapter provides the theoretical background of Virtual Reality, motor learning, visual perspectives, handling physical load and injury risk metrics in a condensed form. For a more detailed description of these topics, please refer to the preceding seminar thesis~\cite{seminarThesis} which is also digitally attached to this master's thesis. These topics are the essential aspects that serve as the foundation for this master's thesis. Subsequently, an analysis of related work is given. Finally, the research contribution statement is provided.

\section{Virtual Reality}
\label{section:mixed_reality}
\begin{figure}[htb]
	\centering
	\includegraphics[width=0.8\textwidth]{figures/milgram_continuum.png}
	\caption[Mixed Reality continuum by Milgram et al.]{Mixed Reality continuum by Milgram and Kishinho~\cite{mrcontinuum}}
	\label{fig:mrCont}
\end{figure}
Milgram and Kishinho~\cite{mrcontinuum} describe Mixed Reality (MR) for visual displays on a continuum, compare figure~\ref{fig:mrCont}. Virtual Reality (VR) is purely digital, and thereby the environment is blocked entirely. In Augmented Reality, the environment is visible and augmented with digital elements. During motor learning, the visual perception of one's own body is desirable because it is the most exact representation of one's own body. Thereby, the approach of augmenting the real-world body with a virtual guidance visualisation (GV) is promising. However, today's AR-technology provides a small field of view. A solution to this could be the video see-through technology, but it is limited by latency and distortion~\cite{max}.\\
The body's perception can also be achieved by tracking the learner's body and render it over the learner's physical body. Thus, the visual perception of the learner's body can be established in VR. Consequentially, this work will focus on motor learning in Virtual Reality.

\section{Motor Learning}
\label{section:motor_learning}
\begin{figure}[htb]
	\centering
	\includegraphics[width=0.49\textwidth]{figures/movement_classification.png}
	\includegraphics[width=0.49\textwidth]{figures/movement_classification2.png}
	\caption[Movement classifications by Scmhidt et al.]{Movement classification by \textit{particualar movement} (left) and \textit{perceptual attributes} by Schmidt et al.~\cite{mlbook}}
	\label{fig:movement_classification}
\end{figure}
Motor learning is achieved through instruction, attempt, imitation or a combination of them~\cite{mlbook}. The process of motor learning can be divided into three parts: cognitive stage, associative stage and autonomous stage (ibid.). In the cognitive stage, training methods are most efficient, and the performance gain is the highest among the stages (ibid.). Tasks that belong to this stage are thereby best suited for an experiment.\\
Movements can be classified by two means: by \textit{particular movements} and based on \textit{perceptual attributes} (ibid.). Based on the \textit{particular movements}, the classification is described by a continuum, compare figure~\ref{fig:movement_classification} left. On the extremes of the continuum are \textit{discrete movements} and \textit{continuous movements}. Between these extremes, \textit{serial movements} are located. \textit{Discrete movements} are too short for an evaluation. \textit{Continuous movements} do not have a recognisable beginning and end, and thereby they are not suitable for the experiment in question either. \textit{Serial movements} are chained \textit{discrete movements} with a recognisable beginning and end. This allows a task decomposition and an evaluation of particular subtasks. Furthermore, \textit{serial movements} are generally more complex than \textit{discrete} or \textit{continuous movements} and therefore more suited for movement training systems. \textit{Serial movements} are widely used for research in motor learning, for example~\cite{lightguide,mythaichicoaches,elearningma}. Therefore, the experiment task of this master's thesis is based on \textit{discrete movements}.\\
The classification based on the \textit{perceptual attributes} is also represented by a continuum and includes the environment in which the movement is performed, compare figure~\ref{fig:movement_classification} right. At the extremes of the continuum, \textit{open skills} and \textit{closed skills} are located. For \textit{closed skills}, the environment is predictable, while for\textit{ open skills}, the environment is not predictable. The experiment aims to analyse the learner's performance of following a movement and not how they can adapt to environmental changes. Thereby, this experiments task for this master's thesis must be located on the left-hand side of the continuum: \textit{closed skills}.


\subsection{Measurements for Motor Learning}
\label{section:measures_for_ml}
The movements of a teacher and the movement of a learner differ. To assess the difference between the two movements, two main classes of measures can be applied~\cite{mlbook}: \textit{measures of error for a single object} and \textit{measures of time and speed}.
\textit{Measure of error for a single object} represent the degree to which the target movement is amiss. Schmidt et al.~\cite{mlbook} provide five \textit{error measures} to calculate this error. Among them, \textit{constant error} is the most common measure in related work to determine the difference between the movement of the learner and the movement of the teacher, for example~\cite{perspectivematters,thaichichua,YouMove,onebody,vrdancetrainer,lightguide,physioathome}. \textit{Constant Error} is defined as the average error between the learner's movement and the teacher's movement and is described as
\begin{equation}
	\label{eq:constanterror}
	CE=\frac{\sum_i(x_i-T)}{n}
\end{equation}
with $x_i$: actual value, $T$: target value, $n$: number of values~\cite{mlbook}.\\

The basic idea of \textit{measures of time and speed} is that a performer who can accomplish more in a given amount of time or who can accomplish a given amount of behaviours in less time is more skilful. In related work, this measure is mostly assessed by the \textit{task completion time} (TCT), for example~\cite{perspectivematters,onebody,lightguide}.


\section{Visual Perspectives}
\label{section:visual_perspectives}
\begin{figure}[H]
	\centering
	\includegraphics[width=\textwidth]{figures/ego_exo_continuum.PNG}
	\caption[Centricity continuum by Wang et al.]{Centricity continuum by Wang and Milgram~\cite{centricitycontinuum}}
	\label{fig:ego-exo-continuum}
\end{figure}
Wang and Milgram~\cite{centricitycontinuum} describe visual perspectives (VP) by the \textit{centricity continuum}, compare figure~\ref{fig:ego-exo-continuum}. On the left extreme on the continuum, the ego-centric VP is located, in literature also called first-person perspective (1PP). On the right extreme is the exo-centric VP, in literature also called third-person perspective (3PP). The middle part represents tethered VP. By moving from the left to the right, the so-called \textit{tethering distance} increases. The \textit{tethering distance} describes the distance of the anchor point of the eyes to the object in question. In this master's thesis, the object in question is the human-shaped guidance visualisation (avatar). VPs can be clustered into three classes: ego-centric VPs (g-class), exo-centric VPs (x-class) and VPs that contain both ego-centric and exo-centric VPs (gx-class). Without the usage of additional perspective influencing artefacts like mirrors, cameras or screens, there are five possible VPs:
\begin{itemize}
	\item \textbf{Ego-centric}: the teacher's avatar is located inside the learner's avatar. The learner sees the GV inside one's own body, compare figure~\ref{fig:perspectives} top left and figure~\ref{fig:ego}.
	\item \textbf{Purely exo-centric}: the teacher's avatar is located outside the learner's avatar. The learner sees the GV, e.g. in front of him/her, compare figure~\ref{fig:perspectives} top middle.
	\item \textbf{Augmented exo-centric}: the teacher's avatar is located outside the learner's avatar. Additionally, a virtual copy of the learner's avatar is located inside the teacher's avatar, compare figure~\ref{fig:perspectives} bottom middle and figure~\ref{fig:exo}.
	\item \textbf{Purely ego- \& exo-centric}: the combination of purely ego-centric VP and purely exo-centric VP. The learner sees the GV as well as inside and outside of one's own body, compare figure~\ref{fig:perspectives} top right.
	\item \textbf{Ego- \& augmented exo-centric}: the combination of the ego-centric VP and the augmented exo-centric VP. The learner sees the GV inside one's own body, as well as outside. Additionally, a virtual copy of the learner is located inside the exo-centric GV, compare figure~\ref{fig:perspectives} bottom right and figure~\ref{fig:egoexo}.	
\end{itemize}

\begin{figure}[htb]
	\centering
	\includegraphics[width=\textwidth]{figures/perspectives_new.png}
	\caption[Visual perspectives on vitual guidance visualisations]{Visual perspectives on  virtual guidance visualisations, clustered by their corresponding class. Icon: created by Ghan Khoon Lay from Noun Project, \href{https://thenounproject.com/}{https://thenounproject.com/}, accessed: 19.06.2020}
	\label{fig:perspectives}
\end{figure}

\section{Handling Physical Load}
\label{section:handlingphysicalload}
Handling physical load is part of the more general topic Manual Material Handling (MMH). MMH is composed of five elemental tasks: lift, lower, push, pull and hold~\cite{mmh}. Additionally, there are non-elemental tasks like turning and sliding (ibid.). The experiment proposed in this master's thesis will use an experiment task that includes the handling of physical load. Evidently, the task should consist of these elemental tasks. A task that consists of elemental tasks can be generalised to other tasks to a certain extend. To gain a more robust data basis, multiple elemental tasks can be chained together and repeated to form a so-called Unit-Combined-MMH task (ibid.). In chapter~\ref{sec:taskDesign} is described how the elemental tasks become subtasks of the experiment task.

\section{Injury Risk Metrics}
\label{section:rm}
Muckell et al.~\cite{muckell} identified four main features which are common in the bio-mechanical evaluation of different lifting and carrying techniques. Based on those four features, they defined four injury risk metrics to define low risk and high risk movements. The four risk metrics (RM) are described in the following. \textit{Support base} describes the distance between the feet. With a proper support base, an individual is more stable while performing a movement like lifting or lowering. \textit{Squat} describes the distance between pelvis and floor. "A proper \textit{squat} reduces injury risk since the lifting force is applied using legs and not the back" (ibid.). \textit{Upright stance} is defined by the angle between the upright vector and the bend of the back of an individual. \textit{Spine twist} is the angle between the lines between the left and right shoulder and the left and right hip.
\newpage

\section{Related Work: Motor Learning in Virtual Reality}
\label{section:related_work}
\begin{table}[H]
	\begin{tabularx}\textwidth{@{}XXXX@{}}
		\toprule
		g-class & x-class & g-class and x-class & gx-class \\ \midrule
		AR-Arm \cite{ararm} & MotionMA \cite{motionma} & OneBody \cite{onebody} & Tai Chi Trainer \cite{thaichichua} \\
		Just Follow Me \cite{justfollowme} & YouMove \cite{YouMove} & LightGuide \cite{lightguide} & SleeveAR \cite{sleevear} \\
		Ghostman \cite{ghostman} & VR Dance Trainer \cite{vrdancetrainer} & MR Dance Trainer \cite{mrdancetrainer} & \\
		Stylo Handifact \cite{stylo} & Physio@Home \cite{physioathome} & Throw Simulator \cite{freethrowsimulator} & \\
		GhostHands \cite{ghosthands} & OutsideMe \cite{outsideme} & Training Phys. Skill \cite{trainingphysicalskills} & \\
		& E-Learning MA \cite{elearningma} & VP Matters \cite{perspectivematters} &  \\
		& My Tai Chi \cite{mythaichicoaches}  & &\\
		& Perform. Training \cite{performancetraining} & &\\
		& RT Gestrue Recognition \cite{rtgesturerecognistion} & &\\
		& KinoHaptics \cite{kinohaptics} & &\\
		& TIKL \cite{tikl} & &\\ \bottomrule
	\end{tabularx}
	\caption[Related work clustered by the visual perspectives.]{Overview of related work clustered by visual perspectives.}
	\label{tab:rw_overview}
\end{table}

Training movements in Virtual Reality were investigated previously in several works. An overview differentiated by the VP the reasearchers used to train movements is provided in table~\ref{tab:rw_overview}. AR-Arm~\cite{ararm}, Just Follow Me~\cite{justfollowme}, Ghostman~\cite{ghostman}, Stylo and Handifact~\cite{stylo} and GhostHands~\cite{ghosthands} used the ego-centric VP. MotionMA~\cite{motionma}, YouMove~\cite{YouMove}, VR Dance Trainer~\cite{vrdancetrainer}, Physio@Home~\cite{physioathome}, OutsideMe~\cite{outsideme}, E-learning Martial Arts~\cite{elearningma}, My Tai Chi coaches~\cite{mythaichicoaches}, Performance Training~\cite{performancetraining}, Real Time Gesture Recognition~\cite{rtgesturerecognistion}, KinoHaptics~\cite{kinohaptics} and TIKL~\cite{tikl} used a VP from the g-class. There are also works that used to train movements in g-class and x-class, like OneBody~\cite{onebody}, LightGuide~\cite{lightguide}, Mixed Reality Dance Trainer~\cite{mrdancetrainer}, Free Throw Simulator~\cite{freethrowsimulator} and Training Physical Skills~\cite{trainingphysicalskills}. It is little research done for movement training in the gx-class, for example Tai Chi Trainer~\cite{thaichichua} and SleeveAR~\cite{sleevear}. The experiment proposed in this master's thesis will compare the ego-centric VP, the augmented exo-centric VP and the ego \& augmented exo-centric VP. In the ego \& augmented exo-centric VP a virtual copy of the learner is located inside exo-centric GV, which was not part of previous works.\\

The task which the refferred works use araise from the fields of dancing~\cite{YouMove,vrdancetrainer,outsideme,performancetraining,mrdancetrainer}, sports~\cite{freethrowsimulator,trainingphysicalskills}, rehabilitation~\cite{motionma,physioathome,kinohaptics,sleevear,veimprovesml}, arts~\cite{ararm,justfollowme,stylo,elearningma,mythaichicoaches,rtgesturerecognistion,onebody,thaichichua} and others~\cite{tikl,lightguide}. None of them include the ergonomic handling of a physical load, but sometimes include physical artefacts like a ball (e.g. Free Throw Simulator~\cite{freethrowsimulator}) or chop sticks (Ghostman~\cite{ghostman}). Also none include locomotion movements. The body parts that are included in the above-mentioned tasks vary, too. For example, \cite{YouMove, thaichichua,onebody,vrdancetrainer} full-body movements are taken into consideration, while \cite{ararm,sleevear,ghosthands} uses arm movements. The experiment proposed in this master's thesis will utilise a task for full-body movements which, includes the handling of physical load and locomotion movements.\\

The guidance visualisations which are used to train movements are stick figures~\cite{onebody,YouMove,vrdancetrainer,performancetraining}, wireframes\cite{thaichichua,mrdancetrainer}, human-shaped avatars\cite{thaichichua,vrdancetrainer,trainingphysicalskills,mythaichicoaches} and indicators~\cite{ararm,physioathome,sleevear,ghostman}. The experiment proposed in this master's thesis will use human-shaped avatars.\\

To determine to what extend the movements of the learner matches the GV, different measures are applied. Most common are performance measures based on the accuracy of the performed movements (~\cite{YouMove,thaichichua,vrdancetrainer,onebody,lightguide,physioathome}) and the time related measurements like the \textit{task completion time} (~\cite{lightguide,onebody}). The experiment proposed in this master's thesis will utilise accuracy measurements, timely measurements, a measurement to assess the learners visual focus and assess quantitative data.\\

How the perspective influences the learner's performance is sparsely investigated. Recently, in December 2020, Yu et al.~\cite{perspectivematters} conducted three independent studies to close this gap. In the first study, Yu et al. compared the ego-centric VP and a 2D-mirror for single arm movements. In the second study, they compared the ego-centric and exo-centric VP for Yoga. In the third study, they compared the ego-centric VP with a 3D-mirror for arm movements. Yu et al. conclude their findings in a design guideline for systems training motor learning in Virtual Reality: use the ego-centric VP if the type of motion allows, consider alternatives for other types of motions (ibid.). In all three studies, the ego-centric VP outperformed the other perspectives if the movement was completely visible from the ego-centric VP. \cite{onebody,lightguide} compared their ego-centric VP with their exo-centric VP. For the task they used, the ego-centric VP outperformed the exo-centric VP. \cite{YouMove,vrdancetrainer} compared the movement learning in VR with traditional video-based movement learning. In both cases, VR movement learning outperformed video movement learning.\\

\subsection{Research Contribution Statement}
\label{sec:delimination_contribution}
As depicted in the last section, motor learning in the gx-class VPs is rarely investigated, especially for full-body movements. Furthermore, motor learning that includes the handling of a physical load in VR in different VPs was not part of investigations. Additionally, previous works used stationary tasks in the ego-centric VP. Moreover, how the visual perspectives influence the learner's visual focus is unexplored.\\
The proposed experiment in this master's thesis will provide an empirical contribution by increasing the empirical evidence of how the VP on GVs influences motor learning by guiding full-body movements in three VPs: ego-centric, exo-centric and the pure combination of them. Furthermore, empirical evidence can be generated with the proposed experiment for motor learning, including the handling of a physical load and the learner's visual focus. An artefact contribution is provided by presenting a method for guiding locomotion movements in the ego-centric VP. \\

The generated data of the proposed experiment will help designers of VR motor learning systems to choose a reasonable perspective for their project.


%The preceding seminar thesis provided an overview over 23 (compare table~\ref{tab:rw_overview}) of these works and evaluated six of them in detail: Tai Chi Trainer by Chua et al.~\cite{thaichichua}, YouMove by Anderson et al.~\cite{YouMove}, VR Dance Trainer by Chan et al.~\cite{vrdancetrainer}, OneBody by Hoang et al.~\cite{onebody}, LightGuide by Sodhi et al.~\cite{lightguide} and Pyhsio@Home by Tang et al.~\cite{physioathome}. Special attention was paid to the VP, task, GV and their independent and dependent variables they used in their investigations. Finally, the results of these works were concluded. An overview is depicted in table~\ref{tab:rw_overview_detail}.

%These works inform this work in various aspects. Chua et al. used the ego \& augmented exo-centric VP, Hoang et al. and Sodhi et al. the ego-centric visual perspective. These visual perspectives proved to be suited for the evaluation of Motor Learning in VR and are adopted for the proposed study design, compare section~\ref{section:visual_perspectives}. Furthermore, Chan et al. and Chua et al. used high realistic avatars as guidance visualisation, which are used in the proposed study design, compare seminar thesis chapter 3.3. Additionally, recent research indicates that high realism avatars outperform abstract avatars~\cite{max,perspectivematters}. All authors used a performance measure to evaluate the performed movements of the participants of their studies. Primarily the distance-based measures informed the measures used in the proposed study design\\%compare figure ??? yxc
%The above mentioned works do not use the relatively new technology of Vive Trackers in combination with Inverse Kinematics (IK, see project report chapter 2.1 and 2.2). Sra et al.~\cite{samesetup} used this technology in 2018 for their system Your Place and Mine to render human-shaped avatars.\\
%The results of related work yielded no clear conclusion about the influence of the perspectives on motor learning. Chua et al. found no difference in the performance between the visual perspectives, Anderson et al. and Chan et al. found out that their exo-centric visual perspectives in Virtual Reality outperform traditional video guidance. Hoang et al. and Sodhi et al. conclude that the ego-centric perspective outperforms the exo-centric visual perspective. Nevertheless, an investigation of how the visual perspective influences motor learning was not investigated.  This work, in contrast, focuses on full-body movements that include the handling of physical load. Furthermore, this work provides a third visual perspective, where the ego-centric and exo-centric visual perspective is combined.

%VR already proved to be a suitable environment for Motor Learning for tasks like dancing~\cite{YouMove,vrdancetrainer,outsideme,performancetraining,mrdancetrainer}, sports~\cite{freethrowsimulator,trainingphysicalskills}, Rehabilitation~\cite{motionma,physioathome,kinohaptics,sleevear,veimprovesml}, arts~\cite{ararm,justfollowme,stylo,elearningma,mythaichicoaches,rtgesturerecognistion,onebody,thaichichua} and others~\cite{tikl,lightguide}.




%\todo{new papers occured, read them, then write this statement. notes:}
%what is done:  v
%comparing ego-centric with exo-centric video.
%comparing ego-centric with exo-centric and the combination, but yielded to no result, because old paper and old pc
%comparing with mirrors,
%comparing isolated body parts
%everyone made his live easy by just looking at stationary movements, mostly containing only some body parts.
%new: nobody did fullbody movements with locomotion. ego-centric locomotion motion guidance is completely new. 
%related work only investigated on stationary movements. but motor learning is not stationary. body parts is also not stationary.
%real-world relation poor because of arts dance or abstract. my work is the first one haveing really a task that is reasonable!\\
%Previous work investigated the differences between the perspectives, but:
%To my knowledge, there is no investigation on full body movements that include locomotion. Furthermore, there are on investigations that include the handling of physical load.
%Perivous works compared ego-centric Motor Learning with video learning\cite{YouMove,vrdancetrainer}, augemnted mirrors\cite{perspectivematters,onebody}.
%The conduction of the proposed study will produce data that serves as a reasonable basis for designers of VR Motor Learning systems choosing a suitable perspectives. This is achieved by an Empirical Research Contribution. The empirical data is gathered by a comparative study between the ego-centric visual perspective, the exo-centric visual perspective and the combination. As novelty, the task includes handling of physical load which consists of the elemental tasks of manual material handling. This allows an evaluation of the elemental tasks per visual perspective and can give insights which perspective is suited for specific tasks.\\
%Additionaly, an artifact contribution is provided by the ego-centric guidance of locomotion movements.

%direct comparison not even in tai chi chua
