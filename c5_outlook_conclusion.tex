\chapter{Conclusion \& Outlook}
\label{chapter:conclusion}

This master's thesis proposes an experiment that can generate data to answer the research question RQ1: How does the visual perspective on a virtual guidance visualisation influence motor learning in Virtual Reality environments?\\
This chapter concludes the achieved work, improvements in experiment design and \exgo, and finally gives an outlook on future work. 

\section{Conclusion}
In chapter~\ref{chapter:studysetting_conduction} an experiment was described to answer the main research question. The decision to compare the ego-centric VP, augmented exo-centric VP and the ego- \& augmented exocentric VP was discussed. To realise the VP, two mechanics were utilised: the \textit{speed mechanic}, which allows ego-centric locomotion guidance and \textit{multiple representations} which allows the learner always seeing an exo-centric GV. A task that includes the handling of physical load was iteratively designed, and an experiment structure was provided. To assess the learner's performance, the experiment requires measures: accuracy measures based on distances and angles, risk metrics, focus measurements and qualitative measurements for the learner's subjective opinion. The next chapter~\ref{chapter:system} describes the implementation of a system to conduct the experiment. The learner and teacher received a virtual representation with the help of trackers and inverse kinematics. Artefacts like the physical load were constructed and represented digitally, too. Measures required by the experiment design were implemented, and finally, the procedure of the experiment could be provided.\\

The experiment was evaluated with a pilot study, and vast parts of the design worked as intended. The mirror as calibration help served its purpose. Table, scale and box proved to be appropriate. The VPs were understood well by the participants. The experiment's task is safe for the participants, and the task design served its purpose for the experiment. The itinerary proved to be suited, but the total time for one participant was shortened by 10 minutes. However, adjustments for the actual experiment were necessary. The fixation of the hip tracker to the participant's body needed to be improved. An additional checklist for task and condition during the experiment were introduced. The identification of the video recordings was reconsidered. The HMD should be wireless for the actual experiment. To improve the recognition of ownership of a stationary box, GVs were suggested to be rendered with light transparency. Ego-centric locomotion guidance with the help of the speed mechanic works. The pilot experiment could successfully eradicate the minor errors in the log file. The method to assess the learner focus need was redesigned. Finally, three new questions were placed on the \textit{after session questionaire}. With these adjustments, the experiment is capable of answering the research questions.\\

The pilot experiment had three participants, which does not allow a full counterbalancing. The evaluation of the data is informal. An answer to the research questions can not be given on that basis. Nevertheless, the data gives a first glimpse on what to expect from the empirical contribution of this master's thesis. The following statements for the research questions are regarding to clear differences in the data but should still be not taken as answers. The statements are more a list of conspicuous elements in the data. The evaluation of the actual experiment can give more detailed insights.

\begin{enumerate}[align=left, leftmargin=0pt, labelindent=\parindent,
	listparindent=\parindent, labelwidth=0pt, itemindent=!]
	\item[RQ1.1] How does the visual perspective on a virtual guidance visualisation influence movements' accuracy?
	\begin{itemize}
		\item[] \begin{itemize}
			\item[RQ1.1.1] How does the visual perspective on a virtual guidance visualisation influence movements' accuracy of the own body?\\
			\textbf{The presence of an ego-centric GV seems to influence the accuracy of body parts positively. The perception of correct feet placement is limited in the ego-centric VP.}
			\item[RQ1.1.2] How does the visual perspective on a virtual guidance visualisation influence the accuracy of handling physical load?\\
			\textbf{The presence of an ego-centric GV seems to influence the accuracy of the box positively.}
			\item[RQ1.1.3] How does the visual perspective on a virtual guidance visualisation influence subtasks' accuracy?
			\textbf{Exo-centric GVs influence the feet placement during lift and lower positively. The overall accuracy of each subtask is highest in the ego-centric VP.}
		\end{itemize}
	\end{itemize}
	
	\item[RQ1.2] Does the visual perspective on a virtual guidance visualisation influence the transfer of ergonomic principles?
	\textbf{RM could not be evaluated. My personal subjective opinion is that the transfer of ergonomic principles is better in VP with an exo-centric VP.}
	\item[RQ1.3] How does the visual perspective on a virtual guidance visualisation influence the learner's visual focus?\\
	\textbf{If an ego-centric GV is present, the learner focusses more on the ego-centric GV than an exo-centric GV.}
	\item[RQ1.4] What is the subjective personal preference of the learner for the visual perspectives?\\
	\textbf{The participants of the pilot experiment tend for the ego- \& exo-centric VP.}
\end{enumerate}

Based on RQ1.1-4, an assumption can be made about the main research question:\\
\textbf{RQ1:} How does the visual perspective on a virtual guidance visualisation influence motor learning in Virtual Reality environments?
The presence of an ego-centric GV increases the overall accuracy of movements for all subtasks and shifts the visual focus of the learner towards the ego-centric GV.\\

Yu et al.~\cite{perspectivematters} published in December 2020 the guideline: use the ego-centric visual perspective if the type of motion allows, consider alternatives for other types of motions (ibid.). This guideline could also hold for full-body movements that include the handling of physical and locomotion movements.

\section{Outlook}
\label{sec:outlook}
\exgo\ could benefit from being extended with the dynamic-time-warp\footnote{https://towardsdatascience.com/dynamic-time-warping-3933f25fcdd, accessed 28.3.2021} algorithm. Till now, the measures are implemented to assess the error between the learner and GV in this exact moment. The dynamic-time-warp algorithm searches in a timely window for the lowest error. This algorithm erases the reaction time. This algorithm is applied in some related work, e.g.~\cite{thaichichua}, too.\\

Besides that, \exgo\ is a system capable of conducting further experiments. Already, all five possible VP, including the two which are not utilised in the proposed experiment, are already implemented. Upcoming experiments could investigate the differences between the two perspectives in the x-class VP, as well as in the gx-class VP.\\
Furthermore, \exgo\ can be used without a physical load. A similar experiment to the proposed experiment but without the physical load could give insights into the outcome of both experiments correspond or differ. \exgo\ can also be easily extended with new physical artefacts. Investigating the influence of shape, size and weight of the physical load on the learner's performance could be interesting, too. Additionally, \exgo\ can be used for tasks where the learner is sitting. Scenarios where the learner operates a machine seated, are possible to be evaluated with \exgo.\\
The subtasks in the proposed experiment all having a specific magnitude. In upcoming experiments, the magnitude can be varied.\\
Dürr et al.~\cite{max} showed that for ego-centric guidance, a high realistic avatar outperforms stylised avatars. \exgo\ can be used to investigate if this can also be applied to full-body avatars. Furthermore, an in-detail evaluation of the number and positions of exo-centric GV could be interesting. The distance between the learner and exo-centric GVs plays a role in how the learner can visually percept the exo-centric GVs. The influence of the distance between exo-centric GV and learner on the learner's performance is a topic of interest.\\
The artefact contribution of guiding ego-centric locomotion movements is limited by the low amount of participants. A larger study would erase the limitation, and the definition of the $ETD_{min}$ and $ETD_{max}$ can be built on empirical data. Such a study is possible with \exgo\, too.\\

The generated data of the experiment proposed in this master's thesis will help designers of VR motor learning systems to choose a reasonable perspective for their project.
% possible improvement: DTW https://towardsdatascience.com/dynamic-time-warping-3933f25fcdd\\