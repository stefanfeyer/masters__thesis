\chapter{Introduction}
The acquisition of movements is a crucial part of human development. Learning movements empowers to be more efficient, faster and more exact. The capability of enhanced movements enables the learner to survive from the very beginning. The process of learning movements is called Motor Learning. Nowadays, Motor Learning is still crucial. Especially for tasks like sports, arts or the ergonomic handling of physical load.\\
Most movements we learn by voyeurism and mimicking: watching and trying it out by yourself. Mastering a movement is performed best with an experienced teacher. A teacher is hardly replaceable because of immediate visual, audible and haptic feedback on a performed movement. However, if a teacher is not available, for example, based on the location or economic reasons, other sources can be used to learn movements. For example, YouTube\footnote{https://www.youtube.com/, accessed 17.2.2021}, TikTok\footnote{https://www.tiktok.com/, accessed 17.2.2021}, and other video platforms have become a great source for learning videos with a wide range of purposes. The downside of videos is the two dimensional (2D) experience of a three dimensional (3D) movement. Mixed Reality (MR) can provide this experience in 3D. Furthermore, MR can provide feedback on the performed movement and has the ability for interactions with the virtual guidance visualisation. MR already proved to be a suitable environment for Motor Learning for tasks like dancing~\cite{YouMove,vrdancetrainer,outsideme,performancetraining,mrdancetrainer}, sports~\cite{freethrowsimulator,trainingphysicalskills}, Rehabilitation~\cite{motionma,physioathome,kinohaptics,sleevear,veimprovesml}, arts~\cite{ararm,justfollowme,stylo,elearningma,mythaichicoaches,rtgesturerecognistion,onebody,thaichichua} and others~\cite{tikl,lightguide}.\\
In the real world, where the student and teacher are real persons, the student sees the teacher, for example, in front of himself/herself. This perspective is called the exo-centric visual perspective. Nevertheless, if we move from the real world to the virtual world of MR, we are no longer restricted to the exo-centric visual perspective. The teacher can be rendered inside the student's body, allowing the student to see the teacher from an ego-centric perspective. The change from the exo-centric to the ego-centric visual perspective potentially influences Motor Learning, shown by previous research; for example, AR-Arm~\cite{ararm} lets the learner experience the movements from an ego-centric perspective. YouMove~\cite{YouMove} teaches dance from an exo-centric perspective. OneBody~\cite{onebody}, Light Guide~\cite{lightguide}, MR Dance Trainer~\cite{mrdancetrainer}, Free Throw Simulator~\cite{freethrowsimulator}, Training Physical skills~\cite{trainingphysicalskills}, Sleeve AR~\cite{sleevear} and Thai Chi Trainer~\cite{thaichichua} use both visual perspectives. However, only OneBody, LightGuide and TaiChi Trainer found a difference between the visual perspectives. Furthermore, none of these works investigated how the visual perspective influences the performance of the learner. Another topic where MR could be a valuable helper is the ergonomic conduction of movements while handling physical load~\cite{nursecare,kitt}. Handling physical load in the correct ergonomic conduct in working routines can prevent injuries in everyday life. However, a kinaesthetics teacher is not always accessible, for example, for economic reasons. The influence of the visual perspective on a virtual guidance visualisation teaching the handling of physical load in mixed reality is sparsely investigated  \todo{is there a source?}. Especially, locomotion movements like walking in the ego-centric perspective is left out. The lack of research in the influence of the visual perspective on a virtual guidance visualisation, especially for handling physical loads, shows the necessity of investigations on:
\begin{itemize}
	\item[RQ1:] How does the visual perspective on a virtual guidance visualisation influence Motor Learning in Virtual Reality environments?
\end{itemize}
To answer this main research question RQ1, several aspects have to be taken into account: accuracy of movements, transfer of information of how to move, the visual focus of the learner and last but not least, the personal preference of the learner. Therefore, to answer the main research question RQ1, it is necessary to answer the following sub-research questions:
\begin{enumerate}[align=left, leftmargin=0pt, labelindent=\parindent,
	listparindent=\parindent, labelwidth=0pt, itemindent=!]
	\item[RQ1.1] How does the visual perspective on a virtual guidance visualisation influence movements' accuracy?
	\begin{itemize}
		\item[] \begin{itemize}
			\item[RQ1.1.1] How does the visual perspective on a virtual guidance visualisation influence movements' accuracy of the own body?
			\item[RQ1.1.2] How does the visual perspective on a virtual guidance visualisation influence the accuracy of handling physical load?
			\item[RQ1.1.3] How does the visual perspective on a virtual guidance visualisation influence sub-tasks' accuracy?
		\end{itemize}
	\end{itemize}
	
	\item[RQ1.2] Does the visual perspective on a virtual guidance visualisation influence the transfer of ergonomic principles?
	\item[RQ1.3] How does the visual perspective on a virtual guidance visualisation influence the learner's visual focus?
	\item[RQ1.4] What is the subjective personal preference of the learner for the visual perspectives?
\end{enumerate}
A detailed discussion of the research questions can be found in \ref{section:study}.\\
\textbf{The answers to these research questions will enable designers of VR Motor Learning training systems to choose a suitable visual perspective on an empirical basis.}

\section{Outline}
This work proposes a study design to answer the research question. To design this study on a solid basis, the theoretical foundations a laid in chapter~\ref{chapter:theoretical_background} with a closer look on Motor Learning (section~\ref{section:motor_learning}), visual perspectives (section~\ref{section:visual_perspectives}) and Mixed Reality (section \ref{section:mixed_reality}). These sections result in the scope and parameters of the study design. Section~\ref{section:related_work} investigates previous works and illustrates the conceptual delimitation of this work from what has already been investigated. Chapter~\ref{chapter:theoretical_background} concludes with a research contribution statement, clarifying the Empirical Contribution and Artifact Contribution of this work.\\
For the proposed study, a system had to be designed to produce data to answer the research questions. This system is called \exgo\. The design and and implementation is described in section~\ref{section:system} followed by the design of the study itself in section~\ref{section:study}.\\
\exgo\ and the study design have been evaluated in a pilot study. The results of the evaluation are depicted in chapter~\ref{chapter:study_evaluation}. Furthermore, this chapter suggests improvements in the study design in section~\ref{section:study_improvements}. This work concludes in chapter \ref{chapter:conclusion} with an outlook on how \exgo\ can be enhanced and expanded as well as used for further investigations.


\begin{enumerate}[align=left, leftmargin=0pt, labelindent=\parindent,
	listparindent=\parindent, labelwidth=0pt, itemindent=!]
	\item[long long label]  asdad asfiojh aojgf oijgoias gjoasig
\end{enumerate}
%Motivation: Motorlearning wichtig zur aneignung von bewegungen. am besten mit echtem lehrer. wenn dieser nicht verfügbar, motor learning in VR möglich und sinnvoll, siehe xyz. allerdings ist der einfluss der perspektive auf die virtuelle guidance vis. noch wenig untersucht. deswegen diese arbeit hier.\\
%ferner, wenig motorlearning in zusammenhang mit physical load und wenig "laufen mit egozentrischer anleitung".\\
%Daraus folgt die forschungsfrage ... und ihre sub forschungsfragen ...\\
%Um daten zu generieren um diese forschungsfragen zu beantworten wurde Exgo entwickelt. Eine studie wurde designed um mit diesem system die notwendigen daten zu generieren.\\
%Diese arbeit beschreibt design und entwicklung von exgo, sowohl als auch die entwicklung dieser studie. EIn pilottest und dessen überarbeitung ist angeschlossen. auch weiter forschungsmöglichkeiten mit dem system sollen aufgezeigt werden.\\
