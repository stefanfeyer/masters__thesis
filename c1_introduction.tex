\chapter{Introduction}
(3 pages)\\
The aquisition of movements is a crucial part in human development. Learning movements empowers to be more efficient, faster and more exact. This enables the learner to survive from the very beginning. The process of learning movements is called Motor Learning. Nowadays, Motor Learning is still crucial. Especially for tasks like sports, arts or the ergonomic handling of physical load.\\
A lot of movements we learn by mimicing: watching and trying it out by yourself. But mastering a movement is performed best with a experienced teacher. A teacher is hardly  replaceable because of imediate visual hand haptical feedback on a performed movement. But if a teacher is not available, for example based on the location or economic reasons, other sources to learn movements can be used to learn movements. For example, YouTube\footnote{https://www.youtube.com/, accessed 17.2.2021}, TikTok\footnote{https://www.tiktok.com/, accessed 17.2.2021} and other video plattorms have become a great source for learning videos for a wide rage of purposes. The downside of videos is two dimensional (2D) experience of a three dimensional (3D) movement. Mixed Reality (MR) can provide this experience in 3D. Furthermore, MR can provide feedback on the performed movement and has the ability for interactions with the virtual guidance visualisation. MR already proved to be a suitable environment for motorlearning for tasks like dancing \todo{YouMove, VR Dance Trainer, OutSide Me, Performance Training, MR Dance Trainer}, sports \todo{Free Throw Simulator, Training Physical Skill}, Rehabilitation \todo{MotionMA, Physio @Home, KinoHaptics, SleeveAR}, arts \todo{AR-arm, just follow me, stylo and handifact, e-learning martial arts, my thai chi coaches, RT gestrue Recognition, onebody, thai chi trainer chua} and others \todo{TIKL, LightGuide}.\\
In the real world, where student and teacher are real persons, the student sees the teacher for example in front of himself/herself. This is called the exo-centric visual perspective. But if we move from the real world to the virtual world of MR, we are no longer restricted to the exo-centric visual perspective. The teacher can be rendered inside of the students body, allowing the student to see the teacher from an ego-centric perspective. The change from the exo-centric to the ego-centric visual perspective potentionally influences Motor Learning \todo{sources}.\\
AR-Arm \todo{source} lets the learner experience the ... from an ego-centric perspective. YouMove teaches dance from an exo-centric perspective. OneBody, Light Guide, MR Dance Trainer, Free Throw Simulator, Training Physical skills, Sleeve AR and Thai Chi Trainer use both visual perspectives. But only OneBody, LightGuide and TaiChi Trainer found a difference in the perspectives, and, furthermore did no investigation on how the visual perspective influenced the performance of the learner. This shows the necessity of investigations on the influence of the visual perspectives on virtual guidance visualisations for motor learning.\\
Another topic, where Motor Learning is a valuable helper are ergonomic condiction of movements \todo{physio at home, max papers?}. The correct handling of physical load in the correct ergonomic conduct in working routines can prevent injuries in everyday life. But teaching kinaestetics is not always accessible for example economic reasons. this field is also very low investigated. In addition, the translation through space in ego centric perspective is not existing.\\
This work adresses the lack of knowledge about the influence of visual perspectives on,  virtual guidance visualisaiton for motor learnign. thats why these research questions:

RQ1: Does the visual perspective on a virtual guidance visualisation influence motor learning in VR environments?\\
subs: 
\begin{itemize}
	\item[RQ1.1] Does the visual perspective on a virtual guidance visualisation influence the accuracy of movements?
	\begin{itemize}
		\item[RQ1.1.1] Does the visual perspective on a virtual guidance visualisation influence the accuracy of movements of the own Body?
		\item[RQ1.1.2] Does the visual perspective on a virtual guidance visualisation influence the accuracy of handling physical load?
		\item[RQ1.1.3] Are there sub-tasks that are influenced by the visual perspective on a virtual guidance visualisation?
	\end{itemize}
	\item[RQ1.2] Does the visual perspective on a virtual guidance visualisation influence the transfer of ergonomic principles?
	\item[RQ1.3] How the visual perspective on a virtual guidance visualisation influence the visual focus of the learner?
\end{itemize}
\cite{Muckell}
\section{Outline}
übersicht über dieses dokument
\exgo


Motivation: Motorlearning wichtig zur aneignung von bewegungen. am besten mit echtem lehrer. wenn dieser nicht verfügbar, motor learning in VR möglich und sinnvoll, siehe xyz. allerdings ist der einfluss der perspektive auf die virtuelle guidance vis. noch wenig untersucht. deswegen diese arbeit hier.\\
ferner, wenig motorlearning in zusammenhang mit physical load und wenig "laufen mit egozentrischer anleitung".\\
Daraus folgt die forschungsfrage ... und ihre sub forschungsfragen ...\\
Um daten zu generieren um diese forschungsfragen zu beantworten wurde Exgo entwickelt. Eine studie wurde designed um mit diesem system die notwendigen daten zu generieren.\\
Diese arbeit beschreibt design und entwicklung von exgo, sowohl als auch die entwicklung dieser studie. EIn pilottest und dessen überarbeitung ist angeschlossen. auch weiter forschungsmöglichkeiten mit dem system sollen aufgezeigt werden.\\
Forschungsfragen:\\