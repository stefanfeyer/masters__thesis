\chapter{Introduction}
The acquisition of movements is a crucial part of human development\cite{mlbook}. Learning movements empowers to be more efficient, faster or more exact (ibid.), for tasks like sports, arts or the ergonomic handling of physical load. The process of learning movements is called motor learning.\\
Movements can be learnerd by voyeurism and mimicking: watching and trying it out by yourself. Mastering a movement mostly includes an experienced teacher. A teacher is hardly replaceable because of immediate visual, audible and haptic feedback on a movement performed. However, if a teacher is not available, for example, based on the location or economic reasons, other sources can be used to learn movements. For example, YouTube\footnote{\href{https://www.youtube.com/}{https://www.youtube.com/}, accessed 17.2.2021} and other video platforms have become a great source for learning videos for a wide range of purposes. The downside of video recordings is the two dimensional (2D) experience of a three dimensional (3D) movement. Mixed Reality devices can transport the learning process into the digital world. In contrast to video recordings, Mixed Reality can provide the learning experience in 3D. Furthermore, Mixed Reality can provide feedback on the performed movement and has the ability to interactions with the virtual guidance visualisation. Mixed Reality has already proven to be a suitable environment for motor learning for tasks like dancing~\cite{YouMove,vrdancetrainer,outsideme,performancetraining,mrdancetrainer}, sports~\cite{freethrowsimulator,trainingphysicalskills}, rehabilitation~\cite{motionma,physioathome,kinohaptics,sleevear,veimprovesml}, arts~\cite{ararm,justfollowme,stylo,elearningma,mythaichicoaches,rtgesturerecognistion,onebody,thaichichua} and others~\cite{tikl,lightguide}. However, this master's thesis will focus on Virtual Reality.\\
In the real world, where the learner and teacher are real persons, the learner sees the teacher, for example, in front of himself/herself. This perspective is called the exo-centric visual perspective. Nevertheless, if we move from the real world to the virtual world of Virtual Reality, we are no longer restricted to the exo-centric visual perspective. The teacher can be rendered inside the learners's body, allowing the learner to see the teacher from an ego-centric perspective. The change from the exo-centric to the ego-centric visual perspective potentially influences motor learning, which is shown by previous research. For example, AR-Arm~\cite{ararm} lets the learner experience the movements from an ego-centric perspective. YouMove~\cite{YouMove} teaches dance from an exo-centric perspective. OneBody~\cite{onebody}, LightGuide~\cite{lightguide}, Mixed Reality Dance Trainer~\cite{mrdancetrainer}, Free Throw Simulator~\cite{freethrowsimulator}, Training Physical Skills~\cite{trainingphysicalskills}, Sleeve AR~\cite{sleevear} and Tai Chi Trainer~\cite{thaichichua} use both visual perspectives or a combination of them. However, only OneBody, LightGuide and Tai Chi Trainer found differences between the visual perspectives.\\
Another topic where Virtual Reality could be a valuable helper is the ergonomic conduction of movements while handling physical loads~\cite{nursecare,kitt}. The handling of physical load is part of working routines and the everyday life. Handling physical load in the correct ergonomic conduct in working routines can prevent injuries in everyday life. However, a kinaesthetics teacher is not always accessible, for example, for economic reasons. The influence of the visual perspective on a virtual guidance visualisation teaching the handling of physical load in Virtual Reality is sparsely investigated. Especially, locomotion movements like walking or carrying in the ego-centric perspective is left out. The lack of research on the influence of the visual perspective on a virtual guidance visualisation, especially for handling physical loads, shows the necessity of investigations on:
\begin{itemize}
	\item[RQ1:] How does the visual perspective on a virtual guidance visualisation influence motor learning in Virtual Reality environments?
\end{itemize}
To answer this main research question RQ1, several aspects have to be taken into account: accuracy of movements, transfer of information about how to move, the visual focus of the learner, and last but not least, the personal preference of the learner. Therefore, to answer the main research question RQ1, it is necessary to answer the following sub-research questions:
\begin{enumerate}[align=left, leftmargin=0pt, labelindent=\parindent,
	listparindent=\parindent, labelwidth=0pt, itemindent=!]
	\item[RQ1.1] How does the visual perspective on a virtual guidance visualisation influence movements' accuracy?
	\begin{itemize}
		\item[] \begin{itemize}
			\item[RQ1.1.1] How does the visual perspective on a virtual guidance visualisation influence movements' accuracy of the own body?
			\item[RQ1.1.2] How does the visual perspective on a virtual guidance visualisation influence the accuracy of handling physical load?
			\item[RQ1.1.3] How does the visual perspective on a virtual guidance visualisation influence subtasks' accuracy?
		\end{itemize}
	\end{itemize}
	
	\item[RQ1.2] Does the visual perspective on a virtual guidance visualisation influence the transfer of ergonomic principles?
	\item[RQ1.3] How does the visual perspective on a virtual guidance visualisation influence the learner's visual focus?
	\item[RQ1.4] What is the subjective personal preference of the learner for the visual perspectives?
\end{enumerate}
The answers to these research questions will enable designers of Virtual Reality motor learning training systems to choose a suitable visual perspective on an empirical basis.

\section{Outline}
This master's thesis aims to increase the empirical evidence of how the visual perspective influences motor learning in Virtual Reality by proposing an experiment. This document will present the development of an experiment and a system with which the experiment can produce this data.\\
First, the theoretical foundations are laid out and a state-of-the art analysis on the basis of related work is provided in chapter~\ref{chapter:theoretical_background}. With the fundamentals at hand, chapter~\ref{chapter:studysetting_conduction} describes in detailed the design of the proposed experiment. Chapter~\ref{chapter:studysetting_conduction} will adress the independent variables (\ref{sec:visualPerspecticves}) and dependent variables (\ref{sec:measures}) of the experiment as well the task design (\ref{sec:taskDesign}).\\
To conduct the experiment a system was implemented that fulfils the needs of the experiement. This system is called \exgo. The design and implementation of \exgo\ is pictured in chapter~\ref{chapter:system}. The experiment requires a digital representation of the learner (\ref{sec:selfperception}) and the guidance visualisation (\ref{sec:gv}). Futhermore, physical and digital artefacts the learner will interact with needed to be constructed and built physically and digitally (\ref{sec:artefacts}). The digital representations of learner and guidance visualisation as well as the artefacts are then arranged to form the visual perspectives (\ref{sec:perspectives}). Subsequently, the aquisition of the quantitative (\ref{sec:logging}) and qualitative (\ref{sec:quali_logging}) data is described. Finally, the elements of the experiment design and \exgo\ are composed and an experiment procedure is defined (\ref{sec:procedure}). The experiment was evaluated by a pilot study, the results and possible improvements are shown in chapter~\ref{chapter:study_evaluation}. In the end, this master's thesis closes with a conclusion and outlook in chapter~\ref{chapter:conclusion}. The appendix provides a Glossarium, additional figures and the experiment's documents.



%This work proposes a study design to answer the research question. To design this study on a solid basis, the theoretical foundations a laid in chapter~\ref{chapter:theoretical_background} with a closer look on Motor Learning (section~\ref{section:motor_learning}), visual perspectives (section~\ref{section:visual_perspectives}) and Mixed Reality (section \ref{section:mixed_reality}). These sections result in the scope and parameters of the study design. Section~\ref{section:related_work} investigates previous works and illustrates the conceptual delimitation of this work from what has already been investigated. Chapter~\ref{chapter:theoretical_background} concludes with a research contribution statement, clarifying the Empirical Contribution and Artifact Contribution of this work.\\
%For the proposed study, a system had to be designed to produce data to answer the research questions. This system is called \exgo\. The design and and implementation is described in section~\ref{section:system} followed by the design of the study itself in section~\ref{section:study}.\\
%\exgo\ and the study design have been evaluated in a pilot study. The results of the evaluation are depicted in chapter~\ref{chapter:study_evaluation}. Furthermore, this chapter suggests improvements in the study design in section~\ref{section:study_improvements}. This work concludes in chapter \ref{chapter:conclusion} with an outlook on how \exgo\ can be enhanced and expanded as well as used for further investigations.

%Motivation: Motorlearning wichtig zur aneignung von bewegungen. am besten mit echtem lehrer. wenn dieser nicht verfügbar, motor learning in VR möglich und sinnvoll, siehe xyz. allerdings ist der einfluss der perspektive auf die virtuelle guidance vis. noch wenig untersucht. deswegen diese arbeit hier.\\
%ferner, wenig motorlearning in zusammenhang mit physical load und wenig "laufen mit egozentrischer anleitung".\\
%Daraus folgt die forschungsfrage ... und ihre sub forschungsfragen ...\\
%Um daten zu generieren um diese forschungsfragen zu beantworten wurde Exgo entwickelt. Eine studie wurde designed um mit diesem system die notwendigen daten zu generieren.\\
%Diese arbeit beschreibt design und entwicklung von exgo, sowohl als auch die entwicklung dieser studie. EIn pilottest und dessen überarbeitung ist angeschlossen. auch weiter forschungsmöglichkeiten mit dem system sollen aufgezeigt werden.\\
